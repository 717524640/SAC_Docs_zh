\section{SAC内置函数\protect\footnotemark}
\footnotetext{本节的所有内容均可很方便的通过脚本实现,建议使用脚本而非SAC内置函数}
\SACTitle{概述}
内置函数是SAC内部实现的一些函数,其可以在SAC命令中使用,每个内置函数都需要置于小括号
内才可以使用。在命令执行之前内联函数首先被计算,其结果将代替该函数在命令中的位置。

有如下几类内联函数:
\begin{itemize}
\renewcommand\labelitemi{\dag}	% \usepackage{pifont}
\item 嵌入式算术运算函数: 以一个数字开头,函数名位于参数表中。
\item 常规算术运算函数: 以函数名开头,后跟零个或多个参数。
\item 字符串操作函数: 以函数名开头,后跟零个或多个参数
\end{itemize}
内置函数可以放在另一内置函数里,即支持嵌套,目前最多可以嵌套10层。宏参数
\footnote{在宏一节会说到,这里可以理解为一个变量}、暂存块变量、头段变量都可以用作内置函数的参数。

\SACTitle{嵌入式算术运算函数:}
嵌入式算术运算函数很像但不完全是一般的加减乘除操作。其一般形式如下:
\begin{SACCode}
       ( 数字 操作符 数字 ... ) 
\end{SACCode}
其中数字是数字值,操作符是下面的几个之一:
\begin{SACCode}
       +  -  *  /  ** 
\end{SACCode}

看几个简单的例子:
\begin{SACCode}
SAC> echo on
SAC> setbb a (4 + 7 / 3)
 setbb a (4 + 7 / 3)
 ==> setbb a 3.666667
SAC> setbb b (4+7/3)
 setbb b (4+7/3)
 ERROR 1020: Invalid inline function name: 4+7/3
\end{SACCode}
注意当ECHO打开时,不论在命令的何处出现了内置函数,SAC都会将处理之后的结果发送到终端。
这就允许你查看命令是否正确的执行了。
上面的例子中两次使用到嵌入式内置函数,一个正常运行,一个报错。这个例子显示出了SAC
嵌入式内置函数与我们常见的C或FORTRAN算术表达式之间的差异:
\begin{itemize}
\renewcommand\labelitemi{\dag}	% \usepackage{pifont}
\item 所有的数都被认为是实型,算术运算按照浮点型运算进行;
\item 操作符没有优先级的问题,计算按照从左到右的顺序进行\footnote{这就是结果为3.66667
	而非6.33333的原因};
\item 内联函数两端的小括号是必须的;
\item 操作符左右必须有空格。
\end{itemize}
内联函数可以通过嵌套改变计算顺序:
\begin{SACCode}
SAC> SETBB A (4 + (7 / 3))
 ==> SETBB A 6.333333
\end{SACCode}
与基本数学或编程语言类似,我们可以通过加入小括号来改变计算顺序,在SAC中由于小括号是
内置函数的标识符,所以这种方式在SAC中称为内置函数的嵌套。需要注意在加号和第二个
左括号之间有空格,这对于SAC正确执行命令很重要。最好的办法是在所有的参数、操作符
以及嵌套的小括号两端都加上空格。(出现这个问题的原因在于SAC在算术计算方面的语法
分析问题,如何正确的识别数字操作符等等是一个很复杂的问题,对于SAC算术计算不是
重头戏,所以在这个地方花费太多的精力没有那么大的必要,用户习惯了就好)。

\SACTitle{常规算术运算函数:}
目前有20个常规算术运算函数,他们对应于EVALUATE命令中的各种算术函数。他们大多比较容易
理解和记忆。
\begin{table}[h]
\caption{常规算数运算函数}
\centering
\begin{tabular}{l|l|l}
	\toprule
	命令	&	语法	&	功能	\\
	\midrule
	ADD		&	( ADD v1 v2 ... vn )	&	Add (sum) a set of numbers.	\\
	SINE	&	( SINE v )				&	Take the sine of a number.v为弧度\\
	SUBTRACT&	( SUBTRACT v1 v2 ... vn)&	Subtract a set of numbers.	\\
	ARCSINE	&	( ARCSINE v)			&	Take the arcsine of a number	\\
	MULTIPLY&	( MULTIPLY v1 v2 ... vn)&	Multiply a set of numbers	\\
	COSINE	&	(COSINE v)				&	Take the cosine of a number.	\\
	DIVIDE	&	( DIVIDE v1 v2 ... vn)	&	Divide a set of numbers	\\
	ARCCOSINE&	( ARCCOSINE v) 			&	Take the arccosine of a number	\\
	SQRT	&	( SQRT v) 				&	Take the square root of a number	\\
	TANGENT	&	( TANGENT v)			&	Take the tangent of a number.	\\
	EXP 	&	( EXP v)				&	Exponentiate a number.	\\
	ARCTANGENT&	( ARCTANGENT v)			&	Take the arctangent of a number.	\\
	ALOG	&	( ALOG v)				&	Take the natural logarithm of a number	\\
	INTEGER &	( INTEGER v)			&	Convert a number to an integer.	\\
	POWER	&	( POWER v)				&	Raise a number to its power of 10	\\
	PI		&	( PI )					&	Return the value of pi	\\
	ALOG10	&	( ALOG10 v)				&	Take the log to base 10 of a number	\\
	MAXIMUM	&	( MAXIMUM v1 v2 ... vn)	&	Maximum value of a set of numbers	\\
	MINIMUM	&	( MINIMUM v1 v2 ... vn)	&	Minimum value of a set of numbers	\\
	ABSOLUTE&	( ABSOLUTE v)			&	Take the absolute value of a number	\\
	\bottomrule
\end{tabular}
\end{table}

下面来看几个例子。为了归一化一组数据首先要找到所有数据中的绝对最大值:
\begin{SACCode}
SAC> r file1 file2 file3 file4
SAC> setbb a (max &1,depmax &2,depmax &3,depmax &4,depmax)
 ==> SETBB A 1.87324
SAC> setbb b (min &1,depmin &2,depmin &3,depmin &4,depmin)
 ==> setbb b -2.123371
SAC> div (max %A (abs %B))
 ==> DIV 2.123371 
\end{SACCode}
这个本可以在单个命令中完成而不必使用暂存块变量,只要通过正确的嵌套内联函数即可,
不过上面的写法可读性更强。

在下面的例子中将要计算一个以度为单位的角的正切值:
\begin{SACCode}
SAC> setbb angle 45.0
 ==> setbb angle 45.000
SAC> setbb value (tan (divide (multiply (pi) %angle%) 180.))
 ==> setbb value 1.00000 
\end{SACCode}
将函数名作为第一个参数使得SAC容易识别这个函数,但是在算术运算是读起来很困难,
因此可以通过将嵌入式算术函数和普通算术函数结合重新改写上述命令:
\begin{SACCode}
SAC> setbb value (tan ((pi) * %angle / 180. ))
 ==> setbb value 1.00000 
\end{SACCode}
思考一下为什么ANGLE在第一个例子中中后面有一个\%而第二个例子中却没有呢?

\SACTitle{其他算术运算函数:}
这类函数目前只有一个:GETTIME,用于返回数据中首先出现某个值的时间相对于文件开始
时间的时间偏移量(s):
\begin{SACCode}
   (GETTIME MAX|MIN [value])
\end{SACCode}
返回内存中第一个文件中数值为value的数据点的时间(以秒为单位),如果没有指定value,
MAX为文件中第一个大于或等于DEPMAX的数据点的时间;MIN为文件中第一个小于或等于DEPMIN
的数据值点的时间。指定value是为了控制正在查找的数据点的数值。

对于所有的文件有一个最大振幅,要找到这些文件中第一个文件中第一次大于该值所对应的时
间偏移量:
\begin{SACCode}
SAC> r FILE1 FILE2 FILE3 FILE4
SAC> setbb maxtime ( gettime max )
 ==> setbb maxtime 41.87
\end{SACCode}

为了找到一个小于或等于123.45的数据点的时间偏移,可以使用如下命令:
\begin{SACCode}
SAC> setbb valuetime ( gettime min 123.45 )
 ==> setbb valuatime 37.9 
\end{SACCode}

\begin{table}[h]
\caption{字符串函数}
\centering
\begin{tabular}{p{2.73cm}p{4.88cm}p{7.1cm}}
	\toprule
	命令	&	语法	&	功能	\\
	\midrule
	CHANGE	&	(CHANGE) s1 s2 s3) 	&	在字符串s3中用字符串s1代替字符串s2	\\
	SUBSTRING&	(SUBSTRING n1 n2 s) &	得到字符串s中第n1到第n2个字符\\
	DELETE	&	(DELETE s1 s2)		&	从字符串s2中删去字符串s1	\\
	CONCATENATE&	(CONCATENATE s1 s2 ... sn)	&	首尾连接一个或多个字符串\\
	BEFORE	&	(BEFORE s1 s2)	&	得到字符串s2中字符串s1前的部分字符串\\
	REPLY	&	(REPLY s1)	&	发送信息到终端并得到回应	\\
	AFTER	&	(AFTER s1 s2)	&	得到字符串s2中字符串s1后的部分字符串\\
	\bottomrule
\end{tabular}
\end{table}

下面的例子展示了上面部分函数的用法,在标题中使用台站和文件名然后绘图(比较一下两次绘图的区别):
\begin{SACCode}
SAC> fg seis
SAC> p
SAC> echo on
SAC> title '(concatenate 'Seismogram of ' &1,kevnm ' ' &1,kstnm )'
 ==> title 'Seismogram of K8108838 CDV'
SAC> p
\end{SACCode}
这个例子中CONCATENATE有四个参数,第一个参数是'Seismogram of ',因为其中间有空格所以
需要用引号引起来;第二、四个参数利用了文件中的头段变量;第三个参数是一个空格以保证
台站名和事件名不会连起来。最外面一对引号是title命令的语法要求的。

下面的例子使用SUBSTRING从事件中提取月份信息:
\begin{SACCode}
SAC> setbb month (substring 1 3 '&1,kzdate&')
 ==> setbb month mar
\end{SACCode}
为什么头段变量KZDATE需要加引号?

下面的例子使用REPLY实现交互\footnote{这个例子是为了演示REPLY的用法,但内容上已经涉及
到宏的部分,但是有编程基础的话应该很容易看懂,不过这个例子不太好用作实验,看看就行}:
\begin{SACCode}
SAC>DO FILE LIST ABC DEF XYZ
SAC>READ $FILE
SAC>DO J FROM 1 TO 10
SAC> MACRO PROCESSFILE
SAC> PLOT
SAC> setbb resp (reply "Enter -1 to stop,0 for next file, 1 for same file:")
SAC> IF %resp LE 0 THEN
SAC>  BREAK
SAC> ENDIF
SAC>ENDDO
SAC>IF %resp LT 0 THEN
SAC>  BREAK
SAC>ENDIF
SAC>ENDDO 
\end{SACCode}
外循环从列表中一次读取一个文件,内循环调用宏文件处理这个文件,内循环执行10次。每次
执行之后,绘图然后REPLY发送信息到终端,并接收用户的输入以决定是否继续执行。

下面这个例子展示的REPLY有一个缺省值:
\begin{SACCode}
SAC> setbb bbday (reply "Enter the day of the week: [Monday]") 
\end{SACCode}
当这个函数执行时,引号中的字符串将出现在屏幕上,提示用户输入。如果用户输入,SAC会将
输入的字符串作为回应值,如果用户只是敲击回车键,SAC则会使用该默认值即MONDAY。
