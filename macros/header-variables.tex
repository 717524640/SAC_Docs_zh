\section{引用头段变量值}
前面已经介绍了SAC中的很多头段变量,也知道如何使用~\nameref{cmd:listhdr}~查看头段
变量的值,lh命令的输出对于人来说很直观,但是对于机器来说却很不友好。有些时候需要
直接使用头段变量的值,这就需要一些特殊的技巧。

最常见的情况是第~\pageref{code:origin-time}~页给出的例子。在使用``ch o gmt''指定发震
时刻后,需要获取头段变量o的值,对该值取负值,并用于``ch allt''中。

在这个例子中,我们需要知道头段变量o的值,并将其值用于其它命令中,准确的说这叫变量
值的引用。在SAC命令中引用SAC头段变量的值有两种方式,分别是~``\lstinline{&fname,header&}''~
和~``\lstinline{&fno,header&}''~
\footnote{实际上,SAC官方文档给出的引用方式中没有末尾的~\lstinline{&}~符号,仅当
一些特殊的情况下才使用,这样容易使得整个语法混乱不堪,所以这里采用了另外一种引用
方式。所有示例均已通过测试。}
。

fname和fno都唯一指向了内存中的某个波形数据,其中fname表示文件名,fno表示文件号
(即内存中的第几个文件,索引值从1开始),header则为头段变量名。

下例展示了如何通过两种方式引用头段变量的值:
\begin{SACCode}
SAC> fg seis
SAC> w seis.SAC
SAC> r ./seis.SAC               // 注意"./"
SAC> lh kevnm o stla            // 查看三个头段变量的值

  FILE: ./seis.SAC - 1          // 这里给出了文件名和文件号
 ----------------

     kevnm = K8108838
         o = -4.143000e+01
      stla = 4.800000e+01
SAC> echo on processed          // 打开回显,显示处理信息
SAC> ch kuser0 &1,kevnm&        // 通过文件号引用头段变量kevnm
 ==>  ch kuser0 K8108838        // 实际执行的效果
SAC> ch user0 &./seis.SAC,o&    // 利用文件名,引用头段变量o
 ==>  ch user0 -41.43
SAC> ch user1 &seis.SAC,stla&   // 文件名少了"./"
 ERROR 1363: Illegal data file list name: seis.SAC
SAC> lh kuser0 user0 user1

  FILE: ./seis.SAC - 1
   ----------------

     kuser0 = K8108838
     user0 = -4.143000e+01
\end{SACCode}

在通过文件名指定波形数据时要注意:SAC记录的是文件的全路径。一般情况下,使用
文件号会更方便些。
