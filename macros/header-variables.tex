\section{SAC头段变量的使用}
    
这里也许该顺势去写在C或FORTRAN中读写暂存块变量的,仔细
想想还是把头段变量的使用放在这里会比较好。两个变量离的比较近,对比记忆起来也许更容易。

SAC的头段变量也可以像暂存块变量一样用于计算并直接代入到其他命令中,当然你必须指定是哪一个
文件(可以使用文件名或者文件号)的哪一个头段变量将用于计算。

你需要在这些标识之前加上一个``\&''以表明这是一个头段变量\footnote{记得暂存块变量需要的符号是``\%''},
同时需要在文件名和变量名之间加上一个逗号\footnote{暂存块变量与头段变量的一个区别!},如下所示:
\begin{SACCode}
SAC> fg seis
SAC> w seis
SAC> r seis
SAC> lh a depmax                            //看看头段
   
  FILE: seis - 1
 ----------

          a = 1.046400e+01
     depmax = 1.520640e+00
SAC> echo on
SAC> evaluate to temp1 &seis,A + 10         //利用文件名获取头段变量值
 evaluate to temp1 &seis,A + 10
 ==> evaluate to temp1 1.046400e+01 + 10
SAC> evaluate to temp2 &1,DEPMAX * 2        //利用文件号获取头段变量值
 evaluate to temp2 &1,DEPMAX * 2
 ==> evaluate to temp2 1.520640e+00 * 2
SAC> ch t5 %temp1                           
 ch t5 %temp1
 ==> ch t5 2.0464001e+01
SAC> ch user0 %temp2
 ch user0 %temp2
 ==> ch user0 3.0412800e+00
SAC> lh t5 user0                            //看看最后的结果
 lh t5 user0
  
  FILE: seis - 1
 ----------

        t5 = 2.046400e+01
     user0 = 3.041280e+00
\end{SACCode}
在上面的例子中文件被读入,一些暂存块变量通过文件的头段变量被计算,
第一个文件头段是用文件名来指定的,第二个文件头段是用文件号来指定的。然后新的头段
变量通过这些暂存块变量被重新定义。

下面将上面的例子改一下:
\begin{SACCode}
SAC> r ./seis 
SAC> lh a 
  
  FILE: ./seis - 1
 ------------

     a = 1.046400e+01
SAC> echo on
SAC> evaluate to temp1 &seis,A + 10
 evaluate to temp1 &seis,A + 10
 ==> evaluate to temp1  + 10
SAC> evaluate to temp2 &1,A + 10
 evaluate to temp2 &1,A + 10
 ==> evaluate to temp2 1.046400e+01 + 10
\end{SACCode}
没看出区别的话可以看看脚注。\footnote{bash的一个重要特性是命令补全,在命令补全是读取的文件名是./seis,
所以引用的时候用\&seis,A就不行了,用文件号是可以的。你可以试一下\&./seis,A行不行。建议还是用文件号吧}
同暂存块变量一样,也可以在头段变量的前后加上字符串,在头段变量前加字符串时可以直接加,在头段变量后联接
字符串的话就需要一个``\&''来作为分隔符,原理同上。假设文件Z的头段变量KA值为IPU0,那么(\&Z,KA\&)的值就是(IPU0)
(这里在头段变量前后分别添加了半个括号。)。
