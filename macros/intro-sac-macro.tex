\section{宏简介}
\label{sec:intro-to-macros}
\subsection{概述}
最简单的,将一系列要一起执行的SAC命令放在一个文件中即构成了SAC宏。就像一般命令
以及内联函数一样,SAC宏文件可以包含头段变量以及暂存块变量,在命令被执行之前这
些变量或者表达式的值将首先被计算出来。

\subsection{简单的例子}
假设你有一系列要重复执行多次的命令,那么SAC宏文件显然可以帮你节省不少时间。只要
打开你喜欢的文本编辑器,把这些命令放在一个文件中,然后你就可以用命令MACRO来执行
这个文件中的全部命令。假设你要读取重复读取相同的三个文件,文件名分别为ABC、DEF、XYZ,
每一个文件分别乘以不同的值,做Fourier变换,然后将频谱的振幅部分绘制到SGF文件中,
这样的一个宏文件如下所示:
\begin{SACCode}
  ** This certainly is a simple little macro.
  r ABC DEF XYZ
  mul 4 8 9
  fft
  bg sgf
  psp am
\end{SACCode}

假设这个文件命名为MYSTUFF,并且该文件在你的当前目录中,你可以通过下面的命令执行宏文件:
\begin{SACCode}
SAC> macro MYSTUFF
\end{SACCode}
注意到宏文件中的命令在执行时是不会在终端中显示该命令的,可以使用ECHO来根据你的意愿
选择在终端显示哪些东西。另外,在任意一行的第一列若有星号则意味着该行为注释行,
SAC不会不解释该行。

\SACTitle{宏搜索路径:}
当你执行一个宏而又没有给出宏文件的路径时,SAC会按照下面的顺序搜索宏文件:
\begin{itemize}
\item 在当前目录搜索
\item 在SETMACRO命令设置的搜索目录中搜索
\item 在SAC的全局宏目录中搜索
\end{itemize}

全局宏目录包含了对系统上的所有人都可以使用的宏文件,这个全局宏目录一般是sac/aux/macros。
可以使用INSTALLMACRO命令将自己的宏文件安装到这个目录中。你也可以通过绝对或相对路径来专门
指定搜索路径。