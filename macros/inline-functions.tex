\section{内联函数}

内置函数是SAC实现的一些函数,其可以在SAC命令中使用。在执行命令时,内联函数会首先被
调用,内联函数的结果将替代命令中的内联函数的位置。

SAC提供了如下几类内联函数:
\begin{itemize}
\item 算术运算符;
\item 常规算术运算函数;
\item 字符串操作函数;
\end{itemize}

所有的内联函数的共同形式是:\lstinline{(func)},其中func为内联函数名,在某种程度
上内联函数与前面说到头段变量(\lstinline{& &})和黑板变量(\lstinline)类似,
可以认为是通过\lstinline{( )}引用了内联函数的结果或值。

内联函数支持嵌套,目前最多可以嵌套10层。

\subsection{算术运算符}
算术运算符即常规的加减乘除运算符,但又有不同,其一般形式如下:
\begin{SACCode}
    ( number operator number ) 
\end{SACCode}
所有的操作数都被认为是实型的,所有的算术运算都按照双精度浮点型进行运算;

SAC支持的操作符是包括:\lstinline{ +  -  *  /  ** }

看几个简单的例子:
\begin{SACCode}
SAC> echo on                
SAC> setbb var1 4+7             // 忘记加括号了!"4+7"被当成了字符串
 setbb var1 4+7
SAC> setbb var2 (4+7)           
 setbb var2 (4+7)
 ==>  setbb var2 11             // 4+7=11
SAC> setbb var3 (4+7/3)         // 优先级正确
 setbb var3 (4+7/3)
 ==>  setbb var3 6.33333
SAC> setbb var4 ((4+7)/3)       // 括号改变优先级
 setbb var4 ((4+7)/3)           // 可以看作是内联函数的嵌套
 ==>  setbb var4 3.66667
SAC> setbb var1 ( ( 4 + 7 ) / 3 )   // 支持空格
 setbb var1 ( ( 4 + 7 ) / 3 )
 ==>  setbb var1 3.66667
\end{SACCode}

\subsection{常规算术运算函数}
SAC提供了20个常规算术运算函数,其基本形式为\lstinline{(func arg1 arg2 ...)}。

\begin{table}[H]
\centering
\caption{常规算数运算函数}
\label{table:regular-arithmetic-functions}
\begin{tabular}{l|l|l}
	\toprule
	命令	&	语法	&	功能	\\
	\midrule
	add		&	( add v1 v2 ... vn )	    &	v1+v2+...+vn	\\
	subtract&	( subtract v1 v2 ... vn )   &	v1-v2-...-vn	\\
	multiply&	( multiply v1 v2 ... vn )   &	v1*v2*...*vn	\\
	divide	&	( divide v1 v2 ... vn )	    &	v1/v2/.../vn	\\
	absolute&	( absolute v )			    &	取绝对值	\\
	power	&	( power v )				    &	取10的v次方     \\
	alog10	&	( alog10 v)				    &	以10为底取v的对数	\\
	alog	&	( alog v)				    &	取v的自然对数	\\
	exp	    &	( exp v)				    &	取e的v次方	\\
	sqrt	&	( sqrt v) 				    &	求v的平方根	\\
	pi		&	( pi )					    &	返回pi值	\\
    sine    &	( sine v )				    &	正弦(v为弧度,下同)\\
	cosine	&	( cosine v )				&	余弦	\\
    tangent	&	( tangent v )			    &	正切	\\
	arcsine	&	( arcsine v )			    &	反正弦	\\
	arccosine&	( arccosine v ) 			&	反余弦	\\
	arctangent&	( arctangent v )			&	反正切	\\
    integer &	( integer v )			    &	取整	\\
	maximum	&	( maximum v1 v2 ... vn )	&	求最大值	\\
	minimum	&	( minimum v1 v2 ... vn )	&	求最小值	\\
	\bottomrule
\end{tabular}
\end{table}

演示如下:
\begin{SACCode}
SAC> echo on processed
SAC> setbb var1 (add 1 3 4)         // 1+3+4
 ==>  setbb var1 8
SAC> setbb var2 (subtract 1 3 4)    // 1-3-4
 ==>  setbb var2 -6
SAC> setbb var3 (multiply 1 3 4)    // 1*3*4
 ==>  setbb var3 12
SAC> setbb var4 (divide 1 3 4)      // 1/3/4
 ==>  setbb var4 0.0833333
SAC> setbb var5 ( absolute -5.1 )   // abs(-5.1)
 ==>  setbb var5 5.1
SAC> setbb var6 ( power 5 )         // 10^5
 ==>  setbb var6 100000
SAC> setbb var7 ( alog10 10000 )    // log10(10000)
 ==>  setbb var7 4
SAC> setbb var8 ( alog 10000 )      // ln(10000)
 ==>  setbb var8 9.21034
SAC> setbb var9 ( exp 5 )           // e^5
 ==>  setbb var9 148.413
SAC> setbb var10 ( sqrt 9 )         // sqrt(9)
 ==>  setbb var10 3
SAC> setbb var11 ( pi )             // PI
 ==>  setbb var11 3.14159
 SAC> setbb var12 ( sine (pi/6) )   // sin(30)
 ==>  setbb var12 0.5
SAC> setbb var13 ((arcsine 0.5)*180/(pi))
 ==>  setbb var13 30
SAC> setbb var14 (integer 3.11)
 ==>  setbb var14 3
SAC> setbb var15 (max 3.11 -1.5 5)  // maximum简写为max
 ==>  setbb var15 5
SAC> setbb var16 (min 3.11 -1.5 5)  // minimum简写为min
 ==>  setbb var16 -1.5
\end{SACCode}

为了对一组数据做归一化,首先要找到所有数据中的绝对最大值,如下:
\begin{SACCode}
SAC> r file1 file2 file3 file4
SAC> echo on processed
SAC> setbb vmax (max &1,depmax& &2,depmax& &3,depmax& &4,depmax&)
 ==> setbb vmax 1.87324
SAC> setbb vmin (min &1,depmin& &2,depmin& &3,depmin& &4,depmin&)
 ==> setbb vmin -2.123371
SAC> div ( max (abs %vmax%) (abs %vmin%) )      // 嵌套
 ==>  div 2.123371 
\end{SACCode}
此例可以通过多重嵌套的方式在单个命令中完成,但上面的写法可读性更强。

\subsection{其他算术运算函数}
这类函数目前只有一个:GETTIME,用于返回数据中首先出现某个值的时间相对于文件开始
时间的时间偏移量(s):
\begin{SACCode}
   (GETTIME MAX|MIN [value])
\end{SACCode}
返回内存中第一个文件中数值为value的数据点的时间(以秒为单位),如果没有指定value,
MAX为文件中第一个大于或等于DEPMAX的数据点的时间;MIN为文件中第一个小于或等于DEPMIN
的数据值点的时间。指定value是为了控制正在查找的数据点的数值。

对于所有的文件有一个最大振幅,要找到这些文件中第一个文件中第一次大于该值所对应的时
间偏移量:
\begin{SACCode}
SAC> r FILE1 FILE2 FILE3 FILE4
SAC> setbb maxtime ( gettime max )
 ==> setbb maxtime 41.87
\end{SACCode}

为了找到一个小于或等于123.45的数据点的时间偏移,可以使用如下命令:
\begin{SACCode}
SAC> setbb valuetime ( gettime min 123.45 )
 ==> setbb valuatime 37.9 
\end{SACCode}

\begin{table}[h]
\caption{字符串函数}
\centering
\begin{tabular}{p{2.73cm}p{4.88cm}p{7.1cm}}
	\toprule
	命令	&	语法	&	功能	\\
	\midrule
	CHANGE	&	(CHANGE) s1 s2 s3) 	&	在字符串s3中用字符串s1代替字符串s2	\\
	SUBSTRING&	(SUBSTRING n1 n2 s) &	得到字符串s中第n1到第n2个字符\\
	DELETE	&	(DELETE s1 s2)		&	从字符串s2中删去字符串s1	\\
	CONCATENATE&	(CONCATENATE s1 s2 ... sn)	&	首尾连接一个或多个字符串\\
	BEFORE	&	(BEFORE s1 s2)	&	得到字符串s2中字符串s1前的部分字符串\\
	REPLY	&	(REPLY s1)	&	发送信息到终端并得到回应	\\
	AFTER	&	(AFTER s1 s2)	&	得到字符串s2中字符串s1后的部分字符串\\
	\bottomrule
\end{tabular}
\end{table}

下面的例子展示了上面部分函数的用法,在标题中使用台站和文件名然后绘图(比较一下两次绘图的区别):
\begin{SACCode}
SAC> fg seis
SAC> p
SAC> echo on
SAC> title '(concatenate 'Seismogram of ' &1,kevnm ' ' &1,kstnm )'
 ==> title 'Seismogram of K8108838 CDV'
SAC> p
\end{SACCode}
这个例子中CONCATENATE有四个参数,第一个参数是'Seismogram of ',因为其中间有空格所以
需要用引号引起来;第二、四个参数利用了文件中的头段变量;第三个参数是一个空格以保证
台站名和事件名不会连起来。最外面一对引号是title命令的语法要求的。

下面的例子使用SUBSTRING从事件中提取月份信息:
\begin{SACCode}
SAC> setbb month (substring 1 3 '&1,kzdate&')
 ==> setbb month mar
\end{SACCode}
为什么头段变量KZDATE需要加引号?

下面的例子使用REPLY实现交互\footnote{这个例子是为了演示REPLY的用法,但内容上已经涉及
到宏的部分,但是有编程基础的话应该很容易看懂,不过这个例子不太好用作实验,看看就行}:
\begin{SACCode}
SAC>DO FILE LIST ABC DEF XYZ
SAC>READ $FILE
SAC>DO J FROM 1 TO 10
SAC> MACRO PROCESSFILE
SAC> PLOT
SAC> setbb resp (reply "Enter -1 to stop,0 for next file, 1 for same file:")
SAC> IF %resp LE 0 THEN
SAC>  BREAK
SAC> ENDIF
SAC>ENDDO
SAC>IF %resp LT 0 THEN
SAC>  BREAK
SAC>ENDIF
SAC>ENDDO 
\end{SACCode}
外循环从列表中一次读取一个文件,内循环调用宏文件处理这个文件,内循环执行10次。每次
执行之后,绘图然后REPLY发送信息到终端,并接收用户的输入以决定是否继续执行。

下面这个例子展示的REPLY有一个缺省值:
\begin{SACCode}
SAC> setbb bbday (reply "Enter the day of the week: [Monday]") 
\end{SACCode}
当这个函数执行时,引号中的字符串将出现在屏幕上,提示用户输入。如果用户输入,SAC会将
输入的字符串作为回应值,如果用户只是敲击回车键,SAC则会使用该默认值即MONDAY。
