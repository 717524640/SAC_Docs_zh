\section{if语句}
\SACTitle{If Tests:}
这个特性让你能够在宏文件中修改命令的执行顺序,其语法与fortran77很相似但是又不完全
相同,要注意区分。
\begin{SACCode}
  IF expr
  	commands
  ELSEIF expr
  	commands
  ELSE
  	commands
  ENDIF
\end{SACCode}
在上面的语句中expr是一个如下形式的逻辑表达式:
\begin{SACCode}
           token 操作符 token
\end{SACCode}
其中token可以是一个常数、宏参数、暂存块变量或者头段变量,操作符则是GT、GE、LE、LT、EQ、NE
中的一个。上面的逻辑表达式在计算之前token会被转换为浮点型数。if语句目前最多允许嵌套10次,
elseif、else是可选的,elseif的次数没有限制。

注意这里的if语句与fortran77中if语句的区别:
在逻辑表达式两边没有小括号,并且没有关键字then!下面给出一个例子:
\begin{SACCode}
  READ $1
  MARKPTP
  IF &1,USER0 GE 2.45
    FFT
    PLOTSP AM
  ELSE
    MESSAGE "Peak to peak for $1 below threshold."
  ENDIF
\end{SACCode}
在这个例子中,一个文件被读入内存,测出其最大峰峰值,如果这个值大于某一确定值,
则对其做Fourier变换并绘制振幅图,否则则输出信息到终端。
