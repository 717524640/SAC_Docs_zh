这一章的内容一直没有想好怎么写,姑且先放在这里,等等再说。

SAC自身已经提供了一系列用于读写SAC文件的子函数,但是函数功能过于单一。
比如读函数只能一次性读取整个文件,无法只读取
文件的一部分(即没有截窗的功能);再比如,想要获取多个头段变量的值,必须
多次调用相应的子函数。

在理解了SAC的内部结构之后,可以完全自定义SAC I/O函数库。

Prof. Lupei Zhu实现了一个比较方便的SAC I/O函数库,可以直接使用或者作为参考。
相关地址为:\url{http://www.eas.slu.edu/People/LZhu/downloads/pssac.tar}。

另一方面,我自己正在计划重写SAC I/O函数库以及相应的一些辅助工具。重写SAC I/O
函数库的原因在于Prof. Lupei Zhu的I/O库中存在一些潜在的Bug以及考虑不周之处,并且
现有的诸如saclst等辅助工具功能尚有欠缺。

项目主页位于:\url{https://github.com/seisman/sac_tools}。
