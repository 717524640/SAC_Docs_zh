\section{在Perl中调用SAC}
\label{sec:sac-perl}

\subsection{简介}
下面的脚本中给出了一个简单的例子,展示了如何在Perl中调用SAC。相对于Bash来说,Perl脚本似乎
需要更多的键入,但是相对于Perl的优势来说,这些都不算什么。
\lstinputlisting[style=Perl]{./call-in-script/simple-script.pl}

\subsection{头段变量}
Perl无法直接引用SAC文件的头段变量值,依然需要利用SAC宏的的语法。Perl的优势在于可以
一边向SAC传递信息,一边运行自身的命令。
\lstinputlisting[style=Perl]{./call-in-script/variables.pl}
本例中第9行,在SAC运行的同时Perl临时定义了一个变量,并成功将其传递给了SAC,这在
Bash中是不容易做到的。

\subsection{内联函数}
Perl可以完成各种复杂的数学运算:
\lstinputlisting[style=Perl]{./call-in-script/arithmetic-functions.pl}

Perl对字符串的处理更是Perl的杀手锏:
\lstinputlisting[style=Perl]{./call-in-script/string-functions.pl}

\subsection{条件判断和循环控制}
Perl也可以很容易地实现条件判断和循环控制:
\lstinputlisting[style=Perl]{./call-in-script/do-loops.pl}
这个例子中只启动了一次SAC,然后开始Perl的循环控制,读取当前目录下的每一个SAC文件,
做一些数据处理,然后写到新文件中。

该Perl脚本与上一个Bash脚本相比,实现了几乎相同的功能,但Perl脚本中仅启动和退出SAC
一次,与Bash脚本的多次启动相比,其效率要高很多。
