\SACCMD{ADDSTACK}
\label{sss:addstack}

\SACTitle{概要}
向迭加文件列表中加入新文件

\SACTitle{语法}
\begin{SACSTX}
    A!DD!S!TACK! filename [W!EIGHT! v] [DI!STANCE! v]
        [BE!GINTIME! v] [END!TIME! v]
        [DE!LAY! v [S!ECONDS!|P!OINTS!]]
        [I!NCREMENT! v [S!ECONDS!|P!OINTS!]]
        [N!ORMAL!|R!EVERSE!]
\end{SACSTX}

\SACTitle{输入}
\begin{description}
\item [filename] 要加入迭加文件列表中的文件
\item [WEIGHT v] 当前文件的权重因子。v的取值范围为0到1,在迭加之前会首先对文件的每个值乘以该权重因子再做迭加。
\item [DISTANCE v] 该文件所对应的震中距,单位为km。用于计算动态时间延迟
\item [BEGINTIME v] 事件开始的时间
\item [ENDTIME] 事件结束时间
\item [DELAY v SECONDS|POINTS] 该文件的静态时间延迟,单位为秒或数据点数
\item [INCREMENT v SECONDS|POINTS] 该文件的静态时间延迟增量,单位为秒或数据点数。在每次执行incrementstack命令时,静态时间延迟会增加一个常数。
\item [NORMAL|REVERSED] 文件拥有正/负极性
\end{description}

\SACTitle{缺省值}
DELAY和INCREMENT选型的缺省单位为SECONDS

\SACTitle{说明}
每个迭加列表中的文件,都有7个属性,分别为:
\begin{enumerate}
\item 权重因子;
\item 台站到震中的距离;
\item 事件开始时间;
\item 事件结束时间;
\item 静态时间延迟;
\item 静态时间延迟增量;
\item 数据极性;
\end{enumerate}

可以用~\nameref{sss:globalstack}~命令为这些迭加属性设置全局值。当一个文件通过
~\nameref{sss:addstack}~命令加入到迭加文件列表中时,若未指定属性值则使用全局属性值。
~\nameref{sss:changestack}~可以用于在文件已经加入迭加文件列表后修改迭加属性值。

\SACTitle{示例}
\begin{SACCode}
SAC/SSS> gs delay 1.0 inc 0.03
SAC/SSS> as filea delay 2.0
SAC/SSS> as fileb delay 3.0 inc 0.01 rev
SAC/SSS> as filec
SAC/SSS> as filed w 0.5
\end{SACCode}

第一个命令修改了时间延迟和时间延迟增量的全局属性值,其他全局属性则使用缺省值。
\begin{itemize}
\item filea:除时间延迟外其他属性均与全局属性相同;
\item fileb:除时间延迟、时间延迟增量、信号极性外,其余属性均与全局属性相同;
\item filec:所有属性与全局熟悉相同;
\item filed:除权重因子外,其他所有属性与全局属性相同;
\end{itemize}

接下来对信号进行迭加:
\begin{SACCode}
SAC/SSS> sumstack
\end{SACCode}

该命令会讲迭加文件列表中的四个文件filea、fileb、filec和filed进行迭加,时间延迟分别为2.0、3.0、1.0和1.0。文件filec的极性反转。文件filed在迭加时的权重是其他文件权重的一半。

\begin{SACCode}
SAC/SSS> incrementstack
SAC/SSS> changestack filec normal
SAC/SSS> sumstack
\end{SACCode}

此次迭加,各个文件使用2.03、3.01、1.03和1.03的延迟。文件filec现在为正极性。

\begin{SACCode}
SAC/SSS> deletestack filed
SAC/SSS> incrementstack
SAC/SSS> sumstack
\end{SACCode}

第三次迭加讲只对文件filea、fileb、filec进行,时间延迟分别为2.06、3.02、1.06。

\SACTitle{错误消息}
\begin{itemize}
\item 5108:超过迭加文件列表的最大长度
\item 1306:不等间隔数据的非法操作
\item 1307:谱文件的非法操作
\item 5109:采样间隔不相等
\end{itemize}

\SACTitle{限制}
迭加文件列表中文件数目的最大限制与SAC所能读取的文件数目一致,即最多1000个。

\SACTitle{概要}
\nameref{sss:globalstack}、\nameref{sss:sumstack}、\nameref{sss:changestack}、\nameref{sss:incrementstack}、\nameref{sss:deletestack}




