\SACCMD{globalstack}
\label{sss:globalstack}

\SACTitle{概要}
设置全局迭加属性

\SACTitle{语法}
\begin{SACSTX}
    G!LOBAL!S!TACK! [W!EIGHT! v] [DI!STANCE! v] [DE!LAY! v [S!ECONDS!|P!OINTS!]]
        [I!NCREMENT! v [S!ECONDS!|P!OINTS!] [N!ORMAL!|R!EVERSED!]
\end{SACSTX}

\SACTitle{输入}
\begin{description}
\item [WEIGHT v] 全局权重因子,取值为0至1;
\item [DISTANCE v] 全局震中距,单位为km;
\item [DELAY v SECONDS|POINTS] 全局静时间延迟,单位为秒或数据点数;
\item [INCREMENT v SECONDS|POINTS] 全局静时间延迟的增量,单位为秒或数据点数;
\item [NORMAL|REVERSED] 正/负极性;
\end{description}

\SACTitle{说明}
该命令用于定义全局迭加属性,这些全局迭加属性用于迭加文件列表中的每个文件。可以使用
\nameref{sss:addstack}命令为某个文件单独设定迭加属性。

\SACTitle{相关命令}
\nameref{sss:addstack}


