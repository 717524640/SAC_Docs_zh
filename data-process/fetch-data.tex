\section{数据的获取}
目前,地震数据的获取主要是通过网络,当然也有直接用移动硬盘甚至U盘的。一般来说,
常见的地震数据格式是SEED格式。SAC格式适合用于地震数据处理,而SEED格式适合用于
数据的保存和交流。SEED格式本质上是一个压缩文件,单个SEED文件可以保存很长时间、
很多台站和很多分量的波形数据。可以用rdseed软件从SEED格式中解压得到SAC格式的
波形数据。当然也有一些数据中心提供了其它格式的数据,一般来说都会提供相应的工具
实现数据格式到SAC格式的转换。
