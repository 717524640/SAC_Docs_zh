\section{震相理论到时}
相关命令:\nameref{cmd:traveltime}

相关头段:tn(n=0-9)

traveltime命令,可以计算iasp91或者ak135模型下的震相理论走时,并将其保存到SAC头段
变量中。

\begin{SACCode}
SAC> dg sub teleseis nykl.z
SAC> traveltime model iasp91 picks 3 phase P S 
traveltime: depth: 0.000000 km
SAC> lh t3 kt3 t4 kt4
  
  FILE: /opt/sac/aux/datagen/teleseis/nykl.z - 1
   ------------------------------------------

         t3 = 4.430530e+02
        kt3 = P
         t4 = 7.999642e+02
        kt4 = S
\end{SACCode}

该命令会将震相P、S的理论到时依次写入头段变量t3、t4中,并写入相应的震相标识信息。

需要注意的是,该命令的正确运行要求内存中的所有波形数据的头段区中,必须包含地震位置
、台站位置、发震时刻信息。
