\section{仪器响应}
相关命令:\nameref{cmd:transfer}

地震仪器观测到的地面运动记录可以表示为

\[  u(t) = s(t) * g(t) * i(t) \] 
其中$s(t)$代表震源项, $g(t)$代表路径效应,$i(t)$代表仪器响应,星号代表卷积。

四个量中,$u(t)$是仪器记录的结果,已知;$i(t)$在仪器设计的时候会给出各种参数,已知;
$s(t)$和$g(t)$是未知的,也是地震学研究的两个主要内容:震源和结构。
因而理解仪器响应$i(t)$的基本原理并准确地去除仪器响应是研究的关键一步。

地震仪器记录的地面运动物理量有很多,常见的有速度和加速度,不太常见的还有位移、应变、旋转。
这些物理量在被地震仪接收到之后,首先要将其转换为电信号,然后对电信号振幅进行放大以及滤波,
再将连续时间序列离散化,最终以我们常见的波形的形式表现出来,所以我们最初见到的波形
表示的是电信号的强弱,单位是counts。

数据分析经常会遇到这样的情况,需要将电信号转换成真正的地面运动物理量,此时需要去除
仪器响应;有时不同数据使用了不同类型的仪器,为了使得数据之间可以相互比较,则需要
从原始数据中去除仪器响应,并卷积上同一类型仪器的仪器响应。

具体的细节可以transfer命令的解释以及如下几篇博文:
\begin{enumerate}
\item \url{http://seisman.info/instrumental-response-in-seismology.html}
\item \url{http://seisman.info/physical-details-of-instrumental-response.html}
\item \url{http://seisman.info/simple-analysis-of-resp.html}
\item \url{http://seisman.info/simple-analysis-of-sac-pz.html}
\item \url{http://seisman.info/difference-between-resp-and-pz-while-deconvolution.html}
\item \url{http://seisman.info/deep-analysis-of-response.html}
\end{enumerate}
