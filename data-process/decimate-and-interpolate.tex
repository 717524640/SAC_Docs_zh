\section{数据重采样}
相关命令:\nameref{cmd:decimate}、\nameref{cmd:interpolate}

不同仪器的采样周期可能不同,在数据处理之前一般需要将所有的数据重采样到相同
的采样周期;又或者数据的采样周期过小,导致数据量过大,此时也需要对数据进行
重采样以减少数据量。

SAC中有两个命令可以对数据进行重采样:一个是用于减采样的decimate,另一个是用于
插值的interpolate。

\begin{SACCode}
SAC> fg seis
SAC> lh delta npts
  
  FILE: SEISMOGR - 1
 --------------

     delta = 1.000000e-02
      npts = 1000
SAC> decimate 5; decimate 2     // 减采样10倍
SAC> lh delta npts
  
  FILE: SEISMOGR - 1
 --------------

     delta = 9.999999e-02       // delta=0.001,忽略此处的数值误差
      npts = 100
SAC> interpolate delta 0.015    // 重新插值到delta=0.015
SAC> lh delta npts
  
  FILE: SEISMOGR - 1
 --------------

     delta = 1.500000e-02
      npts = 666
\end{SACCode}

需要注意的是,interpolate会对数据进行插值,可能会引起插值误差;decimate对数据
进行减采样,为了避免混叠而加入了FIR滤波器。
