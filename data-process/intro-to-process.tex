\section{数据处理流程简介}

地震数据处理的基本步骤分为如下几类:
\begin{description}
\item[组织] SAC数据的准备;
\item[预处理] SAC数据的预处理;
\item[交互] 数据筛选和震相拾取等;
\item[处理] 计算某个属性/估计某个参数/转换数据;
\item[显示] 绘图,展示处理结果;
\end{description}
其中每一类又可以分为更小的步骤。

本章接下来的内容将介绍如何利用SAC完成其中的大部分数据处理步骤,
其中部分内容将在接下来的几章中进一步展开。
