\section*{\centering 写在第三版前的一些废话}

\begin{shadequote*}
\Large\emph{
工欲善其事,必先利其器。
}
\par\hfill\emph{\normalsize---《论语 $\cdot$ 卫灵公》}
\end{shadequote*}

2010年10月,大三,开始接触并学习SAC;2011年的暑假
开始着手SAC文档的翻译工作;2012年01月,文档的v1.0版本发布;2013年03月
,文档的v2.0版发布。3年多的时间过去了,文档更新到了v3.0版。

v2.0大体上算是官方英文文档的译本,整体结构上完全遵循了官方文档的风格。
整个文档的条理不够清晰,教程部分稍显单薄,命令部分也不够完善。

2013年11月,George Helffrich著的
``\emph{The Seismic Analysic Code : A Primer and User's Guide}''一书出版了。
该书基于MacSAC,与本文档所关注的SAC有一些区别,但是精髓部分是一致的。
v3.0版借鉴了该书的整体结构和部分内容,重新设计了文档结构并重写了教程的大
部分内容,希望能够有一个结构更清晰、内容更丰富的版本。

整个文档分为教程部分和命令部分。教程部分又分为如下几章:
\begin{description}
\item[SAC简介] 简单介绍SAC软件的相关信息;
\item[SAC基础] 继续阅读所需的基础知识;
\item[SAC文件格式] 详细介绍SAC文件格式;
\item[SAC数据处理] 介绍如何利用SAC命令进行地震数据处理和分析;
\item[SAC图像] 介绍如何控制SAC绘制的图像的细节;
\item[SAC编程] 介绍如何用SAC进行数据批处理;
\item[SAC与脚本] 如何在脚本语言(Bash和Perl)中调用SAC;
\item[SAC函数库] 在自己的C或Fortran程序中调用SAC提供的子函数;
\item[SAC I/O] 独立实现SAC I/O子函数;
\item[SAC相关工具] 与SAC有关的一些工具;
\end{description}

对本文档的内容有疑问,或发现任何错误、笔误,欢迎邮件联系我或者在本文档的
GitHub项目主页上提交Issue,也欢迎感兴趣的读者fork该项目,共同完善文档。

此文档仅供个人学习使用,希望不涉及版权问题。

\begin{flushleft}
个人博客: \url{http://seisman.info}                         \\
项目主页: \url{https://github.com/seisman/SAC_Docs_zh}      \\
联系方式: \url{seisman.info@gmail.com}                      \\
文档发布及更新: \url{http://seisman.info/sac-manual.html}   \\
捐赠页面:\url{http://seisman.info/donations.html}          \\

\end{flushleft}

\begin{flushright}
作者~:~SeisMan \\
2014年04月14日
\end{flushright}
