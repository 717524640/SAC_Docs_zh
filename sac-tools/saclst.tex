\section{saclst}
\label{sec:saclst}

saclst是很常用的一个SAC工具,用于列出头段变量的值,其语法很简单:
\begin{lstlisting}[style=Shell]
$ saclst header_lists f file_lists
\end{lstlisting}
其中~\lstinline{header_lists}~为要查看的头段变量名列表;\lstinline{f}~为关键字,表明接下来的所有参数
都是SAC文件;\lstinline{file_lists}~为SAC文件列表。需要注意的是,头段变量名是不区分大
小写的,除了头段变量F以外。大写的F被当作头段变量名,小写的f被作为关键字。
\footnote{这是设计不合理的地方。}

查看单个文件的单个头段:
\begin{lstlisting}[style=Shell]
$ saclst npts f seis.SAC
seis.SAC            1000
\end{lstlisting}

查看多个文件的多个头段:
\begin{lstlisting}[style=Shell]
$ saclst stla stlo evla evlo gcar f N.*.U
N.AAKH.U      36.3726      137.92      -5.514     151.161     43.4752
N.ABNH.U      34.6326     137.231      -5.514     151.161     42.0392
N.AC2H.U      35.4786     137.735      -5.514     151.161     42.6857
N.AGMH.U       35.787     137.717      -5.514     151.161     42.9798
N.AGWH.U      43.0842      140.82      -5.514     151.161     49.2714
N.AHIH.U      38.2799     139.549      -5.514     151.161     44.8874
\end{lstlisting}

在Bash脚本中将头段变量的值赋值给变量:
\begin{lstlisting}[style=Bash]
#!/bin/bash
stla=`saclst stla f seis | awk '{print $2}'`
stlo=`saclst stlo f seis | awk '{print $2}'`
echo $stla $stlo
\end{lstlisting}

在Perl脚本中将头段变量的值赋值给变量:
\begin{lstlisting}[style=Perl]
#!/usr/bin/env perl 
use strict;
use warnings;

my ($fname, $stla, $stlo) = split /\s+/, `saclst stla stlo f seis`;
print "$stla $stlo \n";
\end{lstlisting}
