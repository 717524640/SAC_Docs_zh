\section{SAC的读和写}
\label{sec:read-and-write}
SAC的读命令是~\nameref{cmd:read}(简写为r),写命令为~\nameref{cmd:write}(简写为w)。
读和写是紧密联系的,所以把这两者放在一起讲。

注意:本节的所有示例都运行在同一个SAC会话中。

要演示如何读SAC文件,首先得有一些SAC数据才行,利用上一节的datagen生成一些数据。
\begin{SACCode}
$ ls                // 空文件夹
$ sac               // 启动一个SAC会话
 SEISMIC ANALYSIS CODE [11/11/2013 (Version 101.6a)]
 Copyright 1995 Regents of the University of California

SAC> dg sub local cdv.n cdv.e cdv.z     // 生成三个SAC数据
SAC> w cdv.n cdv.e cdv.z                // 将SAC数据写入磁盘
SAC> ls                                 // 在SAC中也可以使用一些常见的系统命令
cdv.e  cdv.n  cdv.z
\end{SACCode}

有了数据之后,就可以练习如何去读了,在读数据之前,先说一说通配符的概念。

SAC中,在指定文件名的时候,可以使用绝对路径,也可以使用相对路径。可以使用
其全名,也可以使用通配符。SAC的通配符与Unix定义的通配符一致,只包含如下三种:
\begin{enumerate}
\item ``\lstinline{*}'' 匹配任意长度的字符串(包括零长度);
\item ``\lstinline{?}'' 匹配任意单个非空字符;
\item ``\lstinline{[]}'' 匹配列表中的任意单一字符;
    \begin{itemize}
        \item ``\lstinline{[ABC]}'' 匹配单个字符A或B或C
        \item ``\lstinline{[A,B,C]}'' 匹配单个字符A或B或C
        \item ``\lstinline{[0-9]}'' 匹配任意一位数字
        \item ``\lstinline{[a-g]}'' 匹配从a到g范围内的任意单个字符
    \end{itemize}
\end{enumerate}

下面的例子展示了如何读取SAC文件:
\begin{SACCode}
SAC> r cdv.n cdv.e cdv.z    // 读入三个文件,分别指定其文件名
SAC> r cdv.?                // 问号可以匹配单个字符。
./cdv.e ...cdv.n ...cdv.z   // 注意!这里文件读入的顺序与上一个命令不同
SAC> r cdv.[nez]            // 还可以这样读
./cdv.e ...cdv.n ...cdv.z
SAC> r *                    // 也可以这样读
./cdv.e ...cdv.n ...cdv.z
\end{SACCode}

需要注意的是,SAC在每次执行读取命令时,都会读入新的波形数据,并删除内存中原有的
波形数据,所以经过上面四次read之后,内存中依然只有三个波形。

当然,read也有选项,使得读取时将波形追加到内存中的波形数据集之后,而不替换
内存中的原有波形:
\begin{SACCode}
SAC> r ./cdv.n              // one file       0 -> 1
SAC> r more ./cdv.e         // one MORE file  1 -> 2
SAC> r more ./cdv.z         // one MORE file  2 -> 3
\end{SACCode}

将数据读入到内存中之后,对内存中的数据做一些处理,然后就需要将内存中的数据
写回到磁盘中:
\begin{SACCode}
SAC> w test.n test.e test.z         // 分别写入到三个新文件中
SAC> w over                         // 覆盖磁盘原文件
SAC> w append .new                  // 在原文件名的基础上加上后缀".new"
cdv.e.new cdv.n.new cdv.z.new
SAC> ls
cdv.e  cdv.e.new  cdv.n  cdv.n.new  cdv.z  cdv.z.new  tesn.n  test.e  test.z
\end{SACCode}
