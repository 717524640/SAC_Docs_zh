\section{样本数据}

想要学习SAC,手头必须有SAC格式的数据,SAC提供了两个命令可以用于生成SAC格式数据,分别是
funcgen和datagen。

\subsection{funcgen}
\nameref{cmd:funcgen}(简写为fg)表示``function generator'',即该命令可以生成一些特定的函数,
比如脉冲、阶跃、正弦等等,还可以生成一个地震波形样本。
\begin{SACCode}
SAC> fg impulse         // 生成脉冲函数
\end{SACCode}
上面的命令生成了一个脉冲函数并存储在SAC的内存中,可以用命令\nameref{cmd:plot}
(简写为p)在图形界面上查看这个函数的样子:
\begin{SACCode}
SAC> p
\end{SACCode}
在学习SAC的过程中,\nameref{cmd:funcgen}可以生成地震波形样本:
\begin{SACCode}
SAC> fg seismogram      // 生成地震波形样本,简写为fg seis
\end{SACCode}
这个命令在SAC内存中产生了一个地震波形样本\footnote{我不会告诉你,这个命令的本质就是
读取\lstinline{$SACAUX}目录中的\lstinline{seismogram}文件到内存。},
同时删除了内存中刚才生成的脉冲信号,可以使用\nameref{cmd:plot}命令查看地震波形。
这个地震波形样本在以后的教程中经常用到。

\subsection{datagen}
\nameref{cmd:datagen}(简写为dg)表示``data generator''。顾名思义,就是用来生成
数据的。

下面的例子在内存中生成了CDV台站记录到的一个近震的三分量波形数据
\footnote{我也不会告诉你,这个命令本质上就是从\lstinline{$SACAUX/datagen}目录中
读取SAC文件到内存。}
,并用\nameref{cmd:plot1}(简写p1)将三个波形画在一张图上:
\begin{SACCode}
SAC> dg sub local cdv.n cdv.e cdv.z
SAC> p1 
\end{SACCode}
简单解释一下,datagen命令中,sub为选项,其参数可以取\lstinline{local}、
\lstinline{regional}、\lstinline{teleseism},对应近震、区域震和远震,分别又可以
使用不同的参数以读取不同台站的SAC数据。具体参考\nameref{cmd:datagen}。

p1会将内存中的所有文件(本例中是3个)同时绘制在一张图上,当内存中的文件数目很多
时,需要修改p1的其它参数,以控制每张图上显示的波形数目。
