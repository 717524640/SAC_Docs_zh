\section{如何阅读本文档?}

本文档的结构大体分为两个部分:教程部分和命令部分。命令部分详细的列出了SAC中的
每一个命令的语法、参数以及一些技术细节,适合作为参考,在需要的时候查阅。

教程部分给出了很多日常数据处理的例子,初学者应该坐在计算机前,打开终端,
键入\footnote{严禁复制!不许偷懒!}书中的例子,试着理解每一个步骤的原理以及结果。

在阅读教程的同时,应随时翻看相应命令的说明,在实践的过程中掌握基础命令的语法和用法。
这样基本就完成了SAC初阶的要求。

在读完教程之后,应浏览SAC的几乎所有命令,并挑选其中感兴趣的一些进行尝试。此后,
在平常的科研工作中经常使用SAC,有了实践经验和对SAC的进一步认识之后,可以阅读文档
中的进阶内容,达到SAC进阶的要求。

最后,如果对SAC的内部机理感兴趣,可以阅读SAC的源码,重新实现一些SAC底层的功能。
