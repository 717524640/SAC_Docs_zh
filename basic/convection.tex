\section{文档约定}
约定这个事情,说起来容易做起来难,遇到不符合约定的地方只能靠读者自己领悟了。

\subsection*{语法约定}
\begin{enumerate}
\item 命令和选项使用大写字母,参数使用小写字母;
\item 命令和选项均使用全称,其简写形式用彩色突出显示;
\item ``\lstinline{[ ]}''表示该项为可选项;
\item ``\lstinline{A|B|C}''表示A、B、C中任选一项;
\end{enumerate}

示例如下:
\begin{SACSTX}
!B!AND!P!ASS [!BU!TTER|!BE!SSEL|C1|C2] [!C!ORNERS v1 v2] [!N!POLES n] 
    [!P!ASSES n] [!T!RANBW v] [!A!TTEN v]
\end{SACSTX}

\subsection*{示例约定}
\begin{enumerate}
\item 命令、选项、参数均使用小写字母;
\item 常见的命令和选项均使用简写表示;
\item 含有提示符``\lstinline{SAC>}''的行是用户键入的命令,无提示符的行是SAC输出行;
\item 示例中加入注释以帮助用户理解,注释使用了C语言的行注释符号``\lstinline{//}'';
\item 除非上下文说明,否则每个例子都运行在单独的SAC会话中,即每个命令都
    省略了启动sac和退出sac的命令。
\item 除特别情况外,均省略~\nameref{cmd:plot}~命令,用户应该学会随时~\nameref{cmd:plot}~
    以查看当前内存中的波形结果。
\end{enumerate}

示例如下:
\begin{SACCode}
$ sac                           // 该行省略
SAC> fg seis                    // 这是注释
SAC> p                          // 该行省略
SAC> lh o
  
  FILE: SEISMOGR - 1
 --------------

     o = -4.143000e+01
SAC> q                          // 该行省略
\end{SACCode}
