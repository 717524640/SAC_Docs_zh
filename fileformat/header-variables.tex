\section{熟悉头段变量}

\subsection{基本变量}
\begin{table}[H]
\centering
\caption{}
\label{}
\begin{tabular}{ccl}
    \toprule
	变量名	&	类型	&	描述\\
	\midrule
    nvhdr*\footnote{星号表示SAC文件中必须定义该变量}	&	N		&	SAC头段版本号,目前值为6,旧版本的SAC文件($nvhdr<6$)在读入时会自动更新。\\
	nzyear	&	N 		&	GMT年,文件参考时间	\\
    nzjday  &   N       &   GMT儒略日,一年中的第几天\footnote{比``月-日''少用一个变量。}    \\
    nzhour  &   N       &   GMT时   \\
    nzmin   &   N       &   GMT分   \\
    nzsec   &   N       &   GMT秒   \\
    nzmsec  &   N       &  GMT毫秒  \\
	nzdttm	&	N 		&	GMT日期-时间数组,不在头段中,用于子程序读取SAC文件。该数组由上面6个变量构成    \\
	kzdate	&	A		&	不在头段中,字母数字格式的GMT参考日期,由NZYEAR和NZJDAY导出\\
	kztime	&	A		&	不在头段中,字母数字格式的GMT参考时间,由NZHOUR, NZMIN, NZSEC和NZMSEC导出\\
	iftype*	&	I		&	文件类型:\\
						&& \quad- ITIME 时间序列文件	\\
						&& \quad- IRLIM 频谱文件实部-虚部格式 \\
 						&& \quad- IAMPH 频谱文件振幅-相位格式	\\
						&& \quad- IXY 一般的x-y数据 	\\
						&& \quad- IXYZ 一般的XYZ(3-D)文件\\	
	idep	&	I		&	因变量类型:\\
						&& \quad- IUNKN (未知)	\\
						&& \quad- IDISP (位移:nm)	\\
						&& \quad- IVEL (速度:nm/sec)	\\
						&& \quad- IVOLTS (速度:volts)	\\
						&& \quad- IACC (加速度:nm/sec/sec)	\\
	iztype	&	I 		&	等效参考时间	\\
						&& \quad- IUNKN (未知)	\\
						&& \quad- IB (文件开始时间)	\\
						&& \quad- IDAY (基准GMT的午夜)	\\
						&& \quad- IO (事件发生时间)	\\
						&& \quad- IA (初动到时)	\\
						&& \quad- ITn (用户自定义的读取时间Tn, n=0,9)	\\
    \bottomrule
\end{tabular}
\end{table}

\subsection{数据相关变量}
\begin{table}[H]
\centering
\caption{}
\label{}
\begin{tabular}{ccc}
    \toprule
	变量名	&	类型	&	描述\\
	\midrule
    npts*	&   N		&	数据点数。\\
	delta*	&	F		&	等间隔数据的采样间隔(标称值)。\\
	DEPMIN	&	F 		&	因变量最小值。	\\
	DEPMAX	&	F		&	因变量最大值	\\
	DEPMEN	&	F		&	因变量平均值	\\
	SCALE	&	F		&	因变量比例因子\footnote{真实物理场被乘以比例因子得到现有数据}	\\
	ODELTA*	&	F		&	采样间隔的观测值,若观测值与标称值不同则有值\\
	B*		&	F		&	自变量起始值(相对参考时间的秒数)\\
	E*		&	F		&	自变量结束值(相对参考时间的秒数)\\
    XMINIMUM&   F       &   X的最小值(限于谱文件)   \\                                                                                                    
    XMAXIMUM&   F       &   X的最大值(限于谱文件)   \\                           
    YMINIMUM&   F       &   Y的最小值(限于谱文件)   \\                           
    YMAXIMUM&   F       &   Y的最大值(限于谱文件)   \\  
    NXSIZE  &   N       &   频谱长度(限于谱文件)    \\                           
    NYSIZE  &   N       &   频谱宽度(限于谱文件)    \\
    IQUAL   &   I       &   数据质量[未使用]:   \\                                                                                                        
                        && - IGOOD (Good data)  \\                               
                        && - IGLCH (Glitches)   \\                               
                        && - IDROP (Dropouts)   \\                               
                        && - ILOWSN (Low signal to noise ratio) \\               
                        && - IOTHER (Other)\\                                    
    ISYNTH  &   I       &   合成数据标识[未使用]:\\                              
                        && - IRLDTA (Real data) \\                               
                        && - ????? (Flags for various synthetic seismogram codes) \\
     LEVEN*  &   L       &   若数据为等间隔则为TRUE。\\
    \bottomrule
\end{tabular}
\end{table}

\subsection{事件相关变量}
\begin{table}[H]
\centering
\caption{}
\label{}
\begin{tabular}{ccc}
    \toprule
	变量名	&	类型	&	描述\\
	\midrule
    KEVNM   &   K       &   事件名  \\
    IEVREG  &   I       &   事件地理区域[未使用]    \\
    IEVTYP  &   I       &   事件类型    \\
    EVLA    &   F       &   事件纬度(度,北为正) \\                                                                                                        
    EVLO    &   F       &   事件经度(度,东为正) \\                                  
    EVEL    &   F       &   事件高程(m). [未使用]   \\                              
    EVDP    &   F       &   事件相对地表深度(m). [未使用]   \\                   
    MAG     &   F       &   事件震级    \\                                       
    IMAGTYP &   I       &   震级类型:\\                       
    IMAGSRC &   I       &   震级来源信息:   \\
    NEVID   &   N       &   事件ID (CSS 3.0)    \\
    NORID   &   N       &   起始时间ID (CSS 3.0)    \\ 
    NWFID   &   N       &   波形ID (CSS 3.0)    \\ 
    KHOLE   &   K       &   核爆事件:孔眼标识;其他:位置标识  \\
    GCARC   &   F       &   台站到事件的大园弧长,即另一种震中距(度).\\ 
    DIST    &   F       &   事件台站距离,即震中距(km)\\                         
    AZ      &   F       &   事件到台站的方位角(度).\\                                                                                                     
    BAZ     &   F       &   台站到事件的方位角(度). \\                           
    O       &   F       &   事件发生时间(相对参考时间的秒数)    \\                                                                                        
    KO      &   A       &   事件发生时间标志    \\  
    \bottomrule
\end{tabular}
\end{table}

\subsection{台站相关变量}
\begin{table}[H]
\centering
\caption{}
\label{}
\begin{tabular}{ccc}
    \toprule
	变量名	&	类型	&	描述\\
    \midrule
    KNETWK  &   K       &   地震台网名  \\
    KSTNM   &   K       &   台站名      \\      
    ISTREG  &   I       &   台站地理区域[未使用]    \\  
    STLA    &   F       &   台站纬度(度,北为正)    \\                           
    STLO    &   F       &   台站经度(度,东为正).   \\                           
    STEL    &   F       &   台站高程(m). [未使用]\\                              
    STDP    &   F       &   台站相对地表深度(m). [未使用]\\                      
    CMPAZ   &   F       &   分量方位角(从北开始顺时针度数).\\                    
    CMPINC  &   F       &   分量倾角(从垂直开始的度数)  \\    
    KCMPNM  &   K       &   分量名称,SEED格式使用那个三字符名称,第三个代表分量的方位(如BHE),
                            对于水平分量,目前的趋势是使用1和2代替N和E  \\ 
    KSTCMP  &   A       &   台站分量,由KSTNM, CMPAZ和CMPINC导出\\
    LPSPOL  &   L       &   如果台站分量为正极性则为真(左手规则)  \\
    \bottomrule
\end{tabular}
\end{table}

\subsection{震相相关变量}
\begin{table}[H]
\centering
\caption{}
\label{}
\begin{tabular}{ccc}
    \toprule
	变量名	&	类型	&	描述\\
    \midrule
    A       &   F       &   初动到时(相对参考时间的秒数)\\                       
    KA      &   K       &   初动到时标志    \\                                                                                                            
    F       &   F       &   事件结束时间(相对参考时间的描述,注意与文件结束时间的区别)\\
    KF      &   A       &   事件结束标志    \\                                   
    Tn      &   F       &   用户定义的时间,在拾取震相时使用,n = 0-9(相对参考时间的秒数)\\
    KTn     &   K       &   用户定义的时间标志, n = 0-9.    \\ 
    \bottomrule
\end{tabular}
\end{table}

\subsection{仪器相关变量}
\begin{table}[H]
\centering
\caption{}
\label{}
\begin{tabular}{ccc}
    \toprule
	变量名	&	类型	&	描述\\
    \midrule
    KINST   &   K       &   记录仪器通用名称    \\                               
    IINST   &   I       &   记录仪器类型[未使用]\\                               
    RESPn   &   F       &   仪器响应参数,n=0,9. [未使用]\\
    \bottomrule
\end{tabular}
\end{table}

\subsection{其它变量}
\begin{table}[H]
\centering
\caption{}
\label{}
\begin{tabular}{ccc}
    \toprule
	变量名	&	类型	&	描述\\
    \midrule
    USERn   &   F       &   用户定义变量存储区, n = 0,9.\\
    KUSERn  &   K       &   用户定义变量存储区,  n = 0,9.\\     
    LOVROK  &   L       &   如果文件可覆盖则为真,类似于写权限  \\  
    LCALDA  &   L       &   如果DIST, AZ, BAZ 和 GCARC可以由台站和事件的坐标计算出来则为真\\
    KDATRD  &   K       &   数据被读入计算机的日期\\ 
    \bottomrule
\end{tabular}
\end{table}
