\section{SAC格式简介}
一个地震波形数据包含了时间上连续的一系列数据点,数据点可以是等间隔或不等间隔采样。
SAC的数据格式要求一个文件中只包含一个地震波形数据,这样的定义更适合单个地震波形的处理。

每个SAC文件包含两个部分:一个头段区和一个/多个数据区。

头段区位于每个文件的起始处,其大小是固定的,用于描述数据的相关信息,比如数据点数、
采样周期等等。

数据区紧跟在头段区之后,数据区的个数的规则大致如下:
\begin{itemize}
\item 如果数据是时间序列,且是等间隔采样的,则只有一个数据区,包含因变量(Y,也就是数据)的值,
    因变量(X)的信息可以直接从头段区中获得;
\item 如果数据是时间序列,但是不等间隔采样的,则有两个数据区,分别包含自变量(X)和因变量(Y)的值;
\item 如果数据是谱数据而非时间序列,则有两个数据区,分别包含振幅和相位或者实部和虚部。
\end{itemize}
