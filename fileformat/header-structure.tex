\section{SAC头段结构}
SAC头段区长度为158个字(每个字占四个字节),由一系列头段变量构成。

表\ref{table:header-variables}给出了SAC头段区的全部头段变量。

\begin{table}[H]
\centering
\caption{SAC头段变量列表}
\label{table:header-variables}
\begin{tabular}{c|c|lllll}
	\toprule
	Word	&	Type	&	\multicolumn{5}{c}{Names}\\
	\midrule
	0		&	F	&	DELTA	&	DEPMIN	&	DEPMAX	&	SCALE	&	ODELTA	\\
	5		&	F	&	B		&	E		&	O		&	A		&	INTERNAL\\
	10		&	F	&	T0		&	T1		&	T2		&	T3		&	T4		\\
	15		&	F	&	T5		&	T6		&	T7		&	T8		&	T9		\\
	20		&	F	&	F		&	RESP0	&	RESP1	&	RESP2	&	RESP3	\\
	25		&	F	&	RESP4	&	RESP5	&	RESP6	&	RESP7	&	RESP8	\\
	30		&	F	&	RESP9	&	STLA	&	STLO	&	STEL	&	STDP	\\
	35		&	F	&	EVLA	&	EVLO	&	EVEL	&	EVDP	&	MAG		\\
	40		&	F	&	USER0	&	USER1	&	USER2	&	USER3	&	USER4	\\
	45		&	F	&	USER5	&	USER6	&	USER7	&	USER8	&	USER9	\\
	50		&	F	&	DIST	&	AZ		&	BAZ		&	GCARC	&	INTERNAL\\
	55		&	F	&	INTERNAL&	DEPMEN	&	CMPAZ	&	CMPINC	&	XMINIMUN\\
	60		&	F	&	XMAXIMUM&	YMINIMUM&	YMAXIMUN&	UNUSED	&	UNUSED	\\
	65		&	F	&	UNUSED	&	UNUSED	&	UNUSED	&	UNUSED	&	UNUSED	\\
	70		&	N	&	NZYEAR	&	NZJDAY	&	NZHOUR	&	NZMIN	&	NZSEC	\\
	75		&	N	&	NZMSEC	&	NVHDR	&	NORID	&	NEVID	&	NPTS	\\
	80		&	N	&	INTERNAL&	NWFID	&	NXSIZE	&	NYSIZE	&	UNUSED	\\
	85		&	I	&	IFTYPE	&	IDEP	&	IZTYPE	&	UNUSED	&	IINST	\\
	90		&	I	&	ISTREG	&	IEVREG	&	IEVTYP	&	IQUAL	&	ISYNTH	\\
	95		&	I	&	IMAGTYP	&	IMAGSRC	&	UNUSED	&	UNUSED	&	UNUSED	\\
	100		&	I	&	UNUSED	&	UNUSED	&	UNUSED	&	UNUSED	&	UNUSED	\\
	105		&	L	&	LEVEN	&	LPSPOL	&	LOVROK	&	LCALDA	&	UNUSED	\\
	110		&	K	&	KSTNM	&	KEVNM*	&			&			&			\\
	116		&	K	&	KHOLE	&	KO		&	KA		&			&			\\
	122		&	K	&	KT0		&	KT1		&	KT2		&			&			\\
	128		&	K	&	KT3		&	KT4		&	KT5		&			&			\\
	134		&	K	&	KT6		&	KT7		&	KT8		&			&			\\
	140		&	K	&	KT9		&	KF		&	KUSER0	&			&			\\
	146		&	K	&	KUSER1	&	KUSER2	&	KCMPNM	&			&			\\
	152		&	K	&	KNETWK	&	KDATRD	&	KINST	&			&			\\
    \bottomrule
\end{tabular}
\end{table}

表\ref{table:header-variables}中,第一列为起始字节数,比如\lstinline{delta}开始于
头段区的第0个字节,而\lstinline{evla}则开始于头段区的第35个字节。第二列给出了
头段变量的类型码,每个头段变量类型码对应的数据类型见表\ref{sec}。

\begin{table}[H]
\caption{变量类型说明}
\label{table:header-variables-type}
\centering
\begin{tabular}{clll}
	\toprule
    类型代码&	数据类型    &   C变量类型           	&	未定义值        \\
	\midrule
    F		&	浮点型		&   \lstinline{float}       &	\lstinline{-12345.0}        \\
    N		&	整型		&   \lstinline{int}         &	\lstinline{-12345}        \\
    I		&	枚举型		&   \lstinline{int}         &	\lstinline{-12345}	        \\
    L		&	逻辑型		&   \lstinline{int}         &	\lstinline{FALSE}	        \\
    K		&	字符型		&   \lstinline{char*}       &	\lstinline[showspaces=true]{"-12345  "}     \\
    A		&	辅助型		&                           &				    \\
	\bottomrule
\end{tabular}
\end{table}

关于变量类型的一些解释:
\begin{description}
    \item [浮点型] 在C语言中变量类型为\lstinline{float},占四个字节;(G15.7)
    \item [整型]  变量名以N开头;(I10)
    \item [枚举型] 变量名以I开头,本质上是\lstinline{int}型,其只能在固定的几个值中取值;(I10)
    \item [逻辑型] 变量名以L开头,取值为true或者false;(I10)
    \item [字符型] 变量名以K开头,其长度为8或16;(A8,A16)
    \item [辅助型] 不存在于SAC文件中,而是由其它头段变量导出;
\end{description}

对于F、N和I型变量,每个变量占据4个字节,对于K型变量,每个变量占8个字节,变量
\lstinline{kevnm}为16个字节,总计632字节。

变量名为\lstinline{INTERNAL},表示该变量为SAC内部所需要的变量,用户不可对其进行操作。

变量名为\lstinline{UNUSED}表示该变量暂时尚未使用。

对于任意文件,并非所有的头段变量都含有有意义的值,这些无定义的值称为未定义变量。
对于任一类型的变量,用上表中类似于-12345之类的数据表明该变量未定义。
