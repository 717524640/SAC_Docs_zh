\section{图像设备与图像保存}
\label{sec:save-image}

\subsection{xwindows}
xwindows是SAC中最常用也是默认的绘图设备。对于震相拾取等交互式操作更是比不可缺。
\begin{SACCode}
SAC> fg seis
SAC> bd x       // begindevice xwinows,可省略
SAC> p          // 绘图
SAC> ed x       // enddevice xwindows,可省略
SAC> q
\end{SACCode}

对于xwindows,最简单的保存图像的方式是截图,常用的工具包括gnome下的screenshot或者
ImageMagick的import命令。

\subsection{sgf}
SGF图像设备是SAC自己设计的与平台无关的图像格式,其包含了SAC图像的全部信息。其流程
为``启用sgf设备''$\rightarrow$``绘图到sgf''$\rightarrow$``关闭sgf设备,退出SAC''
$\rightarrow$``将sgf文件转换为其它格式''。
\begin{SACCode}
SAC> fg seis
SAC> bd sgf         // 启动sgf设备,不可省略
SAC> p 
SAC> ed sgf         // 关闭sgf设备,可省略
SAC> q
$ ls 
f001.sgf            // 生成sgf文件
\end{SACCode}

生成的sgf文件可以通过sgftops等命令转换为其它图像格式,在``\nameref{sec:sgftops}''中会介绍,
也可以使用sgftox直接将sgf文件显示在绘图窗口中。

\subsection{PS和PDF}
自101.5之后,SAC加入了saveimg命令,以获得更高质量的绘图效果。saveimg会将当前xwindow
窗口显示的图像以ps或pdf格式保存到图像文件中。

\begin{SACCode}
SAC> fg seis
SAC> p                      // 首先在xwindows上绘图
SAC> saveimg foo.ps         // 将xwindows上的图像保存到foo.ps中
save file foo.ps [PS]
SAC> q
\end{SACCode}

\subsection{pssac}
pssac是Prof. Lupei Zhu写的C程序。其利用GMT的PS绘图库,直接读取SAC文件并绘制到PS文件中,
得益于GMT的PS库的灵活性,
利用pssac可以绘制出超高质量的图像。关于pssac的用法,参见``\nameref{sec:pssac}''一节。

\subsection{图像保存小结}
\begin{itemize}
\item xwindows绘图简单省事,但质量较差;
\item sgf转换稍显麻烦,但适合在脚本中批量做图;
\item saveimg生成的图件质量相对较高,可以满足大多数需求;
\item pssac功能强大,在需要设计复杂图像时非常有用;
\end{itemize}
