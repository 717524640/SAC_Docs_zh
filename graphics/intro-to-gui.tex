\section{图像输出设备}
SAC支持三类图像输出设备:
\begin{description}
\item [X Window System] 也常称为X11或X,是一种以位图方式显示的软件窗口系统。
    图~\ref{fig:plot}~展示了SAC中的X窗口,这是SAC默认的绘图设备。
\item [SGF] 全称SAC Graphic File,即SAC图像文件。SGF文件中包含了绘制一个图像
    所包含的全部信息,因而可以通过转换程序将其转换到其它图像设备或图像格式。
    SAC提供了三个sgf转换工具:sgftops、sgftoeps和sgftox。
\item [PDF和PS] 从101.5开始,SAC加入了~\nameref{cmd:saveimg}~命令,可以将图像直接
    输出为PDF或PS格式
    \footnote{该命令也支持输出为png和xpm格式,但png和xpm为位图图像格式,精度不够,
    且依赖于其它函数库,因而不推荐使用。}。
\end{description}

SAC提供了~\nameref{cmd:setdevice}~命令设置默认的图像设备;~\nameref{cmd:begindevices}~
和~\nameref{cmd:enddevices}~用于启动/关闭指定的图像设备。

对于xwindows,SAC提供了10个可以使用的窗口,编号为1-10。~\nameref{cmd:window}~命令
可以设置每个绘图窗口的长宽比以及相对于屏幕的位置;~\nameref{cmd:beginwindow}~命令可以
启动某个编号的绘图窗口。

对于SGF,每个图像都将保存到sgf文件中,文件名格式为``fnnn.sgf'',其中``nnn''为图像
编号,起始编号为001。~\nameref{cmd:sgf}~命令可以控制SGF图像设备的一些选项,比如sgf
文件的前缀、起始编号、目录、文件尺寸等。

对于PDF和PS,由于saveimg命令是在101.5之后才加入的,所以使用上与xwindows和sgf有很多不同
之处。

这三者的区别会在``\nameref{sec:save-image}''中详细分析。
