\SACCMD{cor}
\label{spe:cor}

\SACTitle{概要}
计算相关函数

\SACTitle{语法}
\begin{SACSTX}
COR [N!UMBER! n|ON|OFF] [L!ENGTH! v] [P!REWHITEN! ON|OFF|n]
    [S!TOCASTIC!|TR!ANSIENT!]
    [T!YPE! HAM!MING!|HAN!NING!|C!OSINE!|R!ECTANGLE!|T!RIANGLE!]
\end{SACSTX}

\SACTitle{输入}
\begin{description}
\item [NUMBER n] 设定窗口数为n
\item [NUMBER ON] 设定窗口数为先前值
\item [NUMBER OFF] 根据数据长度和窗长计算窗口数。使用该选项时没有数据重叠
\item [LENGTH v] 设置窗长为v秒
\item [TYPE type] 设置窗类型
\item [PREWHITEN ON|OFF] 打开/关闭预白化选项
\item [PREWHITEN n] 打开预白化选项,并设置系数的个数为n
\item [STOCHASTIC] 设置相关定标,假定数据是随机的
\item [TRANSIENT] 设置相关定标,假定数据是瞬态信号
\end{description}

\SACTitle{缺省值}
\begin{SACDFT}
cor number off type hamming prewhiten off
\end{SACDFT}

\SACTitle{说明}
这个命令假定数据是稳态的。基于这一假定,数据被分段成许多窗口,并且对
每一个窗口计算一个相关函数。求这些函数的平均值即生成基于随机过程的相关
函数的更加稳定的估计。窗口数和窗口长度以及窗口类型都由用户控制。如果
窗口长度诚意窗口数目超过了数据总长度则窗口会发生重叠,重叠的部分不在
用户的控制之下。

很明显,对于一个固定的数据长度,在窗口数和窗口尺寸之间存在一个折衷。
这种折衷最终决定了使用相关函数做出的谱估计的混淆和方差之间的折衷。
一个频谱估计算法的频率域分辨率依赖于可以使用的相关的长度,因而也就间接
地依赖于数据窗口的大小。相关窗口越大,由频率平滑产生的频谱估计的混淆就
越小。然而随着数据窗口尺寸加大,在平均计算只能怪可以使用的窗口的数量会
减少,其结果是,相关函数估计的方差会增加,从而频谱估计的方差也会随之
增加。

适当地选择窗口类型可以用来协调混淆和方差之间的折衷。较平滑的窗口使窗口
边缘附近的数据平滑地过渡到零,从而实际上相当于减少了窗口的长度。于是窗口
可以更多的重叠,也就是说可以有更多的窗口,折衷选择是在以增加混淆的代价
减小方差。

在频谱的动态范围相当大的时候混淆还有另外一个来源,即窗口泄漏效应。在使用
PDS估计的时候这一点尤为明显。经由相关窗口的FFT的旁瓣功率给求出的频谱
带来了一个台阶。在典型的地震数据中,这种台阶是规则的,而且出现在高频段
上,在这个频段上的频谱通常较小。相关函数估计有一个可选的预白化功能,它
可以减缓旁瓣泄露问题。一个低阶的预测误差滤波器可以用来在计算相关函数之前
平滑数据的频谱,滤波器的效果将在频谱的计算过程中得到补偿。

数据进行预白化改变了原始信号,如果用户使用了预白化,退出子程序并且想要
在别的操作中再次使用原始的信号,则必须重新读入原始信号到SAC。

这种相关函数用于频谱的计算,\nameref{spe:cor}必须在执行\nameref{spe:pds}、
\nameref{spe:mlm}、\nameref{spe:mem}之前执行,用户可以执行\nameref{spe:plotcor}
命令绘制相关函数并且可以使用\nameref{spe:writecor}命令将其作为SAC文件进行
保存。

\SACTitle{头段变量改变}
depmin、depmax、depmin

\SACTitle{相关命令}
\nameref{spe:plotcor}、\nameref{spe:writecor}、\nameref{spe:whiten}、\nameref{spe:readcor}
