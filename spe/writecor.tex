\SACCMD{writecor}
\label{spe:writecor}

\SACTitle{概要}
写一个包含相关函数的SAC文件

\SACTitle{语法}
\begin{SACSTX}
W!RITE!COR [file]
\end{SACSTX}

\SACTitle{输入}
\begin{description}
\item [file] 要写入的SAC文件名
\end{description}

\SACTitle{缺省值}
\begin{SACDFT}
writecor cor
\end{SACDFT}

\SACTitle{说明}
由此命令写出的相关函数的结构取决于用来计算它的算法。由于数据被分配到几个
窗口中,并且样本相关函数是从每一个窗口算出的然后再去平均,则相关函数的
长度由数据窗口的尺寸决定,它实际上容纳的样本数则为数据窗内数据的二倍减
一。然而,由于计算样本相关函数时使用了FFT算法,文件中的样本数是2的乘方。
也就是说实际上是数据窗口尺寸的2倍。附加的样本都是零。相关函数也是在文件内
循环回转的。这是由于使用FFT算法计算相关函数的特殊性。这意味着零滞后样本是
文件在中的第一个样本,负滞后的样本紧随着正滞后的样本。

\SACTitle{错误消息}
\begin{itemize}
\item 5003:未计算相关函数
\end{itemize}
