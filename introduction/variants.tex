\section{SAC变体}

SAC的发展史还是很曲折的,这也导致SAC存在多个不同的变体。

\begin{description}
\item[Fortran SAC]  即SAC的Fortran语言实现。最后一个分发版本发布于2003年,版本号10.6f。
                    曾经以限制性的形式在IASPEI软件库中分发。
\item[SAC2000]      从Fortran源码转换为C源码,并以C源码为基础继续维护。该版本加入了数据库特性以及
                    一些新的命令。目前该版本已不再分发。
\item[SAC/IRIS]     由SAC2000衍生的版本,不包含数据库特性\footnote{目前的SAC/IRIS中还可以看到一些
                    与该特性相关的命令和选项,这属于历史遗留问题。},
                    也就是本文档所使用的版本。现在由IRIS下的SAC开发小组负责维护,并由IRIS分发。
\item[MacSAC]       也称为SAC/BRIS,仅可在Mac OS下使用。该变种由10.6d Fortran源码衍生而来(后期与10.6f集成),
                    其功能是SAC/IRIS功能的超集。相对于SAC/IRIS的最主要扩展在于宏语言功能的增强以及
                    处理台阵数据的能力。其作者为
                    George Helffrich\footnote{\url{http://www1.gly.bris.ac.uk/~george/gh.html}},
                    针对MacSAC写了一本教程
                    \footnote{G.R. Helffrich, J. Wookey \& I.D. Bastow, The Seismic Analysis Code
                    : A Primer and User's Guide. \textsl{Cambridge University Press}, 2013。
                    可以作为学习SAC/IRIS的辅助教程,但需要注意其中可能存在的一些微小差异。}。
\end{description}
