\section{SAC是什么?}

Seismic Analysis Code (SAC),是天然地震学领域使用最广泛的数据分析软件包之一。

SAC首先是一个软件,主要在命令行下工作,通过各种命令来处理时间序列数据
(尤其是地震波形数据),同时也提供了一个简单的图形界面,使得用户可以方便地查看
波形并拾取震相。

SAC同时还是一种数据格式,定义了以何种方式存储单个时间序列数据。
SAC格式已经成为了地震学的标准数据格式之一,有很多工具可以实现SAC格式
与其它地震数据格式间的相互转换。

SAC实现了地震数据处理过程中的常用操作,包括重采样、插值、自/互相关、震相拾取、
快速Fourier变换、谱估计、滤波、信号叠加等;同时为了满足数据批处理的需求,
SAC设计了一个基本的编程语言
\footnote{SAC设计的编程语言,称之为SAC宏,在``\nameref{chap:sac-programming}''
一章中会详细说明。},
包含了变量、参数、条件判断、循环控制等特性。
