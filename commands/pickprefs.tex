\SACCMD{pickprefs}
\label{cmd:pickprefs}

\SACTitle{概要}
这个命令用于控制SAC管理震相和/或从不同的输入数据格式(例如CSS,GSE,SUDS...)读入到时间标记变量T0到T9,到这个选项为OFF(缺省状态),则载入到时间标记的震相为SAC在输入文件中发现的第一个拾取,如果这个选项为ON,SAC将使用READCSS命令中描述的参考文件。

\SACTitle{语法}
\begin{SACSTX}
PICKPR!EFS! ON|OFF
\end{SACSTX}

\SACTitle{输入}
\begin{itemize}
\item ON : 指示SAC通过参考文件将到时信息由CSS缓冲区传送到SAC缓冲区。这对于需要特定到时位于特定头段变量的宏文件来说很有用
\item OFF : 指示SAC绕过参考文件,载入给定文件的前10个震相拾取。这个是默认值,它允许用户注意一些在PICKPREFS ON时无法注意的拾取。 
\end{itemize}

\SACTitle{缺省值}
\begin{SACDFT}
pickprefs off
\end{SACDFT}

\SACTitle{说明}
从版本0.58开始, sac2000就有两个不同的头段缓冲区:一个根据SAC文件格式构建,一个根据有关的CSS3.0文件格式。添加了CSS数据缓冲区使得读取如CSS、GSE、SUDS等这里相关格式变得更加容易。两个缓冲区同时也使得下面的几个命令得以使用:COMMIT、ROLLBACK、RECALLTRACE(参看具体文档)
有两个缓冲区的一个缺点是将到时从动态的CSS到时表转移到SAC格式中相对静态的Tn拾取变得更加复杂了,这个问题在0.58版本中通过设置了一个称为csspickprefs	的参考文件得以解决。这个文件在aux目录下,你可以覆盖它写入你想要的信息。更	多关于csspickprefs的信息,参见READCSS命令。关于如何覆盖默认参考文件,参看PICKAUTHOR或PICKPHASE

使用参考文件的缺点是它将只接收列在参考文件中或在命令PICKPHASE、PICKAUTHOR中输入的震相或作者列表。换句话说,如果一个CSS数据文件有一个pP的到时,不管其来自于平面文件还是Oracle数据库,而pP未在参考文件中指定,那么用户就绝不会知道pP在那里。pP震相将读入SAC中的CSS数据缓冲区,但是它不会转变到SAC数据缓冲区中,也不会参与任何SAC命令。它可以通过WRITECSS命令写出,或者可能通过COMMIT命令刷新然后全部丢失。

我们给出的解决办法是允许用户绕过参考文件。在0.59版本中,默认是从CSS缓冲区中直接读入前10个可用的拾取到SAC缓冲区中,通过这个新命令PICKREFS的使用,用户可以告诉SAC使用指定的参考文件。

\SACTitle{相关命令}
\nameref{cmd:pickauthor}、\nameref{cmd:pickphase}
