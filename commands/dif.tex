\section{dif}
\label{cmd:dif}

\SACTitle{概要}
对内存中的数据进行微分

\SACTitle{语法} 
DIF [TWo|THree|Five]

\SACTitle{输入}
\begin{itemize}
\item TWo : 两点差分操作
\item THree : 三点差分操作
\item FIVE : 五点差分操作
\end{itemize}

\SACTitle{缺省值}
DIF TWO

\SACTitle{说明}
两点差分算法:
\[ OUT(j) =\frac{DATA(j+1) -DATA(j)}{DELTA} \]
输出的最后一个点在这个算法中没有定义,它也不是一个中心算法。SAC通过将文件点数(NPTS)减去1并且将开始时间(B)增加采样间隔(DELTA)的一半来解决输出点数的问题。 

三点差分(中心两点)算法:
\[ OUT(j) = \frac{1}{2} \frac{DATA(j+1) -DATA(j-1)}{DELTA} \]
这个算法未定义输出点的第一个和最后一个。SAC将NPTS减去2,将B加上DELTA

五点差分算法(中心四点)算法:
\[ OUT(j) = \frac{2}{3} \frac{DATA(j+1) -DATA(j-1)}{DELTA} - \frac{1}{12} \frac{DATA(j+2) -DATA(j-2)}{DELTA} \]
算法未定义输出的前两个和后两个点。SAC对第二个和倒数第二个两个点运用三点操作,将NPTS减去2,将B增加DELTA

\SACTitle{错误消息}
\begin{itemize}
\item[-]1301: 未读入文件
\item[-]1306: 对不等间隔文件非法操作
\end{itemize}

\SACTitle{头段变量改变}
NPTS, B, E, DEPMIN, DEPMAX, DEPMEN

\SACTitle{最近修订}
January 15, 1985 (Version 9.10)
