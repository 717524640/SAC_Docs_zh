\SACCMD{listhdr}
\label{cmd:listhdr}

\SACTitle{概要}
列出被选择的头段变量的值

\SACTitle{语法}
\begin{SACSTX}
L!ist!H!dr! [D!efault!|P!icks!|SP!ecial!] [FILES ALL|NONE|list] [COLUMNS 1|2] 
    [INCLUSIVE ON|OFF] [hdrlist]
\end{SACSTX}

\SACTitle{输入}
\begin{itemize}
\item DEFAULT : 使用默认的头段列表,它列出了所有定义了的头段变量。
\item PICKS : 使用picks列表,包含了用于定义时间拾取的头段 
\item SPECIAL :  使用用户定义的特别列表 
\item FILES ALL : 列出内存中所有文件的头段 
\item FILES NONE : 不列出头段,为将来的命令设置默认值 
\item FILES list : 列出部分文件的头段,list是要处理的文件号 
\item COLUMNS 1 : 每行一列的输出格式  
\item COLUMNS 2 : 每行两列的输出格式  
\item INCLUSIVE :  ON表示包含没有定义的头段变量,OFF则不包含 
\item hdrlist : 要列出的头段列表  
\end{itemize}

\SACTitle{缺省值}
\begin{SACDFT}
LISTHDR DEFAULT FILES ALL COLUMNS 1 INCLUSIVE OFF
\end{SACDFT}

\SACTitle{说明}
用户可以自己定义要列出的项或使用标准列表中的任一个,DEFAULT列表包含全部头段变量。PICKS列表包含直接或间接用于定义时间拾取的头段变量,这个列表包含B, E,O, A, Tn, KZTIME, KZDATE。用户自己可以随时定义一个特殊列表而且可以通过SPECIAL选项再次使用。命令输出列表包含头段变量名,等于号以及当前值。有些头段可能未定义。SAC在这些未定义的头段中储存一个特别的标志以标记它们,对于整数和浮点数未定义值为-12345,字符串以及那些用于表示字符串含义的整数未定义值为``UNDEFINED''.

\SACTitle{错误消息}
\begin{itemize}
\item[-]1301: 未读入文件
\end{itemize}

\SACTitle{例子}
获取picks列表,输出为两列显示:
\begin{SACCode}
SAC> lh picks column 2
\end{SACCode}

获得第三、四个文件的默认头段列表:
\begin{SACCode}
SAC> lh files 3 4
\end{SACCode}

列出文件开始和结束时间:
\begin{SACCode}
SAC> lh b e
\end{SACCode}

定义一个包含台站参数的特殊列表:
\begin{SACCode}
SAC> lh KSTNM STLA STLO STEL STDP
\end{SACCode}

稍后再次使用上面的特殊列表:
\begin{SACCode}
SAC> lh SPECIAL
\end{SACCode}

只是设置默认两列输出:
\begin{SACCode}
SAC> lh COLUMNS 2 FILES NONE
\end{SACCode}
