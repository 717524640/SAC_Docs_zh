\section{ticks}
\label{cmd:ticks}

\SACTitle{概要}
控制绘图上时间标记的位置

\SACTitle{语法}
TICKS ON | OFF | ONLY  ALL | Top | Bottom | Right | Left

\SACTitle{输入}
\begin{itemize}
\item ON : 在所列出的边上打开时标,其他不变 
\item OFF : 在所列出的边上关闭时标,其他不变 
\item ONLY : 仅在列出的边上打开时标,其他关闭 
\item ALL : 所有四边都有时标 
\item TOP : 时标位于视口上部的X轴 
\item BOTTOM : 时标位于视口的下部上的X轴 
\item RIGHT : 时标位于视口右边的Y轴 
\item LEFT : 时标位于视口左边的Y轴 
\end{itemize}

\SACTitle{缺省值}
TICKS ON ALL

\SACTitle{说明}
时标可以画在图形四边的一边或几边上。他们以刻度间隔的当前值绘出,其刻度间隔由	XDIV命令控制。时标自动画在由AXES命令指定的轴上。

\SACTitle{例子}
打开顶端时标,其他不变:
\begin{SACCode}
SAC> TICKS ON TOP
\end{SACCode}

关闭所有时标:
\begin{SACCode}
SAC> TICKS OFF ALL
\end{SACCode}

打开底部时标,其余关闭:
\begin{SACCode}
SAC> TICKS ONLY BOTTOM
\end{SACCode}

\SACTitle{相关命令}
XDIV, AXES

\SACTitle{最近修订}
January 8, 1983 (Version 8.0)
