\SACCMD{grayscale}
\label{cmd:grayscale}

\SACTitle{概要}
产生内存中数据的灰度图像

\SACTitle{语法}
\begin{SACSTX}
GRAYSCALE [VIDEOTYPE NORMAL|REVERSED] [SCALE v] [ZOOM n] [XCROP n1 n2 | ON | OFF] 
    [YCROP n1 n2|ON|OFF]
\end{SACSTX}

注意:这个命令使用了未在SAC中发布的命令,要使用这个命令你必须安装Utah Raster Toolkit.。

\SACTitle{输入}
\begin{itemize}
\item VIDEO NORMAL : 设置video类型为normal。在Normal模式中,最小值附近的数据位黑色,最大值附近的数据为白色
\item VIDEO REVERSED : 设置video类型为reversed。在Reversed模式中,最小值附近的数据位白色,最大值附近的数据为黑色
\item SCALE v : 改变数据的比例因子为v,The data is scaled by raising it to the vth power.小于1的值将平滑图像、降低峰和谷,大于1的值将伸展整个数据
\item ZOOM n : Image is increased to n times its normal size by pixel replication.
\item XCROP n1 n2 : Turn x cropping option on and change cropping limits to n1 and n2. The limits are in terms of the image size.
\item XCROP {ON} : Turn x cropping option on and use previously specified cropping limits.
\item XCROP OFF :  Turn x cropping option off.  All of the data in the x direction is displayed.
\item YCROP n1 n2 : Turn y cropping option on and change cropping limits to n1 and n2. The limits are in terms of the image size.
\item YCROP {ON} : Turn y cropping option on and use previous specified cropping limits.
\item YCROP OFF :  Turn y cropping option off.  All of the data in the y direction is displayed. 
\end{itemize}

\SACTitle{缺省值}
\begin{SACDFT}
GRAYSCALE VIDEOTYPE NORMAL SCALE 1.0 ZOOM 1 XCROP OFF YCROP OFF
\end{SACDFT}

\SACTitle{说明}
这个命令可以用于绘制SPECTROGRAM命令输出的灰度图, 用这个命令显示的SAC数据须是"xyz"文件。

注意:SAC启动了一个脚本来运行图像操作和显示程序,然后再显示SAC的提示符。对于大型图像或较慢的机器,这中间会有个明显的延迟。

\SACTitle{限制}
最大只能显示512*1000

\SACTitle{头段变量改变}
需要:IFTYPE, NXSIZE, NYSIZE

\SACTitle{错误消息}
SAC > getsun: Command not found.  (需要Utah Raster Toolkit提供一些工具程序)

\SACTitle{相关命令}
\nameref{cmd:spectrogram}
