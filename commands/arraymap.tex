\SACCMD{arraymap}
\label{cmd:arraymap}

\SACTitle{概要}
利用SAC内存中的所有文件产生一个台阵或联合台阵的分布图

\SACTitle{语法}
\begin{SACSTX}
!ARRAY!MAP [!A!RRAY|!C!OARRAY]
\end{SACSTX}

\SACTitle{输入}
\begin{itemize}
\item ARRAY: 根据头段变量中的偏移X、Y值绘制台站分布
\item COARRAY: 根据各台站之间的相对坐标绘制台站分布图
\end{itemize}

\SACTitle{缺省值}
\begin{SACDFT}
arraymap array
\end{SACDFT}

\SACTitle{头段数据}
下面的两个头段变量必须使用SAC宏文件wrxyz或者与之功能相似的其他函数提前设定,
所有的偏移是相对于某个参考点的千米数。
\begin{itemize}
\item USER7: 向东的偏移(x).
\item USER8: 向北的偏移(y).
\end{itemize}

\SACTitle{说明}
不是很清楚这个命令的作用是什么,对于每个数据来说,需要用宏文件wrxyz定义头段
变量user7和user8,然后才能利用该命令绘制出arraymap,从命令的名字来理解,应该
是绘制某个台站的台站分布图,理论上只需要台站的真实位置即可。不知这个究竟在什
么场合要使用。

\SACTitle{限制}
在bbfk中允许的最多台站数

\SACTitle{相关命令}
wrxyz是一个SAC宏文件,位于~\lstinline{$SACAUX/macros}~中

\SACTitle{最近修订}
July 22, 1991 (Version 10.5c)
