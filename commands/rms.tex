\section{rms}
\label{cmd:rms}

\SACTitle{概要}
利用测量时间窗计算数据的均方根

\SACTitle{语法}
RMS [ NOISE ON | OFF | pdw ] [ TO USERn ]

\SACTitle{输入}
\begin{itemize}
\item NOISE ON : 打开噪声归一化选项  
\item NOISE OFF : 关闭噪声归一化选项  
\item NOISE pdw : 打开噪声归一化选项并改变噪声的部分数据窗``partial data window''。关于pdw的详细信息参考CUT命令 
\item TO USERn : 定义用于储存结果的头段变量USERn,其中n取0到9 
\end{itemize}

\SACTitle{缺省值}
RMS NOISE OFF TO USER0

\SACTitle{说明}
这个命令计算当期测量时间窗(参见MTW)中的数据的均方根。结果写入一个浮点型头段变量USERn中,,如果定义了一个噪声时间窗结果可以用于纠正噪声。计算的一般形式是:对信号窗做一次求和并对可选的噪声窗做另一次求和

\SACTitle{例子}
为了计算两个头段T1和T2间的数据的未修正的均方根,并将结果保存在头段USER4中:
\begin{SACCode}
SAC> MTW T1 T2
SAC> RMS TO USER4
\end{SACCode}
使用一个5秒长的噪声窗(结束于头段值T3),并就算修正后的均方根:
\begin{SACCode}
SAC> MTW T1 T2
SAC> RMS NOISE T3 -5.0 0.0
\end{SACCode}

\SACTitle{头段变量改变}
USERn

\SACTitle{相关命令}
MTW,CUT

\SACTitle{最近修订}
March 22, 1991 (Version 10.6d)

