\section{symbol}
\label{cmd:symbol}

\SACTitle{概要}
控制符号绘图属性

\SACTitle{语法}
SYMbol [ ON | OFF | n ] [ SIZE v ] [ SPACING v ] [ INCREMENT ON | OFF ] [ LIST STANDARD | nlist ]

\SACTitle{输入}
\begin{itemize}
\item ON : 打开符号绘图开关,不改变符号号 
\item OFF : 关闭符号绘图开关 
\item n : 打开符号绘图开关。将符号号设置为n。目前有16个不同的符号,符号0意味着关闭符号开关  
\item SIZE v : 设置符号尺寸为v,值为0.01意味着占据整个绘图尺寸的1\%	
\item SPACING v : 设置符号间隔为v,这是符号之间的最小间隔,如果你想每个数据点都有符号,则其值为0,对注释行使用0.2到0.4
\item INCREMENT {ON} : 对每个数据文件操作完成后按符号列表中先后顺序改变为另一个符号。 
\item INCREMENT OFF : 关闭上述INCREMENT选项 
\item LIST nlist : 改变符号表的内容。输入符号号码表。设置表中的第一个符号码,并打开符号绘图开关 
\item LIST STANDARD : 改变到标准符号表。设置表中的第一个符号表,并打开符号绘制选项。 
\end{itemize}

\SACTitle{缺省值}
SYMBOL OFF SIZE 0.01 SPACING 0. INCREMENT OFF LIST STANDARD

\SACTitle{说明}
这些符号属性独立于由LINE命令定义的画线属性。打开画线选项,它们也可以用于注释在相同图形上的不同的线。关闭画线选项,则可以绘制散点图。如果你要将几个数据文件画在同一张图上,也许需要使用不同的符号。这是可以使用INCREMENT选项。当这个选项打开时,每次绘制数据文件,都从符号表中将原来的符号码增加1,缺省符号表包含符号表从2到16的符号,你也可以使用LIST选项改变这个表的次序和内容。如果你在绘制一系列重叠绘图,并需要经相同符号用在相同次序的每个图形上,这样做很有用。符号码为0相当于关闭符号绘制选项。这个选项用于LIST选项和LINE选项,以在一张图上用线表示一些数据,用符号表示另外一些数据。

\SACTitle{例子}
为了创建一个散点分布图,关闭画线选项,选择适当的符号,然后绘图:
\begin{SACCode}
SAC> LINE OFF
SAC> SYMBOL 5
SAC> PLOT
\end{SACCode}

为了用符号7、4、6、8注释四条实线,间隔用0.3,用PLOT2绘图:
\begin{SACCode}
SAC> LINE SOLID
SAC> SYM SPACING .3 INCREMENT LIST 7 4 6 8
SAC> R FILE1 FILE2 FILE3 FILE4
SAC> PLOT2
\end{SACCode}

使用PLOT2在相同图形上绘制三个文件,第一个文件图形使用实线无符号;第二个没有线,为三角符号;第三个没有线,带有交叉符号:
\begin{SACCode}
SAC> READ FILE1 FILE2 FILE3
SAC> LINE LIST 1 0 0 INCREMENT
SAC> SYMBOL LIST 0 3 7 INCREMENT
SAC> PLOT2
\end{SACCode}

\SACTitle{相关命令}
LINE

\SACTitle{最近修订}
October 11, 1984 (Version 9.1)
