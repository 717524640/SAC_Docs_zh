\SACCMD{quantize}
\label{cmd:quantize}

\SACTitle{概要}
将连续数据数字化\footnote{无法理解这个命令,请阅读官方文档}

\SACTitle{语法}
\begin{SACSTX}
QUANTIZE [GAINS n ...] [LEVEL v] [MANTISSA n]
\end{SACSTX}

\SACTitle{输入}
\begin{description}
\item [GAINS n ...] 设置允许的增益表,必须单调递减的,增益的最大数目为8 
\item [LEVEL v] 设置最低增益的量化水平,即Least Significant Bit,单位为伏特 
\item [MANTISSA n] 设置Bits,用其尾数表示。
\end{description}

\SACTitle{缺省值}
\begin{SACDFT}
quantize gains 128 32 8 1 level 0.00001 mantissa 14
\end{SACDFT}

\SACTitle{说明}
此命令演示了与Oppenheim和Schafer(1975, Fig. 9.1)描述的``rounding''量化算法类似的
一个量化算法。

算法中的位数通过其尾数表示,则若尾数mantissa为$n$,则表示其所能允许的范围是
$\pm \frac{2^n-1}{2}$。

量化水平level即least significant bit的值,表征了系统所能识别的最小的电压变化,
缺省值为10微伏。

这个量化函数描述的信号误差是量化水平的一半。在频率域中,这个误差或量化噪声为:
\[
    Err = \frac{1}{12}(Delta*Level^2)
\]

其中Delta是采样周期。量化噪声的单位是counts*counts/Hz,相当于功率谱密度。
量化噪声的均方根为$\frac{1}{6}Level^2$。然而,仅当信号的均方根水平远大于量化噪声的
均方根值时,才是量化噪声的一个精确的近似。

\SACTitle{头段变量}
depmin、depmax、depmen
