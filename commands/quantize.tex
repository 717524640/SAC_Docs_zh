\SACCMD{quantize}
\label{cmd:quantize}

\SACTitle{概要}
将连续数据数字化

\SACTitle{语法}
QUANTIZE [ GAINS n ... ] [ LEVEL v ] [ MANTISSA n ]

\SACTitle{输入}
\begin{itemize}
\item GAINS n ...: 设置允许的增益表,他们必须递减的,增益最大值为8 
\item LEVEL v :  设置最低增益的量子化水平,这是最小有效位数的伏特值 
\item MANTISSA n : 设置尾数的位数 
\end{itemize}

\SACTitle{缺省值}
QUANTIZE GAINS 128 32 8 1 LEVEL 0.00001 MANTISSA 14

\SACTitle{说明}
这个命令类似于Oppenheim和Schafer(1975, Fig. 9.1)描述的``rounding'' quantization算法。在这个算法中位数划分为多个用于描述特征的位、符号位和尾数位。用户可以指定尾数的位数。量化水平(最小有效数字位或LSB)也可由用户指定,缺省的量化水平为10微伏,这个量化函数描述的信号误差是这个量化水平的一半。在频率域中,这个误差或量化噪声为:

其中DELTA是采样频率。量化噪声单位是counts*counts/Hz,相当于功率谱密度。量化噪声的均方根为:
然而,仅当信号的均方根水平远大于量化噪声的均方根值时,这才是量化引起的噪声的一个精确的近似。换句话说,如果分辨率不是几百个counts那么在量化噪声与量化的	信号之间便是有关联的。关联系数近似于LEVEL与量化信号的均方根之比。( Fig. 11.13, 	Oppenheim and Schaffer, 1975).

增益可以由用户指定,以在自动增益系统中模拟增益步骤,缺省的增益是区域地震实验台网的增益。

\SACTitle{参考文献}
Oppenheim, Alan V., and Ronald W. Schafer;Digital Signal Processing; Prentice-Hall; 1975; 585pp.

\SACTitle{头段变量改变}
DEPMIN, DEPMAX, DEPMEN

\SACTitle{最近修订}
May 15, 1987 (Version 10.2)
