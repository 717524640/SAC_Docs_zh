\SACCMD{deletechannel}
\label{cmd:deletechannel}

\SACTitle{概要}
从内存中文件列表中删去一个或多个文件

\SACTitle{语法}
\begin{SACSTX}
D!ELETE!C!HANNEL! ALL
\end{SACSTX}
或
\begin{SACSTX}
D!ELETE!C!HANNEL! FILENAME|FILENUMBER|RANGE [FILENAME|FILENUMBER|RANGE ... ]
\end{SACSTX}

\SACTitle{输入}
\begin{itemize}
\item ALL : 删除内存中全部文件,用户不需要指定文件名或文件号
\item filename : 要删除的内存文件列表中的文件名
\item filenumber : 文件列表中指定文件的文件号,第一个文件是1,第二个文件是2等等
\item range : 用破折号隔开的两个文件号,例如11-20.
\end{itemize}

\SACTitle{例子}
\begin{SACCode}
  dc 3 5                         // 删除第三、五个文件
  dc SO01.sz SO02.sz             // 删除这些名字的文件
  dc 11-20                       // 删除11-20的全部文件
  dc 3 5 11-20 SO01.sz SO02.sz   // 删除上面的全部
\end{SACCode}

\SACTitle{错误消息}
\begin{itemize}
\item[-]5106: 文件名不在文件列表中
\item[-]5107: 文件号不在文件列表中
\end{itemize}

\SACTitle{相关命令}
\nameref{cmd:filenumber}
