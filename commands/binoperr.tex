\SACCMD{binoperr}
\label{cmd:binoperr}

\SACTitle{概要}
控制发生在二进制文件操作中的错误

\SACTitle{语法}
\begin{SACSTX}
BINOPERR [N!PTS! F!ATAL!|W!ARNING!|I!GNORE!] [D!ELTA! F!ATAL!|W!ARNING!|I!GNORE!]
\end{SACSTX}

\SACTitle{输入}
\begin{itemize}
\item NPTS: 改变由于数据点不相等的错误条件 
\item DELAT: 改变由于采样间隔不相等的错误条件 
\item FATAL: 是错误条件为致命性条件。一旦发生错误,控制立即返回终端,忽略剩余的全部命令。
\item WARNING: 给用户发送警告消息,校正错误条件并继续执行
\item IGNORE: 纠正错误条件继续执行
\end{itemize}

\SACTitle{缺省值}
\begin{SACDFT}
binoperr npts fatal delta fatal
\end{SACDFT}

\SACTitle{说明}
在对文件执行二元运算(addf、divf等)时,SAC会检测两个文件的数据点数和采样周期是否匹配。

使用这个命令,你可以控制SAC在遇到不匹配时如何处理。如果你设置错误条件为致命,
那么SAC在遇到这种错误时将停止执行当前命令,忽略接下来的所有命令,打印错误信息到终端
并将控制权交还。如果设置错误条件为警告,则发送一个警告消息,尽可能纠正错误并继续执行。
如果设置错误条件为忽略,则SAC会纠正错误并继续而不告诉你发生了什么。

若要操作的两个数据文件数据点数不匹配,SAC会使用数据点最少的那个文件
的数据点数作为参考,以保证正常操作。

若要操作的两个文件采样周期不匹配,则SAC会使用第一个数据文件的采样周期作为结果文件的采样周期。

\SACTitle{例子}
假定FILe1有1000个数据点,file2有950个数据点::
\begin{SACCode}
SAC> binoperr npts fatal
SAC> read file1
SAC> addf file2
ERROR:  Header field mismatch: NPTS file1 file2
\end{SACCode}

上例文件加法未执行,假设你输入:
\begin{SACCode}
SAC> binoperr npts warning
SAC> addf file2
WARNING:  Header field mismatch: NPTS file1 file2
\end{SACCode}
文件加法对每个文件的前50个数据执行了加法操作。

\SACTitle{相关命令}
\nameref{cmd:addf}、\nameref{cmd:subf}、\nameref{cmd:mulf}、\nameref{cmd:divf}
