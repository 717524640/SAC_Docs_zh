\SACCMD{filterdesign}
\label{cmd:filterdesign}

\SACTitle{概要}
产生一个滤波器的数字和模拟特性的图形显示,包括:振幅,相位,脉冲响应和群延迟。

\SACTitle{语法}
FilterDesign [ PRINT [ pname ] ] [ FILE [ prefix ] ] [ filteroptions ] [ delta ]

其中filteroptions与SAC中其他的滤波命令相同,包括滤波器类型。
delta是数据的采样间隔。

**注意**选项的顺序很重要,如果使用了PRINT选项,则其必须是第一个选项。如果使用了FILE选项,其必须放在滤波器选项之前。

\SACTitle{输入}
\begin{itemize}
\item PRINT pname : 将输出结果打印到打印机pname,若未给定名字,则使用默认打印机。
\item FILE prefix : 将三个SAC文件(后缀为spec、gd、imp)写入磁盘,这些文件包括该命令的数字响应。 
	\begin{itemize}
	\item prefix.spec:该命令产生的振幅相位信息,为振幅-相位格式谱文件。
	\item prefix.gd:该命令产生的群延迟信息,是时间序列文件。**注意** 尽管这个文件是时间序列文件,但是实际上群延迟是频率的函数。用户要记住,虽然绘图时横轴单位是秒,实际的单位却是Hz
	\item prefix.imp: 时间序列文件,包含脉冲响应信息
	\end{itemize}
\end{itemize}

在这三个SAC文件中,用户自定义头段变量USERn、KUSERn设置如下:
\begin{itemize}
\item user0:  pass code
	\begin{itemize}
	\item 1 low pass
	\item 2 high pass
 	\item 3 band pass
	\item 4 band reject
	\end{itemize}
\item user1:  type code
	\begin{itemize}
	\item 1 Butterworth
	\item 2 Bessel
	\item 3 C1
	\item 4 C2
	\end{itemize}
\item user2:  number of poles
\item user3:  number of passes
\item user4:  tranbw
\item user5:  attenuation
\item user6:  delta
\item user7:  first corner
\item user8:  second corner if present, or -12345 if not
\item kuser0: pass (lowpass, highpass, bandpass, or bandrej)
\item kuser1: type (Butter, Bessel, C1, or C2 )
\end{itemize}

\SACTitle{缺省值}
只有delta参数有一个缺省值(0.025s)。其他参数必须给出

\SACTitle{说明}
FILTERDESIGN命令使用了一个工具程序xapiir,它是一个递归数字滤波器包。xapiir通过模拟滤波器原型的双线性变换实现标准递归数字滤波器的设计。这些原型滤波器由零点和极点给定,然后使用模拟谱变换,变换到高通、带通和带阻滤波器。

FILTERDESIGN用实线显示数字滤波器响应,用虚线显示模拟滤波器响应。在彩色显示器上,数字曲线是蓝色的而模拟曲线是琥珀色的。

\SACTitle{例子}
下面的例子展示了如何使用FILTERDESIGN命令产生一个高通,拐角频率为2Hz,六极、双通滤波器的数字和模拟响应曲线,数据采样间隔为0.025s:
\begin{SACCode}
SAC> fd hp c 2 n 6 p 2 delta .025
\end{SACCode}

\SACTitle{相关命令}
HIGHPASS, LOWPASS, BANDPASS, BANDREJECT

\SACTitle{参考文献}
UCRL-ID-106005. XAPiir: A Recursive Digital Filtering Package. David Harris.
September 21, 1990 In Xwindows,

\SACTitle{最近修订}
July 22, 1991 (Version 0.58)
