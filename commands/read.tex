\SACCMD{read}
\label{cmd:read}

\SACTitle{概要}
从磁盘读取SAC文件到内存

\SACTitle{语法}
\begin{SACSTX}
R!EAD! [MORE] [DIR CURRENT|name] [XDR|ALPHA|SEGY] [SCALE ON|OFF] [filelist]
\end{SACSTX}
所有的选项必须位于filelist之前。

\SACTitle{输入}
\begin{description}
\item [MORE] 将读入的新文件添加到内存中老文件之后。若选项此忽略,则读入的新数据将替代
    内存中的老数据
\item [DIR CURRENT] 从``当前目录''读取文件列表中的文件。``当前目录''为启动SAC的目录
\item [DIR name] 从目录name中读取文件列表中的文件,其可以为绝对路径或相对路径
\item [XDR] 读取XDR格式的文件。此格式用于实现不同构架的二进制数据的转换 
\item [ALPHA] 输入文件是SAC的字符数字型文件,该选项与XDR选项不兼容
\item [SEGY] 读取IRIS/PASSCAL定义的SEGY格式文件。该格式允许一个文件包含一个波形 
\item [SCALE] 只能和SEGY选项搭配使用,该选项默认是关闭的。当SCALE选项为OFF时,
    SAC直接从SEGY文件中读取数据值;当SCALE为ON时,SAC将每个数据值乘以以文件中给定的
    SCALE因子。若SCALE为OFF,则这个文件中的SCALE值将储存在SAC头段SCALE中;若SCALE为ON,
    SAC的SCALE头段将被设置为1.0。
\item [filelist] 文件列表。可以是简单的文件名,也可以包含相对或绝对路径,也可以使用
    通配符。
\end{description}

\SACTitle{缺省值}
\begin{SACDFT}
read dir current
\end{SACDFT}

\SACTitle{说明}
该命令将SAC文件从磁盘读入到内存中,默认状态下会读取每个磁盘文件中的全部数据点。

CUT命令可以用于指定读取文件的一部分数据。在2000年之后产生的SAC文件会被假定年份为四位数字。
年份为两个数字的文件被假定为20世纪,会被加上1900。

在使用read命令时,正常情况下内存中的老数据会被新读取的数据所替代。若使用more选项,则新数据
将被读入内存并放在老数据的后面。在如下三种情况下more选项可能会有用:
\begin{itemize}
\item 文件列表太长无法在一行中键入
\item 在长文件列表中某个文件名拼错而没有读入,可以使用more选项再次读入
\item 一个文件被读入,作了些处理,然后与原始数据比较
\end{itemize}

\SACTitle{例子}
read命令的简单示例位于~``\nameref{sec:read-and-write}''~一节。

如果你想要对一个数据进行高通滤波,并与原始数据进行对比:
\begin{SACCode}
SAC> r f01
SAC> hp c 1.3 n 6
SAC> r more f01
SAC> p1
\end{SACCode}

假设SAC的启动目录位于``/me/data'',你想要处理其子目录``event1''和``event2''下的文件。
\begin{SACCode}
SAC> read dir event1 f01 f02
\end{SACCode}
读取了目录``/me/data/event1''下的文件。

\begin{SACCode}
SAC> read f03 g03
\end{SACCode}
相同目录下的文件被读入。

\begin{SACCode}
SAC> read dir event2 *
\end{SACCode}
``/me/data/event2''下的全部文件被读入:

\begin{SACCode}
SAC> read dir current f03 g03
\end{SACCode}
目录``/me/data''下的文件被读入。

\SACTitle{错误消息}
\begin{itemize}
\item[-]1301: 未读入数据文件(未给出要读取的文件列表或文件列表中文件均不可读)
\item[-]1320: 可用内存不足以读取文件
\item[-]1314: 数据文件列表不得以数字开头
\item[-]1315: 文件列表的最大数目为1000
\item[-]6002: 没有可用数据集
\end{itemize}

\SACTitle{警告消息}
\begin{itemize}
\item[-]0101: 打开文件
\item[-]0108: 文件不存在
\item[-]0114: 读取文件
\end{itemize}

\SACTitle{头段变量}
e、depmin、depmax、depmen、b 

\SACTitle{相关命令}
\nameref{cmd:cut}、\nameref{cmd:readerr}、\nameref{cmd:wild}
