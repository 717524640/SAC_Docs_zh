\SACCMD{yfudge}
\label{cmd:yfudge}

\SACTitle{概要}
改变Y轴的``插入因子''

\SACTitle{语法}
\begin{SACSTX}
YFUDGE ON|OFF|v
\end{SACSTX}

\SACTitle{输入}
\begin{itemize}
\item ON : 打开插入选项,但不改变插入因子 
\item OFF : 关闭插入选项 
\item v : 打开插入因子,改变插入因子为v 
\end{itemize}

\SACTitle{缺省值}
\begin{SACDFT}
yfudge 0.03
\end{SACDFT}

\SACTitle{说明}
当这个选项打开时,实际轴范围将由插入因子改变,确定线性坐标轴范围的方法是:
\[ YDIFF=YFUDGE*(YMAC-YMIN) \]
\[ YMIN=YMIN-YDIFF \]
\[ YMAC=YMAX+YDIFF \]
其中YMIN和YMAX是数据的极值,YFUDGE是插入因子,这个算法对于对数坐标也是相似的。插入选项值应用于坐标轴的范围设置为数据极值时(参见YLIM)

\SACTitle{相关命令}
\nameref{cmd:ylim}
