\section{plotpm}
\label{cmd:plotpm}

\SACTitle{概要}
针对一对数据文件产生一个``质点运动''图

\SACTitle{语法}
PlotPM [ PRINT pname ]

\SACTitle{输入}
\begin{itemize}
\item PRINT pname : 打印结果到打印机
\end{itemize}

\SACTitle{说明}
在一个质点运动图中,一个等间距文件相对于另一个等间距文件绘图。对于没有自变量的值,第一个文件的因变量的值将沿着y轴绘制,第二个文件的因变量值沿着x轴绘制。对于一对地震图来说,这种图展示了一个质点在两个地震图所确定的平面内随时间的运动。它将产生一个方形图,每个轴的范围为因变量的极大极小值。注释轴沿着底部	和左边绘制。轴标签以及标题可以通过XAXIS, YAXIS和TITLE命令设置。如果未设置x、y标签,则台站名和方位角将用作轴标签。XLIM可以用于控制文件的哪些部分需要被绘制。

\SACTitle{例子}
创建一个两个地震图的质点运动图:
\begin{SACCode}
SAC> READ XYZ.T XYZ.R
SAC> XLABEL 'Radial component'
SAC> YLABEL 'Transverse component'
SAC> TITLE 'Particle-motion plot for station XYZ'
SAC> PLOTPM
\end{SACCode}

如果你想要值绘制每个文件在初动附近的一部分,你可以使用XLIM命令:
\begin{SACCode}
SAC>  XLIM A -0.2 2.0
SAC>  PLOTPM
\end{SACCode}
你也可以使用PLOTPK在之前的绘图窗口中设置新的区域,然后绘制命令如下:

\begin{SACCode}
SAC> BEGINWINDOW 2
SAC> PLOTPK
SAC> ... mark the portion you want using X and S
SAC> ... terminate PLOTPK with a Q
SAC> BEGINWINDOW 1
SAC> PLOTPM
\end{SACCode}

\SACTitle{最近修订}
May 15, 1987 (Version 10.2)
