\SACCMD{bandrej}
\label{cmd:bandrej}

\SACTitle{概要}
应用一个无限脉冲带阻滤波器

\SACTitle{语法}
BandRej [BUtter|BEssel|C1|C2] [Corners v1 v2] [Npoles n] [Passes n] [Tranbw v] [Atten v]

\SACTitle{输入}
\begin{itemize}
\item BUtter: 应用一个Butterworth滤波器
\item BEssel:  应用一个Bessel滤波器
\item C1: 应用一个Chebyshev I型滤波器
\item C2: 应用一个Chebyshev II滤波器
\item Corners v1 v2: 设定拐角频率分别为V1和V2
\item Npoles n: 设置极数为N, 范围: 1-10
\item Passes n: 设通道数为N, 范围: 1-2
\item Tranbw v: 设置Chebyshev转换带宽为v
\item Atten v: 设置Chebyshev衰减因子为v
\end{itemize}

\SACTitle{缺省值}
BANDREJ BUTTER CORNER 0.1 0.4 NPOLES 2 PASSES 1 TRANBW 0.3 ATTEN 30

\SACTitle{说明}
参见命令bandpass的说明

\SACTitle{例子}
为应用一个四极Butterworth,拐角频率为2 、5 Hz.:
\begin{SACCode}
SAC> br n 4 c 2 5
\end{SACCode}
在此之后如果要应用一个二极双通具有相同频率的Bessel:
\begin{SACCode}
SAC> br n 2 be p 2
\end{SACCode}

\SACTitle{错误消息}
\begin{itemize}
\item[-]1301: 未读入文件
\item[-]1306: 对不等间隔文件的非法操作
\item[-]1307: 对谱文件的非法操作
\item[-] 1002: 由于拐角频率大于Nyquist频率而出现的坏值
\end{itemize}

\SACTitle{头段变量改变}
DEPMIN, DEPMAX, DEPMEN

\SACTitle{有关命令}
BANDPASS

\SACTitle{最近修订}
January 8, 1983 (Version 8.0)
