\SACCMD{mtw}
\label{cmd:mtw}

\SACTitle{概要}
决定接下来命令中所使用的测量时间窗

\SACTitle{语法}
\begin{SACSTX}
MTW [ON|OFF|pdw]
\end{SACSTX}

\SACTitle{输入}
\begin{itemize}
\item ON : 打开测量时间窗选项但不改变窗的值 
\item OFF : 关闭测量时间窗选项。测量将对整个文件进行操作 
\item pdw : 打开测量时间窗并设置其新值。pdw包含你想要进行测量的自变量的范围,详细信息参考CUT命令 
\end{itemize}

\SACTitle{缺省值}
\begin{SACDFT}
mtw off
\end{SACDFT}

\SACTitle{说明}
当这个选项为开时,测量只对这个窗内有效,当这个选项为关时,测量针对整个文件。这个选项目前只对markptp和markvalue命令有效

\SACTitle{例子}
pdw的一些例子如下:
\begin{itemize}
\item B 0 30: 文件起始的30s
\item A -10 30: 初动到时前10s到初动后30s
\item T3 -1 T7: 时间标记T3前1s到T7
\item B N 2048: 文件最初的2048个数据点
\item 30.2 48: 相对文件零时的30.2s到48s
\end{itemize}

\SACTitle{相关命令}
\nameref{cmd:cut}、\nameref{cmd:markptp}、\nameref{cmd:markvalue}
