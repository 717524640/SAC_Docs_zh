\section{plotpk}
\label{cmd:plotpk}

\SACTitle{概要}
产生一个用于拾取到时的图

\SACTitle{语法}
PlotPK [ Perplot ON | OFF | n ] [ BELL ON | OFF ] [ Absolute | Relative ] [ Reference ON | OFF | v ] [ Markall ON | OFF ] [ Savelocs ON | OFF ]

\SACTitle{输入}
\begin{itemize}
\item PERPLOT ON : 一张图绘制n个文件,使用n的旧值 
\item PERPLOT OFF : 在一张图上绘制所有文件 
\item PERPLOT n : 一张图上绘制n个文件,比如拾取震相时常使用n=3(三分量) 
\item BELL {ON}|OFF :  击键时是否响铃 
\item ABSOLUTE : 显示绝对GMT格式时间 
\item RELATIVE : 显示相对于每个文件的参考时间 
\item REFERENCE {ON} : 打开参考线显示 
\item REFERENCE OFF : 关闭参考线选项 
\item REFERENCE v : 打开参考线选项并设置参考值为v,设置为0会很方便 
\item MARKALL {ON} : 同时储存一张图上的所有文件的时间标记,用于拾取震相 
\item MARKALL OFF : 只保存光标位置所在的那个文件的震相拾取标记 
\item SAVELOCS {ON} : 保存拾取的位置(由命令L得到,稍后介绍)到暂存块变量中 
\item SAVELOCS OFF :  不保存拾取位置到暂存块变量 
\end{itemize}

\SACTitle{其他形式}
GMT可以用于代替ABSOLUTE,ZERO可以代替RELATIVE.

\SACTitle{缺省值}
PLOTPK PERPLOT OFF ABSOLUTE REFERENCE OFF MARKALL OFF SAVELOCS OFF

\SACTitle{说明}
这个命令的格式类似于PLOT1,这个绘图需要一个带有光标的终端。在该命令后将在屏幕上出现光标,输入单个字符即可执行不同的操作。有些但不是全部字符将在屏幕上产生图形输出。错误以及输出信息将在屏幕上部打印,当前在头段中的震相拾取将自动	显示在屏幕上。不同的光标响应的输出可以定向到SAC头段中,也可以定向到一个字符数字型震相拾取文件(OAPF),或者到一个HYPO震相拾取文件,或者到终端。如果你使用SAVELOCS选项保存由L光标选项得到的光标位置到暂存块变量,那么下面的暂存块变量将被定义:
\begin{itemize}
\item NLOCS: 在命令执行期间拾取的光标位置数,每次执行PPK时初始化为0,每次拾取光标位置则加1
\item XLOCn: 拾取光标位置的x值,如果头段中的参考时间被定义其是GMT时间,否则其为偏移时间
\item YLOCn: 第n个拾取光标位置的y值
\end{itemize}

\SACTitle{头段变量改变}
A, KA, F, KF, Tn, KTn.

\SACTitle{错误消息}
\begin{itemize}
\item[-]1301: 未读入数据文件
\item[-]1202: 变量块个数超过了最大数目
\end{itemize}

\SACTitle{警告消息}
\begin{itemize}
\item[-]1502: 错误的光标位置,请重新尝试
	\begin{itemize}
	\item[-]光标位于绘图窗口之外
	\end{itemize}

\item[-]1503: 无效字符,请重新尝试
	\begin{itemize}
 	\item[-]输入的字符SAC无法识别为合法响应
	\end{itemize}
\item[-]1905: 需要一个整数,请重新尝试
	\begin{itemize}
 	\item[-]在响应T之后未输入整数
	\end{itemize}
\item[-]1906: 需要一个0到4的正式,请重新尝试
	\begin{itemize}
  	\item[-]在响应Q之后未输入0, 1, 2 或 3
	\end{itemize}
\item[-]1907: HYPO行已经写入
\item[-]1908: HYPO震相拾取文件未打开
\item[-] 1909: 无法计算波形
	\begin{itemize}
  	\item[-]调整光标位置并重试,绘图总是ABSOLUTE模式
	\end{itemize}
\end{itemize}

\SACTitle{相关命令}
PLOT1 , OHPF, OAPF, APK, PLOTPKTABLE

\SACTitle{最近修订}
May 16, 200 (version 100.1)

\begin{table}
\centering
\caption{PLOTPK选项表}
\begin{tabular}{p{1cm}p{14cm}}
	\toprule
	字符	&	含义	\\
	\midrule
	A	&	定义头段中的初至[1,7]	\\
	B	&	如果有,则显示上一张绘图	\\
	C	&	计算事件的初至和结束[1,4,7]	\\
	D	&	设置震相方向为DOWN	\\
	E	&	设置震相起始为EMERGENT(急始)	\\
	F	&	定义事件结束[1,2,3,7]	\\
	G	&	以HYPO格式将拾取显示到终端[4]	\\
	H	&	将拾取写成HYPO格式[3,4]	\\
	I	&	设置震相起始为IMPULSIVE	\\
	J	&	设置噪声水平[2,6,8]	\\
	L	&	显示光标当前位置[2,4]	\\
	M	&	计算最大振幅波形[2,3,5]	\\
	N	&	显示下一绘图	\\
	O	&	显示前一个绘图窗,最多可以保存5个绘图窗	\\
	P	&	定义P波到时[1,2,3,7]	\\
	Q	&	结束PLOTPK	\\
	S	&	定义S波到时[1,2,3,7]	\\
	T	&	用户自定义到时Tn,输入T之后需要输入0到9中的任一数	\\
	U	&	设置震相方向为UP	\\
	V	&	定义一个Wood-Anderson波形[2,5]	\\
	W	&	定义一个通用波形[2,5]	\\
	X	&	对当前plot使用一个新的x轴时间窗。简单说就是放大。新绘图从当前
			光标位置开始。然后将光标移动到新位置,并键入任意键。新绘图将以
			光标当前位置作为结束。如果第二次键是S,那么新的x轴方位就被保存
			并一直用于接下来的一系列绘图中。	\\
	Z	&	设置参考水平[2,6,8]	\\
	\	&	删除当前全部拾取的定义。当一个文件中包含多个事件时有用。	\\
	+	&	设置震相方向为PLUS	\\
	-	&	设置震相方向为MINUS	\\
	space &	设置震相方向为NEUTRAL	\\
	n	&	设置震相质量为n,n取0-4	\\
	\bottomrule
\end{tabular}
\end{table}

\SACTitle{PLOTPK命令选项说明}
\begin{enumerate}
\item 写入SAC头段
\item 写入字符数字型震相拾取文件(如果已打开)
\item 写入HYPO震相拾取文件(如果已打开)
\item 写入终端
\item 在参考水平处放置水平光标,第一个最值处放置垂直光标。终端回显包含波形的盒子
\item 在指定的水平处放置水平光标
\item 终端回显有标签的垂直线
\item 终端回显有标签的水平线
\end{enumerate}
