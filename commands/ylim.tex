\SACCMD{ylim}
\label{cmd:ylim}

\SACTitle{概要}
设定图形中y轴的范围

\SACTitle{语法}
\begin{SACSTX}
YLIM [ON|OFF|ALL|min max|PM v]
\end{SACSTX}

\SACTitle{输入}
\begin{description}
\item [ON] 打开y轴范围选项,但不改变范围值 
\item [OFF] 关闭y轴范围选项
\item [ALL] 根据内存中所有文件的最大和最小值限定y轴的范围
\item [min max] 设定y轴的范围为min到max之间
\item [PM v] 打开y轴范围选项,并设置y轴的范围为-v到+v之间。
\end{description}

\SACTitle{缺省值}
\begin{SACDFT}
ylim off
\end{SACDFT}

\SACTitle{说明}
当此选项关闭时,根据数据因变量的范围决定绘图时y轴的范围。当此选项打开时,限定y轴的范围,
也可以根据内存中的整个数据集来限定y轴的范围。你可能想要对内存中不同的
文件设定不同的y轴范围,此时可以多次使用这些选项。第一个条目用于第一个文件,第二个条目
用于第二个文件,最后一个条目用于内存中余下的全部文件。

\SACTitle{示例}
\begin{SACCode}
SAC> ylim 0.0 30.0 all off
SAC> r file1 file2 file3
SAC> p
\end{SACCode}
file1的y轴范围为0.0到30.0,file2的y轴范围为内存中所有文件的最大、最小值,file3的
y轴范围将限定为文件自身的最大、最小值。如果文件多于三个,则其余的所有文件都限定
为文件自身的最大、最小值。

\SACTitle{相关命令}
\nameref{cmd:cut}
