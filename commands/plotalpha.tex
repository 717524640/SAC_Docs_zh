\SACCMD{plotalpha}
\label{cmd:plotalpha}

\SACTitle{概要}
从磁盘读入字符数据型文件到内存并将数据绘制出来,这个与readtable之后跟个plot不同,因为它允许你在每个数据点上绘制标签

\SACTitle{语法}
\begin{SACSTX}
P!LOT!A!LPHA! [MORE] [DIR CURRENT|name] [FREE|FORMAT text] [CONTENT text] 
    [PRINT printer] [filelist]
\end{SACSTX}

\SACTitle{输入}
\begin{itemize}
\item MORE : 将新读入的文件加到内存中老文件之后。如果没有这个选项,新文件将代替内存中的老文件。参见READ命令。
\item DIR CURRENT : 从当前目录读取并绘制所有文件 
\item DIR name : 从文件夹name中读取并绘制所有文件,其可以是相对或绝对路径 
\item FREE : 以自由格式(以空格分隔数据各字段)读取并绘制filelist中的数据文件 
\item FORMAT text : 以固定格式读取并绘制filelist中的数据文件,格式声明位于text中 
\item CONTENT text : 定义filelist中数据每个字段的含义。text的含义参见READTABLE命令中的 
\item PRINT pname : 打印输出结果到打印机 
\item filelist : 字符数字型文件列表,其可以包含简单文件名,绝对/相对路径,通配符。 
\end{itemize}

\SACTitle{缺省值}
\begin{SACDFT}
plotalpha free content y. dir current
\end{SACDFT}

\SACTitle{说明}
参考readtabel命令的相关说明。

\SACTitle{例子}
读取并绘制一个自由格式的X-Y数据,且其第一个字段是标签:
\begin{SACCode}
SAC> plotalpha content lp filea
\end{SACCode}

\SACTitle{错误消息}
\begin{itemize}
\item[-]1301: 未读入数据文件(未给出文件列表或列表中的文件不可读)
\item[-]1020: 无效的内置函数名
\item[-]1320: 可用内存太小不足以读取文件
\item[-]1314: 数据文件列表不能以数字开头
\item[-]1315: 数据文件列表中文件的最大数目为:
\end{itemize}

\SACTitle{警告消息}
\begin{itemize}
\item[-]0101: 打开文件
\item[-]0108: 文件不存在
\end{itemize}

\SACTitle{头段变量改变}
b, e, delta, leven, depmin, depmax, depmen.

\SACTitle{相关命令}
\nameref{cmd:read}、\nameref{cmd:write}、\nameref{cmd:readtable}
