\section{writesp}
\label{cmd:writesp}

\SACTitle{概要}
将谱文件作为一般文件写入磁盘

\SACTitle{语法}
WriteSP [ ASIS | RLIM | AMPH | RL | IM | AM | PH ] [ COMMIT|ROLLBACK|RECALLTRACE]  [ OVER | filelist ]
 
\SACTitle{输入}
\begin{itemize}
\item ASIS :  按照谱文件当前格式写入 
\item RLIM :  写入实部和虚部分量 
\item AMPH :  写入振幅和相位分量 
\item RL :  只写入实部分量 
\item IM :  只写入虚部分量 
\item AM :  只写入振幅分量 
\item PH :  只写入相位分量 
\item filelis :  SAC二进制数据文件列表,这个列表可以包含简单文件名和绝对/相对路径名
\end{itemize}

\SACTitle{缺省值}
WRITESP ASIS COMMIT

\SACTitle{说明}
SAC数据文件可以为时间序列文件或谱文件。头段中的IFTYPE用于区分这两种格式。当你读取一个时间序列到内存,对其做快速Fourier变换,然后将数据写回磁盘,此时的文件即为谱文件。

某些操作只能对时间序列文件进行,而某些操作只能对谱文件进行。比如,你无法对一个谱文件应用taper命令或者将两个谱文件乘起来。这是SAC的保护机制。

然而有时你需要对谱文件做这些操作,为了越过SAC的保护机制,你可以使用这个命令将谱文件像时间序列数据一样写入磁盘。每一个分量都将作为一个单独文件写入磁盘。然后你可以将这些文件读入SAC并进行任何你想要的操作。因为SAC认为其为时间序列文件。一旦这些计算完成了,你可以将修改之后的数据通过WRITE命令写回磁盘。如果你想要读回这个谱文件,可以使用READSP命令。

为了帮助你跟踪磁盘上的数据,SAC将在你给出的文件名后加一个后缀以标识储存在文件的谱分量。后缀分别为``.RL'', ``.IM'', ``.AM''和``.PH''分别对应不同的分量。

\SACTitle{例子}
假设你想要对FILE1的谱文件振幅进行一些操作:
\begin{SACCode}
SAC> READ FILE1
SAC> FFT AMPH
SAC> WRITESP OVER
\end{SACCode}

SAC将输出两个文件FILE1.AM和FILE1.PH,现在可以对振幅文件进行操作:
\begin{SACCode}
SAC> READ FILE1.AM
SAC> ...perform operations.
SAC> WRITE OVER
\end{SACCode}

现在磁盘中的文件为修改后的谱文件,如果你想要重建SAC谱数据并进行反变换:
\begin{SACCode}
SAC> READSP FILE1
SAC> IFFT
SAC> WRITE FILE2
\end{SACCode}

\SACTitle{错误消息}
\begin{itemize}
\item[-]1301: 未读入数据文件
\item[-]1305: 对时间序列文件非法操作
\end{itemize}

\SACTitle{头段变量改变}
磁盘文件中的B, E, DELTA将包含频率的起始值、结束值和增值,单位Hz

\SACTitle{相关命令}
READSP, COMMIT, ROLLBACK, RECALLTRACE

\SACTitle{最近修订}
Oct. 27, 1998 (Version 0.58)
