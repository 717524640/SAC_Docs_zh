\SACCMD{apk}
\label{cmd:apk}

\SACTitle{概要}
应用一个自动事件拾取算法(由连续信号判断是否其中是否包含地震事件)

\SACTitle{语法}
APK [param v [param v] ... ] [Validation ON|OFF]

\SACTitle{输入}
\begin{itemize}
\item param v:为下面给出的拾取参数中的一个参数赋值
\item param v:param可以取C1|C2|C3|C4|C5|C6|C7|C8|D5|D8|D9|I3|I4|I6,
这些参数的意义在下面会具体说明,v为给每个参数赋值。
\item VALIDATION ON:打开震相检验。
\item VALIDATION OFF:关闭震相检验。
\end{itemize}

\SACTitle{缺省值}
APK C1 0.985 C2 3.0 C3 0.6 C4 0.03 C5 5.0 C6 0.0039 C7 100. C8 -0.1 D5 		2. D8 3. D9 1. I3 3 I4 40 I6 3 VALIDATION ON

\SACTitle{说明}
用于自动震相拾取的算法最初来自于USGS,基于Rex Allen的工作(参见下面的参考文献)。拾取的检测依据是基于信号短期滑动平均值与长期滑动平均值的比值的突然变化。一旦事件被检测到,这次拾取将被赋予一个可选的验证状态以试图从噪声中区分出真	正的事件。一旦检测到的事件的有效性被确认,这次读取到的事件将被计算以决定事件的其他特征。目前只能给出事件持续的时间,如果需要其他比如最大振幅、周期以及衰减率之类的信息也可以加入。注意这里读取的是一次事件而非一个震相!

这个命令的大多数参数永远不需要改变,如果用户想要改进算法这些参数也可以做修改。这些参数中的大多数和参考文献中有相同的含义:

C1是用于滤去直流偏差的递归高通滤波器中的常数

C2是用于改变特征函数的振幅以及一阶微分的权重的常数

C3是用于计算特征函数的短期平均值的时间常数

C4是用于计算特征函数的长期平均值的时间常数

C5是用于计算参考水平阀值的常数。当信号的短期平均值大于C5乘以长期平均值,那么这样一个信号就是一个有效事件

C6是用于计算滤波后数据的滑动平均绝对值的时间常数

C7 当特征函数的绝对值大于C7则认为此台站无效

C8是用于确定信号终止的参数。当信号的绝对值低于C8的时间超过D8秒则认为信号已经终止。目前有两种不同的算法所以C8有两种不同的解释。如果C8为正,那么终止水平为C8乘以事件到达之前的信号的滑动平均绝对值。这个方法对于对于背景噪声很大的台站是很有用的。如果C8为负,则终止水平为C8的绝对值。如果噪声水平比终止水平低的多,则这种方法将给出不同台站的较为一致的终止判据。

D5是一个事件被判定为有效所要达到的最小持续时间的秒数。

D9是用于初始化特征函数长期平均值的持续时间值的,单位为秒

I3、I4和I6是用于震相验证的整型常数,需要保证不被改变

这个命令用于在一个连续信号中检测是否存在地震事件而非震相拾取,主要用于数据最初的事件拾取,对于该命令我还有很多不理解的地方。

\SACTitle{头段变量改变}
事件读取到的时间储存在A(即初动到时)中,运动的质量和方向储存在KA中,事件结束时间储存在F中

\SACTitle{例子}
\begin{SACCode}
SAC> fg seis                //利用这个数据做个例子
SAC> lh a                   //这个数据本身是标有A的,即初动到时
  FILE: SEISMOGR - 1
 --------------
     a = 1.046400e+01
SAC> apk v on               //apk,且打开震相验证
 CDV IPD0 81 329103824.49   //台站名,KA的值,后面两个不知道是什么
SAC> lh a ka f
  FILE: SEISMOGR - 1
 --------------
      a = 1.049000e+01      //新标记的初动到时,可以p看一下效果
     ka = IPD0              //初动信息,I表示急始,P表示初动P波,
                            //D表示初动向下,0不清楚
SAC> apk v off              //apk,关闭震相验证
 CDV -123 81 329103824.49                                              53897
SAC> lh a ka f		
  FILE: SEISMOGR - 1
 --------------
     a = 1.049000e+01       //初始震相
     f = 5.389773e+06       //事件结束,这里的f好像有问题?
\end{SACCode}

\SACTitle{错误消息}
\begin{itemize}
\item[-]1301: 未读入文件
\item[-]1306: 对非等间隔文件的非法操作
\item[-]1307: 对谱文件非法操作
\end{itemize}

\SACTitle{警告消息}
\begin{itemize}
\item[-]1910: 下面的文件没有找到有效的事件:
\end{itemize}

\SACTitle{相关命令}
OHPF,OAPF 

\SACTitle{参考文献}
Rex V. Allen, Automatic Earthquake Recognition and Timing from Single Traces, BSSA, Vol.68, No. 5, Oct. 1978.

\SACTitle{最近修订}
May 15, 1987 (Version 10.2)
