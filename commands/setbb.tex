\section{setbb}
\label{cmd:setbb}

\SACTitle{概要}
设置暂存块变量值

\SACTitle{语法}
SETBB variable  [ APPEND ] value [ variable [ APPEND ] value ... ]

\SACTitle{输入}
\begin{itemize}
\item variable : 暂存块变量名,可以是一个新变量或一个已经有值的变量,变量名最多32字符长
\item value : 暂存块变量的新值,如果也空格必须用引号括起来 
\item APPEND : 将值加到变量的旧值之后,如果该选项忽略,则新值将代替旧值 
\end{itemize}

\SACTitle{说明}
暂存块是用于临时储存信息的地方,这些信息稍后可以通过GETBB命令获取,或通过在变量名前加``\%''直接在命令中使用变量值。如果你想在暂存块变量的末尾连接一些其他文本字符串,你需要在变量名的末尾再放置一个百分号。你可以使用EVALUATE对暂存块变量做基本算术操作,并将结果保存在新的暂存块变量中,你也可以通过UNSETBB命令删除一个暂存块变量。

\SACTitle{例子}
同时设置多个暂存块变量:
\begin{SACCode}
SAC>  SETBB C1 2.45 C2 4.94
\end{SACCode}

稍后在命令中使用这些变量:
\begin{SACCode}
SAC>  BANDPASS CORNERS %C1 %C2
\end{SACCode}

设置包含空格的暂存块变量:
\begin{SACCode}
SAC>  SETBB MYTITLE 'Sample filter response'
\end{SACCode}
检查以确保值正确:
\begin{SACCode}
SAC>  GETBB MYTITLE
 MYTITLE = Sample filter response
\end{SACCode}

在title命令中使用这个变量,其必须用引号括起来,且在变量名前后都要有百分号:
\begin{SACCode}
SAC>  TITLE '%MYTITLE%'
\end{SACCode}

\SACTitle{相关命令}
GETBB, EVALUATE, UNSETBB

\SACTitle{最近修订}
May 15, 1987 (Version 10.2)
