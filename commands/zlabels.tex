\SACCMD{zlabels}
\label{cmd:zlabels}

\SACTitle{概要}
根据等值线的值控制等值线的标记

\SACTitle{语法}
\begin{SACSTX}
ZLABELS [ON|OFF] [SPACING v1 [v2 [v3]]] [SIZE v] [ANGLE v] [LIST c1 c2 ... cn]
\end{SACSTX}
注意:LIST选项只能放在这个命令的最后

\SACTitle{输入}
\begin{itemize}
\item ON|OFF : 打开/关闭等值线标签选项开关 
\item SPACING v1 v2 v3 : 设置相邻标签名的最小、适中和最大间隔(视口坐标系)分别为v1、v2和v3。如果第二、三个值省略则使用前面一个值 
\item SIZE v : 设置标签的尺寸(高度)为v 
\item ANGLE v : 设置标签文本最大角度为v(自水平方向起算的角度,单位为度) 
\item LIST c1 c2 . cn : 设置使用的等值线标签的列表。在这个表上的每个输入用于相应的等值线,如果等值线数目大于这个表的长度,则重复使用整个等值线表 
\item cn :  ON|OFF|INT|FLOATn|EXPn|text 
\item ON : 在相应的等值线上放置标签,使用FORTRAN自由格式,用等值线值形成标签名 
\item OFF : 在相应的等值线上不放置标签名 
\item INT : 在相应的等值线上放置整数标签名 
\item FLOATn : 在相应的等值线上放置小数点后面n位的浮点数作为标签名。如果n被忽略则使用先前值 
\item EXPn : 在相应的等值线上放置小数点后面n位数的指数幂形式标签名,如果n忽略则使用先前值 
\item text : 使用文本标注相应的等值线 
\end{itemize}

\SACTitle{缺省值}
\begin{SACDFT}
ZLABELS  OFF  SPACING 0.1 0.2 0.3  SIZE  0.0075 ANGLE 45.0  LIST ON
\end{SACDFT}

\SACTitle{相关命令}
\nameref{cmd:contour}
