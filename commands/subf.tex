\SACCMD{subf}
\label{cmd:subf}

\SACTitle{概要}
将一组文件中的数据分别减去内存中不同文件的数据

\SACTitle{语法}
\begin{SACSTX}
SUBF [NEWHDR ON|OFF] filelist
\end{SACSTX}

\SACTitle{输入}
\begin{itemize}
\item NEWHDR ON|OFF : 操作得到的结果文件默认从内存中的原始文件获取头段值,如果NEWHDR ON,则结果文件的头段将从filelist中的新文件获取 
\item filelist : SAC二进制文件列表。这个列表可以包含简单的文件名,绝对或相对路径以及通配符。详细信息参见READ命令。 
\end{itemize}

\SACTitle{说明}
这个命令用于将一个文件减去一群文件中或者将一群文件减去另一群文件中。下面针对每种情况各给出一个例子。为了使得文件与文件之间能够相减,必须保证这些文件是等间隔的而且需要有相同的采样间隔和数据点数。后面两个限制可以通过使用BINOPERR命令得到削弱。如果内存中的文件数多于filelsit中文件数,则filelist	的最后一个文件将加到余下的全部文件中。如果NEWHDR OFF(默认情况)内存中的头段大部分不会被改变。如果NEWHDR ON,内存中的头段将被filelist文件的头段取代。

\SACTitle{例子}
将一个文件减去其他三个文件:
\begin{SACCode}
SAC> read file1 file2 file3
SAC> addf file4
\end{SACCode}

将两个文件减去另外两个文件:
\begin{SACCode}
SAC> read file1 file2
SAC> addf file3 file4
\end{SACCode}

\SACTitle{头段变量改变}
depmin, depmax, depmen

\SACTitle{错误消息}
\begin{itemize}
\item[-]1301: 未读入文件
\item[-]1803: 未读入二进制文件
\item[-]1306: 对不等间隔文件的非法操作
\item[-]1307: 对谱文件的非法操作
\item[-]1317: 文件不是SAC数据文件
\item[-]1801: 头段值不匹配(采样间隔或数据点数不相等)
\end{itemize}

\SACTitle{警告消息}
\begin{itemize}
\item[-]1802: 时间重叠  *什么意思?如何犯错?*
\end{itemize}

\SACTitle{相关命令}
\nameref{cmd:read}、\nameref{cmd:binoperr}
