\section{pickphase}
\label{cmd:pickphase}

\SACTitle{概要}
告诉SAC从一个用户定义的参考文件中读入震相列表(也可能是作者信息),或在该命令行上输入震相列表

\SACTitle{语法}
PICKPHASE header phase author [ header phase author ... ]

PICKPHASE FILE {filename}

PICKPHASE AUTHOR {filename}

\SACTitle{输入}
\begin{itemize}
\item header : 头段变量名:t0 - t9. 
\item phase : 对于给出的头段变量对应的拾取的震相名 
\item author : 告诉SAC作者列表,为某个头段所对应的作者,或者是"-" 
\item FILE : 如果使用了FILE选项,SAC将从参考文件中读取震相列表。如果参考文件的文件名在命令行上给出,则SAC将读取这个指定的文件,否则SAC将根据上一次执行PICKPHASE读取最近输入的文件名.如果未给出文件名,则SAC使用sac/aux/csspickprefs 
\item AUTHOR : 这个选项和FILE选项相似,其另一个功能是允许SAC读取指定的头段变量信息
\end{itemize}

\SACTitle{缺省值}
PICKPHASE FILE

\SACTitle{说明}
PICKPHASE用于在命令行上覆盖参考文件。其可以用于在命令行提供指定的头段/震相/作者信息,或者将SAC从一个参考文件重定向到另一个。更多关于参考文件的信息,参见PICKPREFS以及READCSS

注意:如果当数据在数据缓冲区内而用户修改了参考文件,那么在SAC缓冲区中的震相拾取将可能被修改。(缓冲区的信息可以通过LISTHDR和CHNHDR查看)。

例:如果当一些含有某些pP或PKiKP震相的SAC文件通过READ命令读入时,被允许的震相包括pP以及PKiKP,那么这些拾取将出现在Tn时间标记中。如果PICKPHASE在稍后再次使用将pP以及PKiKP从允许的震相中去除,那么pP以及PKiKP到时将不会从CSS文件中读取,已经在内存中的数据的pP和PKiKP拾取将从Tn时间标记中去除。

\SACTitle{相关命令}
PICKPREFS , READCSS, PICKAUTHOR
