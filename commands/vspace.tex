\SACCMD{vspace}
\label{cmd:vspace}

\SACTitle{概要}
改变图形的最大尺寸和形状

\SACTitle{语法}
VSPACE FULL | v

\SACTitle{输入}
\begin{itemize}
\item FULL : 使用整个视窗,这是可能的最大屏幕或窗口尺寸 
\item v : 使视窗比y:x为v,具有这个纵横比的最大的区域称为视窗 
\end{itemize}

\SACTitle{缺省值}
VSPACE FULL

\SACTitle{说明}
视窗代表了屏幕上可以用于绘图的部分。视窗形状和尺寸在不同图形设备之间有很大的变化。
\begin{enumerate}
\item 尽管在尺寸上有很大不同,许多图形终端都具有0.75的纵横比。
\item SGF文件的纵横比为0.75,其大约是标准的8.5*11英寸纸张的纵横比。
\item  由XWINDOWS或SUNWINDOWS图形设备建立的窗口可以有你想要的任意纵横比
\end{enumerate}

这个命令可以控制纵横比,从而使你能够控制图形的形状缺省绘图是在整个视窗上,如果确定了一个纵横比,则视窗就是设备上具有这个纵横比的最大区域。

当你使用PLOTC命令在交互设备上建立一张图,并且最终要将它发送到SGF设备上,这个命令特别有用,在绘制任何图形之前,必须设置纵横比为0.75。这将保证图形在SGF文件上与在交互设备上相同。如果你要建立一个独立于图形设备的正方形视窗,则可以简单地设置纵横比为1.0

\SACTitle{最近修订}
May 15, 1987 (Version 10.2)

