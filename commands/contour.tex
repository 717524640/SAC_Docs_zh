\section{contour}
\label{cmd:contour}

\SACTitle{概要}
利用内存中的数据绘制等值线图

\SACTitle{语法}
CONTour [Aspect ON|OFF]

\SACTitle{输入}
\begin{itemize}
\item Aspect ON: 打开视图比开关。当这个开关打开时,等值线图的视口将会调整保持数据中y与x的比值
\item Aspect OFF: 关闭视图比开关,这时将使用整个视口。
\end{itemize}

\SACTitle{缺省值}
CONTOUR  ASPECT  OFF

\SACTitle{说明}
这个命令用于绘制二维数组数据的等值线图,包括SPECTROGRAM命令的输出。这个文件
操作的SAC文件必须``XYZ''类型的(SAC头段中IFTYPE为``IXYZ'')。有些命令可以
控制数据显示的方式:ZLEVELS控制等值线的数目以及间隔, ZLINES控制线型,
ZLABELS控制等值线标签,ZTICKS控制方向标记,ZCOLORS控制线条颜色。根据
contour选项的不同,有两种不同的绘制等值线算法。一种快速扫描方法用于既不
选择实线型也没有时标和标识的情况。另一种慢一点的方法,在绘图之前要组合
全部的线段。你可以使用快速扫描方法粗看你的数据,然后选择其他选项绘制最终
图形。

\SACTitle{例子}
下面的第一个例子中,读取一个文件并用缺省值绘图。你可以在sac/macros/data下找到名为contourdata的文件作为下面的例子文件。

在这个例子中,文件被读入,头段列出以决定z数据的范围(DEPMIN和DEPMAX.)这里只显示了lh命令列出信息的部分。选择等值线范围为700km到1150km,增量为25km。选择包括四种线型的线型表,其中第一个为实线。这个列表将每四条等值线重复一次。然后给等值线图起了个名字,最后绘制出来:
\begin{SACCode}
SAC> r ./contourdata 
SAC> lh
  
  FILE: ./contourdata - 1
 -------------------
         NPTS = 10000
            B = 0.000000e+00
            E = 9.999000e+03
       IFTYPE = GENERAL XYZ (3-D) FILE
        LEVEN = TRUE
        DELTA = 1.000000e+00
       DEPMIN = 6.977119e+02
       DEPMAX = 1.154419e+03
       DEPMEN = 8.316542e+02
       LOVROK = TRUE
       NXSIZE = 100
     XMINIMUM = 8.257441e+04
     XMAXIMUM = 8.699228e+04
       NYSIZE = 100
     YMINIMUM = 4.743853e+05
     YMAXIMUM = 4.772037e+05
        NVHDR = 6
        NORID = 0
        NEVID = 0
       LPSPOL = FALSE
       LCALDA = TRUE
SAC> zlevels range 700 1150 increment 25
SAC> zlines list 1 2 3 4
SAC> title 'Katmai topography from survey data [inc = 25 km]'
SAC> contour
\end{SACCode}

\begin{figure}[h]
\centering
\includegraphics[width=14cm]{contour_1}
\caption{CONTOUR例子I示意图}
\end{figure}

下面的例子中,使用同样的文件,但是显示选项不同。每四条等值线有一个整数标签。	每条等值线之间都有一个指向向下的箭头。所有等值线为实线型:
\begin{SACCode}
SAC> r ./contourdata 
SAC> zlevels range 700 1150 increment 25
SAC> zlabels on list int off off off
SAC> zticks on direction down
SAC> zlines list 1
SAC> title 'Katmai topography from survey data [labels and ticks]'
SAC> contour
\end{SACCode}

\begin{figure}[h]
\centering
\includegraphics[width=14cm]{contour_2}
\caption{CONTOUR例子II示意图}
\end{figure}

\SACTitle{头段变量改变}
要求: IFTYPE (为``IXYZ''), NXSIZE, NYSIZE

使用: XMINIMUM, XMAXIMUM, YMINIMUM, YMAXIMUM

\SACTitle{相关命令}
ZCOLORS, ZLABELS, ZLEVELS, ZLINES, ZTICKS, SPECTROGRAM

\SACTitle{致谢}
快速扫描等值线子程序由Dave Harris(DBH)开发。

\SACTitle{最近修订}
JULY 22, 1991 (Version 10.6d)
