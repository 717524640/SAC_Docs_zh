\section{ohpf}
\label{cmd:ohpf}

\SACTitle{概要}
打开一个HYPO格式的震相文件

\SACTitle{语法}
OHPF {file}

\SACTitle{输入}
\begin{itemize}
\item file  要打开的文件名。如果文件已经存在,则打开并将新的震相加到文件底部
\end{itemize}

\SACTitle{缺省值}
OHPF HPF

\SACTitle{说明}
SAC产生的HYPO震相拾取文件可以用于程序HYPO71以及其他类似事件定位程序的输入。由APK和PPK得到的震相拾取信息被写入这个打开的文件中,这个文件可以使用CHPF关闭。打开一个新的HYPO文件会自动关闭前一个已经打开的文件。打开一个已经存在的HYPO文件的同时也会自动删除文件的最后一行,这一行原本有一个指令标记10作为HYPO文件的结束标志符,删除最后一行意味着可以在其后添加新的拾取。终止SAC也会自动关闭任何已经打开的拾取文件,事件分割符能够用WHPF命令写入震相拾取文件。

\SACTitle{错误消息}
\begin{itemize}
\item[-]1901: 无法打开HYPO震相拾取文件
	\begin{itemize}
	\item[-]可能是文件名只能中的非法字符
	\item[-]偶尔的系统错误
	\end{itemize}
\end{itemize}

\SACTitle{相关命令}
APK , PLOTPK , WHPF , CHPF

\SACTitle{参考文献}
W.H.K. Lee and J.C. Lahr; HYPO71 (Revised): A Computer Program for Determining Hypocenter, Magnitude, and First Motion Pattern of Local Earthquakes; U. S. Geological Survey report 75-311.

\SACTitle{最近修订}
March 20, 1992 (Version 10.6e)
