\SACCMD{add}
\label{cmd:add}

\SACTitle{概要}
给每一个数据点加上一个常数

\SACTitle{语法}
%ADD \{v1 \{v2 ... vn\} \}
ADD [v1 [v2 ... vn]]
\footnote{
在语法中,被中括号包围起来的部分表示是可选项,省略号表示选项可以重复出现,
``|''代表分隔的多个选项最多只能存在一个。以后的命令也用此规则。
}

\SACTitle{输入}
\begin{itemize}
\item v1:  加到第一个文件的常数
\item v2:  加到第二个文件的常数
\item vn:  加到第n个文件的常数
\end{itemize}

\SACTitle{缺省值}
ADD 0.0

\SACTitle{说明}
这个命令给内存中的每个数据文件的每个元素加一个常数。这个常数对于每一个数
据文件来说可以是相同的也可以是不同的。如果内存中数据文件的数目比命令给出的常
数要多,那么命令中的最后一个常数将用于剩下的所有文件。如果内存中数据文件
数目比给出的常数少,则多余的常数将被忽略。ADD的默认值使得若只输入ADD而不加
任何选项时则给内存中所有文件加0,注意这个命令只对内存中的文件进行操作,SAC的
几乎所有命令都是如此。


\SACTitle{例子}
为了给文件F1的每个元素加5.1,F2和F3的每个元素加6.2:
\begin{SACCode}
SAC> r f1 f2 f3
SAC> add 5.1 6.2
\end{SACCode}

\SACTitle{错误消息}
\begin{itemize}
\item[-]1301: 未读入数据文件
\item[-]1307: 对谱文件的非法操作
\end{itemize}

\SACTitle{头段变量改变}
DEPMIN, DEPMAX, DEPMEN

\SACTitle{最近修订}
January 8, 1983 (Version 8.0)
