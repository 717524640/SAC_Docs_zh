\SACCMD{add}
\label{cmd:add}

\SACTitle{概要}
给每一个数据点加上一个常数

\SACTitle{语法}
\begin{SACSTX}
add [v1 [v2 ... vn]]
\end{SACSTX}

\SACTitle{输入}
\begin{itemize}
\item v1:  加到第一个文件的常数
\item v2:  加到第二个文件的常数
\item vn:  加到第n个文件的常数
\end{itemize}

\SACTitle{缺省值}
\begin{SACDFT}
add 0.0
\end{SACDFT}

\SACTitle{说明}
这个命令用于给数据文件中的每个数据都加上一个常数。对于不同的数据文件,
这个常数可以是相同的也可以是不同的。若内存中数据文件的数目比命令给出的常
数要多,则剩下的所有文件将都加上命令的最后一个常数;若内存中数据文件
数目比给出的常数少,则多余的常数被忽略。add的默认值使得若只输入add而不加
任何选项时对内存中所有文件加上常数0。注意这个命令只对内存中的文件进行操作,
SAC的几乎所有命令都是如此。

\SACTitle{例子}
为了给文件F1的每个数据加5.1,F2和F3的每个数据加6.2:
\begin{SACCode}
SAC> r f1 f2 f3
SAC> add 5.1 6.2
\end{SACCode}

\SACTitle{错误消息}
\begin{itemize}
\item[-]1301: 未读入数据文件
\item[-]1307: 对谱文件的非法操作
\end{itemize}

\SACTitle{头段变量改变}
depmin, depmax, depmen

\SACTitle{最近修订}
January 8, 1983 (Version 8.0)
