\SACCMD{addf}
\label{cmd:addf}

\SACTitle{概要}
读入一组数据文件,并与内存中的另一种数据文件相加

\SACTitle{语法}
\begin{SACSTX}
ADDF [N!EWHDR! ON|OFF] filelist
\end{SACSTX}

\SACTitle{输入}
\begin{itemize}
\item NEWHDR ON|OFF:默认情况下,文件相加生成的文件会使用内存中的原文件的头段。
    若NEWHDR ON,则生成的文件将使用filelist中的新文件的头段。
\item filelist: SAC二进制文件列表。
\end{itemize}

\SACTitle{说明}
这个命令用于将一组文件同时加上同一个文件,或者将一组文件与另一组文件相加。下面针对
每种情况各给出一个例子。为了使得文件与文件之间能够相加,必须保证这些文件是等间隔的且
有相同的采样周期和数据点数。可以通过使用binoperr命令解除对于必须拥有相同采样周期
和数据点数的限制。

若内存中的文件数多于filelsit中文件数,则filelist的最后一个文件将加到余下的全部文件中。这个
命令可以用于数据的迭加,实际处理数据时这样的情况会经常遇到,对于文件的限制也很容易满足。

\SACTitle{例子}
将一个文件加到其他三个文件中:
\begin{SACCode}
SAC> r FILE1 FILE2 FILE3
SAC> addf FILE4
\end{SACCode}

将两个文件分别加到另外两个文件中:
\begin{SACCode}
SAC> r FILE1 FILE2
SAC> addf FILE3 FILE4
\end{SACCode}

\SACTitle{头段变量改变}
depmin, depmax, depmen

\SACTitle{错误消息}
\begin{itemize}
\item[-]1301: 未读入文件
\item[-]1803: 未读入二进制数据文件
\item[-]1306: 对不等间隔文件的非法操作
\item[-]1307: 对谱文件的非法操作
\item[-]1801: 头段值不匹配(采样周期或数据点数不相等)
\end{itemize}

\SACTitle{警告消息}
\begin{itemize}
\item[-]1802: 时间重叠
\end{itemize}

\SACTitle{相关命令}
\nameref{cmd:read}、\nameref{cmd:binoperr}
