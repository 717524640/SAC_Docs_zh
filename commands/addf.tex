\section{addf}
\label{cmd:addf}

\SACTitle{概要}
将一组数据文件与内存中的一组数据文件相加

\SACTitle{语法}
ADDF [Newhdr ON|OFF] filelist\footnote{还记得符号约定吗?NEWHDR是关键字,
因而用大写,对于简写可以省略的部分用小写表示,filelist算是参数吧,所以都小写}

\SACTitle{输入}
\begin{itemize}
\item NEWHDR ON|OFF:默认情况下,文件加法操作得到的结果文件从内存中的原始文件获取头段值,如果NEWHDR ON,则结果文件的头段将从filelist中的新文件获取。
\item filelist: SAC二进制文件列表。这个列表可以包含简单的文件名,绝对或相对路径以及通配符。详细信息参见READ命令。
\end{itemize}

\SACTitle{说明}
这个命令用于将一堆文件同时加上同一个文件,或者将一群文件与另一群文件相加。下面针对
每种情况各给出一个例子。为了使得文件与文件之间能够相加,必须保证这些文件是等间隔的而且
需要有相同的采样间隔和数据点数。后面两个限制可以通过使用BINOPERR命令得到削弱。如果内存
中的文件数多于filelsit中文件数,则filelist的最后一个文件将加到余下的全部文件中。这个
命令可以用于数据的迭加,实际处理数据时这样的情况会经常遇到,对于文件的限制也很容易满足。

\SACTitle{例子}
将一个文件加到其他三个文件中:
\begin{SACCode}
SAC> r FILE1 FILE2 FILE3
SAC> addf FILE4
\end{SACCode}
将两个文件分别加到另外两个文件中:
\begin{SACCode}
SAC> r FILE1 FILE2
SAC> addf FILE3 FILE4
\end{SACCode}

\SACTitle{头段变量改变}
如果NEWHDR OFF(默认情况)内存中的头段大部分不会被改变。如果NEWHDR ON,内存中的头段将被filelist文件的头段取代。可能会改变的头段变量有:DEPMIN, DEPMAX, DEPMEN

\SACTitle{错误消息}
\begin{itemize}
\item[-]1301: 未读入文件
\item[-]1803: 未读入二进制数据文件
\item[-]1306: 对不等间隔文件的非法操作
\item[-]1307: 对谱文件的非法操作
\item[-]1801: 头段值不匹配
	\begin{itemize}
	\item[-]采样间隔或数据点数不相等
	\item[-]可以通过BINOPERR命令控制是否严格执行限制
	\end{itemize}
\end{itemize}




\SACTitle{警告消息}
\begin{itemize}
\item[-]1802: 时间重叠\footnote{不理解这个警告}
	\begin{itemize}
  	\item[-]文件相加操作依然会执行
	\end{itemize}
\end{itemize}

\SACTitle{相关命令}
READ, BINOPERR

\SACTitle{最近修订}
May 26, 1999 (Version 0.58)
