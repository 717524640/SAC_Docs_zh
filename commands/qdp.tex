\SACCMD{qdp}
\label{cmd:qdp}

\SACTitle{概要}
控制低分辨率快速绘图选项``quick and dirty plot''

\SACTitle{语法}
\begin{SACSTX}
QDP [ON|OFF|n] [TERM ON|OFF|n] [SGF ON|OFF|n]
\end{SACSTX}

\SACTitle{输入}
\begin{itemize}
\item ON|OFF : 打开/关闭终端和SGF文件的QDP选项 
\item n : 打开终端和SGF文件的QDP选项,并改变绘制的数据点数为n 
\item TERM ON|OFF : 打开/关闭终端qdp绘图选项 
\item TERM n : 打开终端的QDP选项,并改变绘制的数据点数为n 
\item SGF ON|OFF : 打开/关闭终端qdp绘图选项 
\item SGF n : 打开终端的QDP选项,并改变绘制的数据点数为n 
\end{itemize}

\SACTitle{缺省值}
\begin{SACDFT}
qdp term 5000 sgf 5000
\end{SACDFT}

\SACTitle{说明}
绘制大型文件会消耗很长的时间,``quick and dirty plot''选项通过不绘制每一个点来加速绘图。
当这个选项为开时,SAC将通过文件中数据点数除以你要求显示的数据点计算得到区间大小。文件越大,每个区间中数据点就越多。

SAC然后后计算出每个区间内最小和最大数据点数。如果这个选项为开则SAC将在绘图的角落的小框中显示采样因子(区间尺寸的一半)。实际显示的数据点可能接近于它的值也可能相差很多,因为这个命令仅仅绘制区域中的极值。

对于目前的机器来说,一般大文件的绘图已经不会耗费太多时间了,所以一般都直接qdp off

\SACTitle{例子}
假设文件FILE1有20000个数据点,文件FILE2有40000个数据点,如果你输入:
\begin{SACCode}
sac> r file1 file2
sac> bd x
sac> p
\end{SACCode}

那么两张图都将准确包含5000个点,第一个文件将每4个点取一个点用于绘图,第二个每8个点取一个点绘图。数据区间的尺寸尽量取下限以保证你至少可以看到你所要求的点数。如果你要将图绘制到SGF:
\begin{SACCode}
SAC> begindevices sgf
SAC> plot
\end{SACCode}
