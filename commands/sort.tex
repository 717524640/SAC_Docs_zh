\section{sort}
\label{cmd:sort}

\SACTitle{概要}
根据头段值对内存中的文件进行排序

\SACTitle{语法}
SORT COMMIT|ROLLBACK|RECALLTRACE header [ ASCEND | DESCEND ] [ header [ ASCEND | DESCEND ]... ]

\SACTitle{输入}
\begin{itemize}
\item COMMIT|ROLLBACK|RECALLTRACE : 参见具体章节
\item header : 依据该头段排序文件 
\item ASCEND : 按头段值升序排列 
\item DESCEND : 按头段值降序排列 
\end{itemize}

\SACTitle{说明}
根据给出的头段值对内存中的文件进行排序。头段变量在命令行中出现的越早,这个变量字段就具有越高的优先权,后面的变量字段用于解决无法根据第一个变量字段排序的情况。最多可以输入5个头段变量。

每个都可以跟着ASCEND或DESCEND来表明那个特定字段的排序方式。如果未指定,则默认为升序排列。如果使用sort命令但未指定任何头段值,它将根据上一次执行sort命令时的头段去排序,如果第一次调用sort但没有给出头段,则会产生错误1379.

\SACTitle{缺省值}
默认所有的字段都是升序排列

\SACTitle{错误消息}
\begin{itemize}
\item[-]301:  内存溢出
\item[-]1379: 未给出SORT参数
\item[-]1380: SORT参数过多
\item[-]1381: 不是一个有效的SORT参数
\item[-]1383: SORT失败
\end{itemize}

\SACTitle{警告消息}
- 1384

\SACTitle{相关命令}
COMMIT, ROLLBACK, RECALLTRACE

\SACTitle{最近修订}
October 27, 1998 (Version 0.58)
