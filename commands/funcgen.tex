\section{funcgen}
\label{cmd:funcgen}

\SACTitle{概要}
生成一个函数并将其存在内存中

\SACTitle{语法}
\begin{SACSyntax}
!f!unc!g!en [!type!] [ !d!elta v ] [ !n!pts n ] [ !be!gin v ]
\end{SACSyntax}
%FuncGen [ type ] [ Delta v ] [ Npts n ] [ BEgin v ]

其中type是下面中的一个:

IMPulse | STep | Boxcar | Triangle | SINE [v1 v2] | Line [v1 v2] |
Quadratic [v1 v2 v3] | CUBIC [v1 v2 v3 v4] | SEISmogram | Random [v1 v2]
| IMPSTRIN  [n1 n2 ... nN]

\SACTitle{输入}
\begin{itemize}
\item IMPULSE : 在中心数据点处的脉冲函数 
\item IMPSTRIN : 一系列在指定数据点处的脉冲函数  
\item STEP :  阶越函数,在数据的前半段为0,后半段为1 
\item BOXCAR : 矩形函数,前后三分之一的数据为0,中间三分之一为1 
\item TRIANGLE : 三角函数,前后四分之一为0,在中间两个四分之一线性增加到1,再线性减小到0 
\item SINE v1 v2 : 正弦函数,v1表示频率,单位为Hz,v2为以度为单位的相位角,振幅为1,注意在相位参数中有一个$2\pi$因子: $ Fsin = 1.0 \sin (2\pi (v_1t+v_2))$ 
\item LINE v1 v2 : 线性函数,斜率为v1,截距为v2 
\item QUADRATIC v1 v2 v3 : 二次函数 $v_1 t^{2} + v_2 t + v_3 $
\item CUBIC v1 v2 v3 v4 : 三次函数 $ v_1 t^{3} + v_2 t^2 + v_3t + v_4 $
\item SEISMOGRAM : 地震样本。对于这个函数DELTA, NPTS和BEGIN选项忽略,这个样本有1000个数据点 
\item RANDOM v1 v2 :  随机序列(高斯白噪声)生成器。v1是要生成的随机序列文件数,v2是用于产生第一个随机数的``种子'',这个种子值保存在USER0中,因而如果需要你可以稍后生成一个完全相同的随机序列。 
\item DELTA v : 设置采样间隔为v,储存在头段DELTA中 
\item NPTS n : 设置函数的数据点数为n,储存在头段NPTS中 
\item BEGIN v : 设置起始时间为v,储存在头段B中 
\end{itemize}

\SACTitle{缺省值}
FUNCGEN IMPULSE NPTS 100 DELTA 1.0 BEGIN 0.

对于正弦函数频率和相位缺省值分别为0.05和0.

一次、二次、三次函数的系数都是1。

随机序列数为1,种子是12357。

\SACTitle{说明}
执行这个命令相当于读取了单个文件(除了RANDOM选项会产生多个文件之外)到内存中,产生的文件名即为函数名,内存中先前的数据都会丢失。

\SACTitle{头段变量改变}
在函数生成时内存中的头段变量也会相应生成。

\SACTitle{相关命令}
DATAGEN

\SACTitle{最近修订}
October 11, 1984 (Version 9.1)
