\SACCMD{interpolate}
\label{cmd:interpolate}

\SACTitle{概要}
对等间隔或不等间隔数据进行插值以得到新采样率

\SACTitle{语法}
\begin{SACSTX}
INTERP!OLCATE! [D!ELTA! v|N!PTS! v] [B!EGIN! v]
\end{SACSTX}

\SACTitle{输入}
\begin{description}
\item [DELTA v] 设置新采样率为v。数据的时间跨度(E-B)保持不变,npts变化,E由于
    需要与b的间距为delta的整数倍,所以可能会有微调。
\item [NPTS n] 强制设置插值后文件的数据点数为n。时间宽度不变,delta发生变化。
\item [BEGIN v] 在v处开始插值,该值将作为插值文件的起始时间。BEGIN可以和DELTA或NPTS选项
    一起使用。
\end{description}

\SACTitle{说明}
该命令使用Wiggins的weighted average-sloped插值方法将不等间隔数据转换为等间隔数据,
以及对等间隔数据插值得到新的采样率。不像三次样条插值,在输入样本数据点间不会存在
极值。如果要降低采样率,即减采样,由于该命令没有抗混叠滤波器,所以最好使用decimate命令。

DELTA选项和NPTS选项只能同时使用一个,若二者同时使用,则命令中的后者起作用。

BEGIN选项用于控制输入数据的插值起点,也可以通过cut命令设置b和e再进行插值操作。

\SACTitle{示例}
假定filea是等间隔数据,采样率为0.025,为了将转换到采样率为0.02秒:
\begin{SACCode}
SAC> r filea
SAC> interp delta 0.02
\end{SACCode}
由于新delta小于原数据delta,可能会出现混叠现象,所以会输出警告信息。

假定fileb数据点数npts=3101,想要保持其时间跨度,并采样至npts=4096个点:
\begin{SACCode}
SAC> r fileb
SAC> interp npts 4096
\end{SACCode}

假设filec是不等间隔数据,为了将其转换为采样率为0.01秒的等间隔数据:
\begin{SACCode}
SAC> read filec
SAC> interpolate delta 0.01
\end{SACCode}

\SACTitle{警告消息}
\begin{itemize}
\item[-]2008: 要求的开始时间小于文件起始时间(输出会被截断)
\item[-]2015: 要求的开始时间大于文件结束时间(不执行操作)
\end{itemize}

\SACTitle{头段变量}
delta、npts、e、b、leven
