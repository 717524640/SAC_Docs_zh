\SACCMD{marktimes}
\label{cmd:marktimes}

\SACTitle{概要}
根据一个速度集得到走时并对数据文件进行标记

\SACTitle{语法}
\begin{SACSTX}
MARKT!IMES! [T!O! marker] [D!ISTANCE! H!EADER!|v] [O!RIGIN! H!EADER!|v|GMT time] 
    [V!ELOCITIES! v ... ]
\end{SACSTX}

\SACTitle{输入}
\begin{itemize}
\item TO marker : 定义头段中用于储存结果的第一个时间标记。对下一个的速度使用下一个时间 
\item marker :  T0|T1|T2|T3|T4|T5|T6|T7|T8|T9 
\item DISTANCE HEADER : 使用头段中的DIST代表距离用于走时计算 
\item DISTANCE v : 使用v作为走时计算中的距离 
\item ORIGIN HEADER : 使用头段中的参考时间(O)用于走时计算 
\item ORIGIN v : 使用相对参考时间偏移v秒用于走时计算 
\item ORIGIN GMT time : 使用Greenwich平均时间作为参考时间 
\item time :  GMT时间包含六个整数:年、儒略日、时、分、秒、毫秒 
\item VELOCITIES v ... : 设置用于走时计算的速度集,最多可以输入10个速度 
\end{itemize}

\SACTitle{缺省值}
\begin{SACDFT}
marktimes velocities 2. 3. 4. 5. 6.  distance header origin header to t0
\end{SACDFT}

\SACTitle{说明}
这个命令在头段中标记震相的到时,给定事件的发生时间,震中距以及速度即可计算到	时。下面的方程用于简单估计走时:
 		\[ time(j) = origin + \frac{distance}{velocity(j)} \]
结果被写入指定的时间标记中

\SACTitle{例子}
使用默认的速度集,强制距离为340km,第一个时间标记为T4:
\begin{SACCode}
SAC> marktimes distance 340. to t4
\end{SACCode}

选择一个不同的速度集:
\begin{SACCode}
SAC> markt v 3.5 4.0 4.5 5.0 5.5
\end{SACCode}

设置新的参考时间并将结果保存在T2中:
\begin{SACCode}
SAC> markt origin gmt 1984 231 12 43 17 237 to t2
\end{SACCode}

\SACTitle{头段变量改变}
Tn, KTn
