\SACCMD{wiener}
\label{cmd:wiener}

\SACTitle{概要}
设计并应用一个自适应Wiener滤波器

\SACTitle{语法}
\begin{SACSTX}
WIENER [WINDOW pdw] [NCOEFF n] [MU OFF|ON|v] [EPS ILON OFF|ON|e]
\end{SACSTX}

\SACTitle{输入}
\begin{itemize}
\item WINDOW pdw : 设置滤波器设计窗口为pdw。关于pdw参见CUT命令 
\item NCOEFF n : 设置滤波器系数为n个 
\item MU off | on | v : 设置自适应步长参数 
\item Off : 将mu设置为0 
\item On : 将mu设置为mu = 1.95 / Rho(0)。其中Rho(0)是pdw中延迟为0时的自相关系数。 
\item v : 设置 mu = v. 
\item EPSILON e :  Set ridge regression parameter to epsilon. 
\end{itemize}

\SACTitle{缺省值}
\begin{SACDFT}
wiener window b 0 10 ncoeff 30 mu off epsilon off
\end{SACDFT}

\SACTitle{说明}
误差预测滤波器使用Yule-Walker方法,从指定的部分数据窗中由自相关函数给出。这个窗口可以是文件的任何部分,然后滤波器应用到这个个信号,即信号由残差序列置换。这个滤波器可以用作预白化或用作瞬时信号的检波预处理器。对MU指定非0值,滤波器可作成时域自适应的,大MU可能会导致不稳定。

\SACTitle{例子}
下面的命令将应用一个非自适应滤波器,将第一个十秒指定为窗:
\begin{SACCode}
SAC> wiener window b 0 10 mu 0.
\end{SACCode}

下面命令将应用带40个系数的滤波器,指定设计窗为从文件开始到第一个到时前1秒:
\begin{SACCode}
SAC> wiener ncoeff 40 window b a -1
\end{SACCode}

\SACTitle{头段变量改变}
depmin, depmax, depmen

\SACTitle{错误消息}
\begin{itemize}
\item[-]1301: 未读入数据文件
\item[-]1306: 对不等间隔文件的非法操作
\item[-]1307: 对谱文件的非法操作
\item[-]1608: 不良的Wiener滤波器噪声窗口(滤波器设计窗口不在文件窗口内,或用于窗口的
    头段变量值未定义)
\end{itemize}

\SACTitle{警告消息}
\begin{itemize}
\item[-]1609: Wiener滤波器数字不稳定
\end{itemize}

\SACTitle{相关命令}
\nameref{cmd:cut}
