\SACCMD{getbb}
\label{cmd:getbb}

\SACTitle{概要}
获取或打印黑板变量的值

\SACTitle{语法}
\begin{SACSTX}
GETBB [TO TERM!inal!|filename] [NAMES ON|OFF] [NEWLINE ON|OFF] 
    ALL|variable [variable ...]
\end{SACSTX}

\SACTitle{输入}
\begin{itemize}
\item TO TERMINAL : 打印值到终端
\item TO filename : 将值附加到文件后
\item NAMES [ON] : 输出格式为头段变量名后加一个等于号以及其值
\item NAMES OFF :  只打印头段变量的值 
\item NEWLINE [ON] : 在每一个头段变量后换行 
\item NEWLINE OFF : 咋每个值后不换行 
\item ALL :  打印当前定义的全部头段变量 
\item variable : 打印列表指定的头段变量值 
\end{itemize}

\SACTitle{缺省值}
\begin{SACDFT}
getbb to terminal names on newline on all
\end{SACDFT}

\SACTitle{说明}
这个命令允许你打印指定的黑板变量。变量可以由SETBB定义,也可以由EVALUATE命令对黑板变量
进行基本算术操作并将结果储存在新黑板变量中。黑板变量也可以在SAC命令中直接引用。

命令控制打印哪些值以及打印的格式,你可以将其打印到终端或者文件中。你可以使用这些选项对一系列数据文件进行测量,将结果保存到文本文件中,然后用READALPHA命令将这个文件读回SAC,绘图或者进行更多的分析。

\SACTitle{例子}
假设你已经设置了一些头段变量:
\begin{SACCode}
SAC> setbb c1 2.45 c2 4.94
\end{SACCode}

稍后可以这样打印他们的值:
\begin{SACCode}
SAC> getbb c1 c2
 c1 = 2.45
 c2 = 4.94
\end{SACCode}

想要在一行内只打印其值:
\begin{SACCode}
SAC> getbb names off newline off c1 c2
 2.45 4.94
\end{SACCode}

假设你有一个宏文件叫GETXY,其可以对单个文件进行某些分析操作,并将结果储存在两个头段变量中X和Y中。你想要对当前目录中所有垂直分量进行操作,保存每对X和Y的值,然后绘图。下面的宏文件的第一个参数是用于储存这些结果的文本文件:
\begin{SACCode}
DO FILE WILD *Z
  READ FILE
  MACRO GETXY
  GETBB TO 1 NAMES OFF NEWLINE OFF X Y
ENDDO
GETBB TO TERMINAL
READALPHA CONTENT P 1
PLOT
\end{SACCode}
最终这个文本文件将包含成对的x-y数据点,每行一个,对应一个垂直分量的数据文件。为了关闭文本文件并清空缓存区,最后将输出重定向到终端的GETBB命令是必要的。

\SACTitle{相关命令}
\nameref{cmd:setbb}、\nameref{cmd:evaluate}
