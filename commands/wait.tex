\SACCMD{wait}
\label{cmd:wait}

\SACTitle{概要}
通知SAC在绘制不同图形的操作之间是否暂停

\SACTitle{语法}
WAIT [ ON | OFF | EVERY ]

\SACTitle{输入}
\begin{itemize}
\item ON : 按常规模式打开等待选项 
\item OFF : 关闭等待选项 
\item EVERY : 按每个图形模式打开等待选项
\end{itemize}

\SACTitle{缺省值}
WAIT ON

\SACTitle{说明}
当你读取了多个数据文件并使用PLOT绘制,每个文件将产生一个框架,如果你绘制到终端,正常情况下SAC将在每张图后暂停并发送信息``WAITING''到终端。然后你可以键入回车看到下一张图,或输入``GO''使SAC不暂停地绘制剩下的图形,或键入``KILL''终止绘制这组文件。SAC绘制最后一张图之后不再暂停,因为通常的输入提示符提供了相同的功能。当这个选项关闭时,SAC在不同的绘图之间不暂停。在每个图形模式下,SAC不仅在用PLOT命令产生的绘图之间,而且在绘制每个图形时都暂停。如果你是在命令文件或者作业控制程序的控制下使用SAC,那么这一点将很有用。

\SACTitle{例子}
下面的例子展示了SAC的正常暂停模式如何起作用的:
\begin{SACCode}
SAC> READ FILE1 FILE2 FILE3 FILE4
SAC> PLOT
  waiting  // plot of FILE1 to terminal
SAC> (return)
SAC> waiting  //plot of FILE2
SAC> kill  //user has seen enough
  (prompt) //SAC now waiting for next command
\end{SACCode}

\SACTitle{最近修订}
October 11, 1984 (Version 9.1)

