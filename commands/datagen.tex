\section{datagen}
\label{cmd:datagen}

\SACTitle{概要}
产生样例波形文件并储存在内存中。

\SACTitle{语法}
DATAGEN [MORE] [COMMIT|ROLLBACK|RECALLTRACE] [SUB name] [filelist]

其中name可以是:LOCAL | REGIONAL | TELESEISMIC

\SACTitle{输入}
\begin{itemize}
\item MORE : 将新产生的样本数据放在内存中旧文件后,如果此项省略,新数据将替代旧数据。
\item COMMIT、ROLLBACK、RECALLTRACE : 不想解释
\item SUB name : 选择要读数据的子目录名,其中子目录名可以使local、regional、teleseismic,后接文件列表。
\item name : LOCAL|REGIONAL|TELESEIS
\item filelist : 样本数据文件列表,文件名在下面给出。可以使用简单文件名或通配符,参见READ和WILD。
\end{itemize}

\SACTitle{缺省值}
DATAGEN COMMIT SUB LOCAL cdv.z

\SACTitle{说明}
该命令从一系列样本地震图中读取一个或多个到内存。事实上,该命令与READ相同,只是此处从特殊的数据目录读取特殊文件。
你可以使用通配符来读取多个文件。

\SACTitle{错误信息}
\begin{itemize}
\item[-]1301: 未读入数据
	\begin{itemize}
  	\item[-]未给出要读的文件列表
	\item[-] 列表给出的文件不可读
	\end{itemize}
\item[-]1314: 数据文件列表不得以数字开头
\item[-]1315: 超过最大文件数
\end{itemize}

\SACTitle{警告信息}
\begin{itemize}
\item[-]0101: 打开文件
\item[-]0108: 文件不存在
\item[-]0114: 读文件
	\begin{itemize}
	\item[-]一般当SAC遇到这些错误时可以跳过该文件然后读取余下文件。详情参见READERR
	\end{itemize}
\end{itemize}
  
\SACTitle{LOCAL事件}
近震发生在Livermore Valley of California。震级ML=1.6,其被Livermore Local Seismic Network (LLSN)所记录,
LLSN拥有一系列垂直和三分量台站。这个数据集中包含了9个三分量台站的数据。数据时长40s,每秒100个采样点。
台站信息、事件信息、p波及尾波到时都包含在头段中,这些文件包括:
\begin{SACCode}
    cal.z, cal.n, cal.e
    cao.z, cao.n, cao.e
    cda.z, cda.n, cda.e
    cdv.z, cdv.n, cdv.e
    cmn.z, cmn.n, cmn.e
    cps.z, cps.n, cps.e
    cva.z, cva.n, cva.e
    cvl.z, cvl.n, cvl.e
    cvy.z, cvy.n, cvy.e
\end{SACCode}

\SACTitle{REGIONAL事件}
区域地震发生在Nevada,被Digital Seismic Network (DSS)所记录。 DSS包含了美国西部的四个宽频带三分量台站。这些台站包括:
\begin{SACCode}
   		elk: Elko, NV
   		lac: Landers, CA
   		knb: Kanab, UT
   		mnv: Mina, NV
\end{SACCode}
采样率为每秒40个采样点,数据时长为300s,从事件发震前5s开始,文件名为:
\begin{SACCode}
    elk.z, elk.n, elk.e
    lac.z, lac.n, lac.e
    knb.z, knb.n, knb.e
    mnv.z, mnv.n, mnv.e
\end{SACCode}

\SACTitle{TELESEISMIC事件}
远震事件于September 10, 1984发生在加州北海岸靠近Eureka的地方。其为中等偏大的地震(ML 6.6, MB 6.1, MS 6.7) ,
多地有感。Regional Seismic Test Network(RSTN)包含了美国和加拿大的5个台站,这些台站分别为:
\begin{SACCode} 
	cpk: Tennessee
    ntk: Northwest Territories, Canada
    nyk: New York
    onk: Ontario, Canada
    sdk: South Dakota
\end{SACCode}
该数据集中包含了中等周期和长周期数据,cpk数据不可得,sdk的长周期数据被截断。这个数据集的数据时长1600s,
长周期每秒1个采样点,中等周期每秒4个采样点。文件包括:
\begin{SACCode} 
    ntkl.z, ntkl.n, ntkl.e, ntkm.z, ntkm.n, ntkm.e
    nykl.z, nykl.n, nykl.e, nykm.z, nykm.n, nykm.e
    onkl.z, onkl.n, onkl.e, onkm.z, onkm.n, onkm.e
    sdkl.z, sdkl.n, sdkl.e, sdkm.z, sdkm.n, sdkm.e
\end{SACCode}

\SACTitle{最近修订}
August 2011 (version 101.5)
