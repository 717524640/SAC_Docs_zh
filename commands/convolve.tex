\SACCMD{convolve}
\label{cmd:convolve}

\SACTitle{概要}
计算主信号与内存中所有信号的卷积

\SACTitle{语法}
\begin{SACSTX}
CONVO!LVE! [M!ASTER! name|n] [N!UMBER! n] [L!ENGTH! ON|OFF|v] 
    [T!YPE! R!ECTANGLE!|HAM!MING!|HAN!NING!|C!OSINE!|T!RIANGLE!]
\end{SACSTX}

\SACTitle{输入}
\begin{description}
\item [MASTER name|n] 通过文件名或文件号指定某文件为主文件,内存中的所有文件将与主文件进行卷积
\item [NUMBER n] 设置卷积窗的数目
\item [LENGTH ON] 打开/关闭固定窗长选项开关
\item [LENGTH v] 打开固定窗长选项开关,并设置窗长度为v秒
\item [TYPE RECTANGLE] 对每个窗应用一个矩形函数,这等价于不对窗加上函数
\item [TYPE HAMMING|HANNING|COSINE|TRIANGlE] 对每个窗应用xx函数,详情参考correlate命令中的说明
\end{description}

\SACTitle{缺省值}
\begin{SACDFT}
convolve master 1 number 1 length off type rectangle
\end{SACDFT}

\SACTitle{说明}
该命令允许用户指定一个主信号,并将主信号与自己及其它信号做卷积。如果内存中有N个SAC文件,
则输出文件为N个文件与主文件卷积的结果。卷积公式如下,
	\[ CV(t) = \int_{-\infty} ^\infty f(\tau)g(t-\tau)d\tau \]
需要注意的是,实际代码中该卷积在频率域完成,且没有进行归一化。内存中所有的信号需要
有相同的delta。

该算法鉴定所有的时间序列都是因果的,因此如果你想要将信号卷积上一个boxcar函数
(非因果低通滤波器,可用于平滑合成波形中的尖峰),输出信号将出现半个boxcar宽度的
时移。

\SACTitle{头段变量}
depmin、depmax、depmen
