\section*{功能命令列表}

\subsection*{信息模块}
\begin{itemize}
\item \nameref{cmd:about}:显示版本和版权信息
\item \nameref{cmd:news}:终端显示关于SAC的一些信息
\item \nameref{cmd:quit}:退出SAC
\item \nameref{cmd:help}:在终端显示SAC命令的语法和功能信息
\item \nameref{cmd:printhelp}:调用打印机打印帮助文档
\item \nameref{cmd:inicm}:重新初始化SAC
\end{itemize}

\subsection*{一元操作模块}
\begin{itemize}
\item \nameref{cmd:add}:为每个数据点加上同一个常数
\item \nameref{cmd:sub}:给每个数据点减去同一个常数
\item \nameref{cmd:mul}:给每个数据点乘以同一个常数
\item \nameref{cmd:div}:对每个数据点除以同一个常数
\item \nameref{cmd:sqr}:对每个数据点做平方
\item \nameref{cmd:sqrt}:对每个数据点取其平方根
\item \nameref{cmd:abs}:对每一个数据点取其绝对值
\item \nameref{cmd:log}:对每个数据点取其自然对数($\ln y$)
\item \nameref{cmd:log10}:对每个数据点取以10为底的对数($\log_{10} y$)
\item \nameref{cmd:exp}:对每个数据点取其指数($e^y$)
\item \nameref{cmd:exp10}:对每个数据点取以10为底的指数($10^y$)
\item \nameref{cmd:int}:利用梯形法或矩形法对数据进行积分
\item \nameref{cmd:dif}:对数据进行微分操作
\end{itemize}

\subsection*{二元操作模块}
\begin{itemize}
\item \nameref{cmd:addf}:使内存中的一组数据加上另一组数据
\item \nameref{cmd:subf}:使内存中的一组数据减去另一组数据
\item \nameref{cmd:mulf}:使内存中的一组数据乘以另一组数据
\item \nameref{cmd:divf}:使内存中的一组数据除以另一组数据
\item \nameref{cmd:binoperr}:控制二元操作addf、subf、mulf、divf中的错误
\item \nameref{cmd:merge}:将多个数据文件合并成一个文件
\end{itemize}

\subsection*{信号校正模块}
\begin{itemize}
\item \nameref{cmd:rq}:从谱文件中去除Q因子
\item \nameref{cmd:rglitches}:去掉信号中的坏点
\item \nameref{cmd:rmean}:去除均值
\item \nameref{cmd:rtrend}:去除线性趋势
\item \nameref{cmd:taper}:对数据两端应用对称的taper函数,使得数据两端平滑地衰减到零
\item \nameref{cmd:rotate}:将成对的正交分量旋转一个角度
\item \nameref{cmd:quantize}:将连续数据数字化
\item \nameref{cmd:interpolate}:对等间隔或不等间隔数据进行插值以得到新采样率
\item \nameref{cmd:stretch}:拉伸(增采样)数据,包含了一个可选的FIR滤波器
\item \nameref{cmd:decimate}:对数据减采样,包含了一个可选的抗混叠FIR滤波器
\item \nameref{cmd:smooth}:对数据应用算术平滑算法
\item \nameref{cmd:reverse}:将所有数据点逆序
\end{itemize}

\subsection*{数据文件模块}
\begin{itemize}
\item \nameref{cmd:funcgen}:生成一个函数并将其存在内存中
\item \nameref{cmd:datagen}:产生样本波形数据并储存在内存中
\item \nameref{cmd:read}:从磁盘读取SAC文件到内存
\item \nameref{cmd:readbbf}:将黑板变量文件读入内存
\item \nameref{cmd:readerr}:控制在执行read命令过程中的错误的处理方式
\item \nameref{cmd:readhdr}:从SAC数据文件中读取头段到内存
\item \nameref{cmd:write}:将内存中的数据写入磁盘
\item \nameref{cmd:writebbf}:将黑板变量文件写入到磁盘
\item \nameref{cmd:writehdr}:用内存中文件的头段区覆盖磁盘文字中的头段区
\item \nameref{cmd:listhdr}:列出指定的头段变量的值
\item \nameref{cmd:chnhdr}:修改指定的头段变量的值
\end{itemize}

\subsection*{图形环境模块}
\begin{itemize}
\item \nameref{cmd:xlim}:设定图形中x轴的范围
\item \nameref{cmd:ylim}:设定图形中y轴的范围
\item \nameref{cmd:linlin}:设置X、Y轴均为线性坐标
\item \nameref{cmd:loglog}:设置X、Y轴均为对数坐标
\item \nameref{cmd:linlog}:设置X轴为线性坐标,Y轴为对数坐标
\item \nameref{cmd:loglin}:设置X轴为对数坐标,Y轴为线性坐标
\end{itemize}

\subsection*{谱分析模块}
\begin{itemize}
\item \nameref{cmd:hanning}:对每个数据文件应用一个``hanning''窗
\item \nameref{cmd:mulomega}:在频率域进行微分操作
\item \nameref{cmd:divomega}:在频率域进行积分操作
\item \nameref{cmd:fft}:对数据做快速离散傅立叶变换
\item \nameref{cmd:ifft}:对数据进行离散反傅立叶变换
\item \nameref{cmd:keepam}:保留内存中谱文件的振幅部分
\item \nameref{cmd:khronhite}:对数据应用Khronhite滤波器
\item \nameref{cmd:correlate}:计算自相关和互相关函数
\item \nameref{cmd:convolve}:计算主信号与内存中所有信号的卷积
\item \nameref{cmd:hilbert}:应用Hilbert变换
\item \nameref{cmd:envelope}:利用Hilbert变换计算包络函数
\item \nameref{cmd:benioff}:对数据使用Benioff滤波器
\item \nameref{cmd:unwrap}:计算振幅和展开相位
\item \nameref{cmd:wiener}设计并应用一个自适应Wiener滤波器
\end{itemize}

\subsection*{分析工具}
\begin{itemize}
\item \nameref{cmd:linefit}:对内存中数据的进行最小二乘线性拟合
\end{itemize}

\newpage
