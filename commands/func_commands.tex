\section*{功能命令列表}

\subsection*{信息模块}
\begin{itemize}
\item \nameref{cmd:about}:显示版本和版权信息
\end{itemize}

\subsection*{一元操作模块}
\begin{itemize}
\item \nameref{cmd:add}:为每个数据点加上同一个常数
\item \nameref{cmd:sub}:给每个数据点减去同一个常数
\item \nameref{cmd:mul}:给每个数据点乘以同一个常数
\item \nameref{cmd:div}:对每个数据点除以同一个常数
\item \nameref{cmd:sqr}:对每个数据点做平方
\item \nameref{cmd:sqrt}:对每个数据点取其平方根
\item \nameref{cmd:abs}:对每一个数据点取其绝对值
\item \nameref{cmd:log}:对每个数据点取其自然对数($\ln y$)
\item \nameref{cmd:log10}:对每个数据点取以10为底的对数($\log_{10} y$)
\item \nameref{cmd:exp}:对每个数据点取其指数($e^y$)
\item \nameref{cmd:exp10}:对每个数据点取以10为底的指数($10^y$)
\item \nameref{cmd:int}:利用梯形法或矩形法对数据进行积分
\item \nameref{cmd:dif}:对数据进行微分操作
\end{itemize}

\subsection*{二元操作模块}
\begin{itemize}
\item \nameref{cmd:addf}:使内存中的一组数据加上另一组数据
\item \nameref{cmd:subf}:使内存中的一组数据减去另一组数据
\item \nameref{cmd:mulf}:使内存中的一组数据乘以另一组数据
\item \nameref{cmd:divf}:使内存中的一组数据除以另一组数据
\item \nameref{cmd:binoperr}:控制二元操作addf、subf、mulf、divf中的错误
\item \nameref{cmd:merge}:将多个数据文件合并成一个文件
\end{itemize}

\subsection*{信号校正模块}
\begin{itemize}
\item \nameref{cmd:rq}:从谱文件中去除Q因子
\item \nameref{cmd:rglitches}:去掉信号中的坏点
\item \nameref{cmd:rmean}:去除均值
\item \nameref{cmd:rtrend}:去除线性趋势
\item \nameref{cmd:taper}:对数据两端应用对称的taper函数,使得数据两端平滑地衰减到零
\item \nameref{cmd:rotate}:将成对的正交分量旋转一个角度
\item \nameref{cmd:quantize}:将连续数据数字化
\item \nameref{cmd:interpolate}:对等间隔或不等间隔数据进行插值以得到新采样率
\item \nameref{cmd:stretch}:拉伸(增采样)数据,包含了一个可选的FIR滤波器
\item \nameref{cmd:decimate}:对数据减采样,包含了一个可选的抗混叠FIR滤波器
\item \nameref{cmd:smooth}:对数据应用算术平滑算法
\item \nameref{cmd:reverse}:将所有数据点逆序
\end{itemize}

\newpage
