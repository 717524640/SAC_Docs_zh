\SACCMD{report}
\label{cmd:report}

\SACTitle{概要}
告知用户当前SAC的状态

\SACTitle{语法}
\begin{SACSTX}
REP!ORT! APF|COLOR|CUT|DEVICES|FILEID|GTEXT|HPF|LINE|MEMORY|MTW|PICKS|
    SYMBOL|TITLE|XLABEL|XLIM|YLABEL|YLIM  
\end{SACSTX}

\SACTitle{输入}
\begin{itemize}
\item APF : 字符数字型震相拾取文件名 
\item COLOR : 当前颜色属性,需要一个图形设备被激活 
\item CUT : 当前CUT状态 
\item DEVICES : 在你的系统上可用的图形设备 
\item FILEID : 当前文件id显示属性 
\item GTEXT : 当前图形文本属性 
\item HPF : HYPO震相拾取文件名 
\item LINE : 当前线型属性 
\item MEMORY : 内存管理器的有效存储块的转储。 
\item MTW : 当前的测量时间窗状态 
\item PICKS : 当前时间拾取显示属性 
\item SYMBOL : 当前符号绘制属性 
\item TITLE : 当前绘制标题属性 
\item XLABEL : 当前x轴标签属性 
\item XLIM : 当前x轴范围 
\item YLABEL : 当前y轴标签属性 
\item YLIM : 当前y轴范围 
\end{itemize} 

\SACTitle{说明}
这个命令用于找出一些SAC选项的当前值,其值将打印到终端

\SACTitle{例子}
为了获取当前颜色属性的列表:
\begin{SACCode}
SAC> report color
 COLOR option is ON
 DATA color is YELLOW
 INCREMENT data color is OFF
 SKELETON color is BLUE
 BACKGROUND color is NORMAL
\end{SACCode}

为了获取HYPO文件名:
\begin{SACCode}
SAC> report apf hpf
 Alphanumeric pick file is MYPICKFILE
 HYPO pick file is HYPOPICKFILE
\end{SACCode}
