\section{transcript}
\label{cmd:transcript}

\SACTitle{概要}
控制输出副本文件

\SACTitle{语法}
TRANSCRIPT [ OPEN | CREATE | CLOSE | CHANGE | WRITE | HISTORY ] [ FILE filename]  [ CONTENTS ALL | ERRORS | WARNINGS | OUTPUT | COMMANDS | MACROS | PROCESSED ] [ MESSAGE text ]

\SACTitle{输入}
\begin{itemize}
\item OPEN : 打开副本文件,并在已存在的文件底部添加副本 
\item CREATE : 创建一个新的副本文件 
\item CLOSE : 关闭一个已经打开的副本文件 
\item CHANGE : 改变一个已经打开的副本文件的内容 
\item WRITE : 写信息到一个副本文件,不改变其状态或内容 
\item HISTORY FILE filename : 将命令行历史保存到文件 
\item FILE filename : 定义副本文件的名字 
\item MESSAGE text : 向副本文件中写入text给出的信息。这个信息可以用于指定正在进行的进程或指定正在处理的不同事件,在两次执行这个命令的过程之中这个信息不保存。 
\item CONTENTS ALL : 定义副本文件的内容为全部输入输出的信息。 
\item CONTENTS list : 定义副本文件的内容,即包含在文件中的输入输出的类型表
	\begin{itemize}
	\item ERRORS : 执行命令期间产生的错误消息
	\item WARNINGS : 执行命令期间产生的警告消息
	\item OUTPUT : 执行命令期间的输出消息 
	\item COMMANDS : 终端输出的原始命令 
	\item MACROS : 宏文件中出现的原始命令 
	\item PROCESSED : 经终端或宏文件处理之后的命令。处理之后的命令其所有宏参数、暂存块变量、头段变量、内联函数都会被计算并带入原始命令中形成新的命令。\\
	\end{itemize}
\end{itemize}

\SACTitle{缺省值}
TRANSCRIPT OPEN FILE TRANSCRIPT CONTENTS ALL

\SACTitle{说明}
副本文件用于记录SAC执行的结果。其可以是一个全部结果或部分结果的副本,可以包含一个或多个操作的执行结果,你可以一次最多拥有5个活动的副本文件,每个文件用于跟踪不同的方面。一个用处是记录在终端输入的命令然后用于一个宏文件中,如下例所示。

\SACTitle{例子}
为了创建一个新的副本文件,文件名为MYTRAN,包含了除已处理命令之外的其他全部类型:
\begin{SACCode}
SAC> TRANSCRIPT CREATE FILE MYTRAN CONTENTS ERRORS WARNINGS OUTPUT COMMANDS MACROS
\end{SACCode}

如果之后不想把宏命令送入这个文件,你可以使用CHANGE选项:
\begin{SACCode}
SAC> TRANSCRIPT CHANGE FILE MYTRAN CONTENTS ERRORS WARNINGS OUTPUT COMMANDS
\end{SACCode}

为了定义一个名为MYRECORD的副本文件,其记录了终端输入的命令:
\begin{SACCode}
SAC> TRANSCRIPT CREATE FILE MYRECORD CONTENTS COMMANDS
\end{SACCode}

以后,经过适当的编辑,这个文件可以用作宏命令文件,去自动执行相同的一组命令。在最后的例子中假设你需要彻夜处理许多事件。你可以设置每个事件一个副本文件(用不同的副本文件名)去记录处理的结果。另外你可以将处理所有事件得到的任何错误消息保存到一个副本文件中:
\begin{SACCode}
SAC> TRANSCRIPT OPEN FILE ERRORTRAN CONTENTS ERRORS
SAC> TRANSCRIPT WRITE MESSAGE 'Processing event 1'
\end{SACCode}

这些命令将放在处理每个事件的宏文件中,它假设事件名作为第一个参数带入宏。使用打开选项,运行信息和出错信息将天骄到文件的后面,第二天检查一下这个出错信息副本文件,就可以快速查阅在处理期间是否出现了错误。

为了将保存一个命令行副本,以记录SAC当前和将来要运行的命令:
\begin{SACCode}
SAC> TRANSCRIPT HISTORY FILE .sachist
\end{SACCode}
这就在当前目录创建并写入了一个副本文件``./.sachist''。任何储存在那里的文件将被载入命令历史中。如果这个命令位于你的启动初始化宏文件中,则每次你运行SAC时将在当前目录产生一个单独的命令行历史。在一个新执行的SAC中,上下键将浏览完整的命令历史,你可以修改以前输入的命令并再次执行它,如果你没有在SAC内或初始化宏文件中输入这个命令,则命令行历史将自动保存到\~/.sac\_history

\SACTitle{最近修订}
September 2008 (version 101.2)
