\SACCMD{synchronize}
\label{cmd:synchronize}

\SACTitle{概要}
同步内存中所有文件的参考时间

\SACTitle{语法}
SYNCHRONIZE [ ROUND ON | OFF ] [ BEGIN ON | OFF ]

\SACTitle{输入}
\begin{itemize}
\item ROUND ON : 打开开始时间取整的选项。当这个选项打开时,每个文件的开始时间都将以最近采样间隔的倍数取整
\item ROUND OFF : 关闭开始时间取整的选项 
\item BEGIN {ON} : 设置每个文件的开始时间为0 
\item BEGIN OFF : 保持参考时间的GMT值 
\end{itemize}

\SACTitle{缺省值}
SYNCHRONIZE ROUND OFF BEGIN OFF

\SACTitle{说明}
这个命令同步内存中所有文件的参考时间。它通过检查所有文件的参考时间和文件起始的偏移时间决定最晚的开始时间,并取这和最晚的开始时间为内存中所有文件的参考时间。然后计算每个文件中的所有相对时间(B, E, A, Tn 等)相对于新参考时间的值。当一系列文件具有不同的参考时间,而你想要用CUT或XLIM命令分析或绘制这些文件中的一部分数据这个命令会很有用。一旦参考时间被同步了,则CUT操作将针对严格相同的GMT时间窗进行。

如果使用了BEGIN选项,参考时间的GMT值则不被保留,BEGIN选项将所有文件的kztime、kzdate设置为相同,而且将所有文件的开始时间设置为0。其他的点保持与文件开始时间的关系。

\SACTitle{例子}
假定你读取两个不同参考时间的文件到内存:
\begin{SACCode}
SAC> READ FILE1 FILE2
SAC> LISTHDR B KZTIME KZDATE

  FILE: FILE1
  -
 B = 0.
 KZTIME = 10:38:14.000
 KZDATE = MAR 29 (088), 1981

  FILE: FILE2
  -
 B = 10.00
 KZTIME = 10:40:10.000
 KZDATE = MAR 29 (088), 1981
\end{SACCode}

这些文件有相同的参考日期,不同的参考时间以及不同的开始时间偏移量。你可以执行	SYNCHRONIZE然后使用LISTHDR,你会发现:
\begin{SACCode}
SAC> SYNCHRONIZE
SAC> LISTHDR

  FILE: FILE1
  -
 B = -126.00
 KZTIME = 10:40:20.000
 KZDATE = MAR 29 (088), 1981

  FILE: FILE2
  -
 B = 0.
 KZTIME = 10:40:20.000
 KZDATE = MAR 29 (088), 1981
\end{SACCode}
现在内存中的所有文件有相同的参考时间,如果头段中有任何已定义的时间标记,它们的值也会调整以使得其GMT值是不变的。

\SACTitle{头段变量改变}
NZYEAR, NZJDAY, NZHOUR, NZMIN, NZSEC, NZMSEC, B, E, A, O, Tn.

\SACTitle{最近修订}
May 15, 1987 (Version 10.2)
