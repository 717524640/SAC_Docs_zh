\SACCMD{readerr}
\label{cmd:readerr}

\SACTitle{概要}
控制发生在READ命令过程中的错误

\SACTitle{语法}
READERR [ Badfile Fatal | Warning | Ignore ] [ Nofiles Fatal | Warning | Ignore ] [ Memory Save | Delete ]

\SACTitle{输入}
\begin{itemize}
\item BADFILE : 当文件不可读或不存在时出现的错误 
\item NOFILES : 文件列表中没有文件可读时出现的错误 
\item FATAL : 设置错误条件为fatal,发送错误消息并停止执行这个命令 
\item WARNING : 发送警告消息,但继续执行命令 
\item IGNORE : 忽略错误条件,继续执行命令 
\item MEMORY : 如果无文件可读则对内存中原有的数据进行处理 
\item DELETE : 先前在内存中的数据将被删除 
\item SAVE : 内存中的数据将保持在内存中 
\end{itemize}

\SACTitle{缺省值}
READERR BADFILE WARNING NOFILES FATAL MEMORY DELETE

\SACTitle{说明}
当你试着使用READ命令将数据文件读入内存时可能会发生错误。文件可能不存在或虽然存在但不可读。当SAC遇到这些badfiles时,一般会发送警告消息,然后试着读取文件列表中余下的文件。如果你想要SAC在遇到坏文件时停止读取文件可以设置BADFILE为FATAL。如果你不想看到警告信息,可以设置BADFILE为INGORE。如果文件列表中文件均不可读,SAC将发送错误信息并停止处理,如果你想要SAC发送	警告信息或完全忽略这个问题,设置NOFILES为INGORE。当然,SAC内存中先前的文件也可以从内存中删除或者保留在内存中。

\SACTitle{相关命令}
READ, CUTERR

\SACTitle{最近修订}
March 20, 1992 (Version 10.6e)
