\SACCMD{width}
\label{cmd:width}

\SACTitle{概要}
控制所选图形设备的线宽

\SACTitle{语法}
\begin{SACSTX}
WIDTH  [ON|OFF|linewidth] [SKELETON width] [INCREMENT ON|OFF] 
    [LIST STANDARD|widthlist]
\end{SACSTX}
其中linewidth, width, widthlist是整数值,且LIST选项必须放在命令的最后

\SACTitle{输入}
\begin{itemize}
\item WIDTH ON : 打开WIDTH选项但是不改变当前线宽值 
\item WIDTH OFF : 关闭width选项 
\item WIDTH linewidth : 改变数据的线宽为linewidth并打开WIDTH选项。 
\item SKELETON width : 改变图形边框宽度为width并打开WIDTH选项 
\item INCREMENT {ON} : 按照widthlist表中的次序,依次改变一个宽度值 
\item INCREMENT OFF : 关闭线宽递变功能 
\item LIST widthlist : 改变宽度列表的内容。输入宽度列表。设置数据宽度为列表中的第一个宽度,并打开width选项。 
\item LIST STANDARD : 设置为标准线宽列表,设置数据宽度为列表中的第一个宽度,并打开width选项。 
\end{itemize}

\SACTitle{缺省值}
\begin{SACDFT}
WIDTH OFF SKELETON 1 INCREMENT OFF LIST STANDARD
\end{SACDFT}

\SACTitle{说明}
这个命令控制那些可显示很多不同线宽的设备的宽度属性。数据宽度是指绘制数据文件时使用的宽度。数据宽度可在每个数据文件绘图之后,按照宽度表中的次序自动地变成另一个宽度。边框宽度是用于绘制坐标轴的宽度。在SKELETON选项下,仅修改坐标轴的宽度。网格、文本、标签和框架号,总是用1号细线显示。
如果在同一张绘图中同时绘制几个数据文件,你也许需要每个文件有不同的宽度。此时可使用INCREMENT选项。在这个选项打开时,每次绘制一个数据文件后,都按照宽度表中的次序自动地变成另一个宽度。宽度值和次序在标准宽度表中为:
\begin{SACCode}
1, 2, 3, 4, 5, 6, 7, 8, 9, 10
\end{SACCode}
你可以使用LIST选项改变这个表的次序或内容。这个命令常用于重叠绘图(参见PLOT2),此时你可能需要每张图上的数据宽度都按相同的顺序排列。

\SACTitle{例子}
选择自动变换的数据宽度起始值为1:
\begin{SACCode}
SAC> width 1 increment
\end{SACCode}

边框宽度起始值为2,并按1、3、5的增量变化:
\begin{SACCode}
SAC> width skeleton 2 increment list 1 3 5
\end{SACCode}
