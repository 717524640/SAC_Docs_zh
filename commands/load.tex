\section{load}
\label{cmd:load}

\SACTitle{概要}
导入外部命令(在SAC的linux版本中外部命令的导入需要额外的工作)

\SACTitle{语法}
LOAD comname [ ABBREV abbrevname ]

\SACTitle{输入}
\begin{itemize}
\item comname : 从一个共享目标文件中载入的一个外部函数名 
\item ABBREV abbrevname : 命令的简写 
\end{itemize}

\SACTitle{说明}
这个命令允许用户载入由SAC外部命令接口写的命令。(参见EXTERNAL\_INTERFACE文档)。这个命令必须是一个储存在共享目标库(.so文件)中。SAC将检查所有环境变量SACSOLIST中的共享目标库,这个环境变量需要包含一个或多个共享目标库,每个库以空格分割。到这些共享目标库的路径必须在环境变量LD\_LIBRARY\_PATH中指定。如果SACSOLIST为设置,则SAC将使用由LD\_LIBRARY\_PATH中指定的路径在共享文件库libsac.so中寻找。一个叫做libcom.so的库随着SAC发布。

\SACTitle{例子}
设置你的环境变量使得SAC在当前目录从库文件libbar.so中查找一个称为foo的命令,并为foo设置别名为myfft:
\begin{SACCode}
% export SACSOLIST = "libcom.so libbar.so"
#  Add the current directory to the search path.
% export LD_LIBRARY_PATH {$LD_LIBRARY_PATH}:. 
% sac
SAC> load foo abbrev myfft* load the command
SAC> read file1.z file2.z file3.z  * input files to pass to the command
SAC> myfft real-imag* invoke command with its arguments,
* commands must parse their own args.
SAC> psp
\end{SACCode}

如何创建一个包含你的命令的共享目标库:
\begin{SACCode}
Solaris::
cc -o libxxx.so -G extern.c foo.c bar.c
SGI::
cc -g -o libxxx.so -shared foo.c bar.c
LINUX: (gcc)::
gcc -o libxxx.so -shared extern.c foo.c bar.c sac.a
\end{SACCode}
其中sac.a可以从你得到sac的地方获得

\SACTitle{SAC发布的外部命令}
在SAC的发布版中有一个外部命令,叫做FLIPXY。FLIPXY将一个或多个X-Y数据文件作为输入,并调换X和Y。这个命令在SAC/aux/external中的libcom.so中,同时还有FLIPXY的源代码作为参考。为了导入FLIPXY,libcom.so必须包含在SACSOLIST中

\SACTitle{错误消息}
\begin{itemize}
\item[-]1028: 外部命令不存在:这意味着SAC无法找到你的外部命令
\end{itemize}

这个错误产生的原因很多,一个可能是环境变量LD\_LIBRARY\_PATH不包含到达你的共享库的目录。
另一个可能的原因是SACSOLIST不包含你的共享库的名字

\SACTitle{最近修订}
March 21, 1996 (Version 00.50)

