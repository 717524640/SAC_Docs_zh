\SACCMD{writebbf}
\label{cmd:writebbf}

\SACTitle{概要}
将一个暂存块变量写入磁盘

\SACTitle{语法}
\begin{SACSTX}
W!RITE!BBF [file]
\end{SACSTX}

\SACTitle{输入}
\begin{itemize}
\item file : 暂存块变量文件的名字,其可以是简单文件名或相对/绝对路径。
\end{itemize}

\SACTitle{缺省值}
\begin{SACDFT}
WRITEBBF BBF
\end{SACDFT}

\SACTitle{说明}
这个命令让你能够将暂存块变量写入磁盘,其可以稍后用READBBF命令读入SAC,这个特性允许你保存SAC某次执行得到信息到另一个。你也可以在自己的程序中写代码去访问这些暂存块变量文件的信息。这使你可以在自己的程序与SAC之间交换信息。

\SACTitle{相关命令}
\nameref{cmd:readbbf}、\nameref{cmd:setbb}、\nameref{cmd:getbb}
