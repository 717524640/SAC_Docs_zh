\SACCMD{xlim}
\label{cmd:xlim}

\SACTitle{概要}
设定图形中x轴的范围

\SACTitle{语法}
\begin{SACSTX}
XLIM [ON|OFF|pdw|SIGNAL]
\end{SACSTX}

\SACTitle{输入}
\begin{description}
\item [ON] 打开x轴范围选项,但不改变范围值
\item [OFF] 关闭x轴范围选项,即根据数据的自变量范围决定X轴范围
\item [pdw] 打开x轴范围选项并设置范围为新的``partial data window''。关于pdw请参见CUT命令
\item [SIGNAL] 等同于输入:``A -1 F +1'',即初至前1秒到事件结束的后1秒。
\end{description}

\SACTitle{缺省值}
\begin{SACDFT}
xlim off
\end{SACDFT}

\SACTitle{说明}
当此选项关闭时,根据数据自变量的范围决定绘图时x轴的范围。当此选项打开时,限定x轴的范围,
可以通过此种方式``放大''当前内存中数据的图形。

\SACTitle{相关命令}
\nameref{cmd:cut}
