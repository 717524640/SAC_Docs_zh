\SACCMD{zlines}
\label{cmd:zlines}

\SACTitle{概要}
控制后续等值线绘图上的等值线线型

\SACTitle{语法}
ZLINES  [ ON | OFF ] [ LIST n1 n2 ... nn ] [ REGIONS v1 v2 ... vn ]

\SACTitle{输入}
\begin{itemize}
\item ON|OFF : 打开等值线显示选项 
\item LIST n1 n2 .. nn : 设置要使用的线型表,这个表上的每个输入用于相应的等值线。如果等值线的数目大于这个表中给出的线型的数目,则使用整个线型表 
\item REGIONS v1 v2 .. vn : 设置等值线范围表。这个表的长度应小于线型表的长度,小于范围值的等值线使用线型表中相应的线型。超过最后一个范围值的等值线采用线型表中最后一个线型的值
\end{itemize}

\SACTitle{缺省值}
ZLINES ON LIST 1

\SACTitle{例子}
循环四种不同线型,建立等值线:
\begin{SACCode}
SAC> ZLINES LIST 1 2 3 4
\end{SACCode}

设置虚线表示低于0.0等值线,实线表示高于0.0的等值线:
\begin{SACCode}
SAC> ZLINES LIST 2 1 REGIONS 0.0
\end{SACCode}

\SACTitle{相关命令}
CONTOUR

\SACTitle{最近修订}
April 30, 1990 (Version 10.5b)
