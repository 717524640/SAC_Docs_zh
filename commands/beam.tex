\SACCMD{beam}
\label{cmd:beam}

\SACTitle{概要}
利用内存中的全部数据文件计算射线束

\SACTitle{语法}
BEAM [Bearing v] [Velocity v] [REFerence ON|OFF| lat lon [el]] [OFFSET REF|USER|STATION|EVENT|CASCADE] [Ec anginc survel] [Center x y z] [WRite fname]

\SACTitle{输入}
\begin{itemize}
\item BEARING v: 方位,由北算起的度数
\item VELOCITY v: 速度,单位为公里每秒
\item REFERENCE lat lon [el]: 参考点,打开REFERENCE选项并定义参考点,这样其他文件的偏移量以此而定。lat、lon、el分别代表纬度、经度、深度(下为正)
\item REFERENCE ON|OFF: 开或关REFERENCE选项
\item OFFSET REF: 偏移量是相对于REFERENCE选项设置的参考点的。这要求开启REFERENCE选项
\item OFFSET USER: 偏移量直接从USER7, USER8以及USER9中获取,(分别代表纬度、经度以及海拔)。这就要求所有文件的USER7和USER8必须定义。如果设置了EC选项,则OFFSET USER要求USER9必须被设置。
\item OFFSET STATION: 偏移量相对于第一个台站的位置,这要求所有文件的STLA、STLO必须定义
\item OFFSET EVENT: 偏移量相对于第一个事件的位置,这要求所有文件的EVLA、EVLO必须定义
\item OFFSET CASCADE: SAC将会按照前面给出的顺序考虑决定偏移量的方法,并检查必要的数据是否具备。它将使用第一个满足要求的方法
\item EC: 高程校正
	\begin{itemize}
	\item anginc: 入射角,从z轴算起,单位为度(震源距离越远,入射角越小
	\item survel: 表面介质速度(km/s)
	\end{itemize} 
\item CENTER :用于计算射线束的中心台站
	\begin{itemize}
	\item x: 距参考台站的东向偏移
	\item y: 距参考台站的北向偏移
	\item z: 距参考台站的向上偏移,其单位为米
	\end{itemize}
\item WRITE fname: 将射线束写入磁盘
\end{itemize}

\SACTitle{缺省值}
BEAM  B 90  V 9.0 EC 33  6.0 C  0. 0. 0. W beam

\SACTitle{说明}
BEAM不覆盖SAC内存中的输入数据,因而当变换方位和速度时这一操作可以重复执行。
射线结果写入到磁盘文件中,并且每次可以写到不同的文件。这个设计考虑到了用户
的需求,即比较反复使用这一命令的不同结果,以寻找最佳射线束的方位和速度。

\SACTitle{头段数据}
参见BBFK命令

\SACTitle{错误消息}
CENTER参数缺失偏移量,EC参数需要正值

\SACTitle{相关命令}
MAP

\SACTitle{最近修订}
July 22, 1991 (Version 10.5c)
