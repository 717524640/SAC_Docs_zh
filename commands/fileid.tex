\SACCMD{fileid}
\label{cmd:fileid}

\SACTitle{概要}
控制文件id在SAC绘图上的显示

\SACTitle{语法}
\begin{SACSTX}
FILEID [ON|OFF] [T!ype! D!efault!|N!ame!|L!ist! hdrlist] Location [UR|UL|LR|LL] 
    [F!ormat! E!quals!|C!olons!|N!onames!]
\end{SACSTX}

\SACTitle{输入}
\begin{itemize}
\item FILEID ON : 打开文件id选项,不改变文件id类型或位置 
\item FILEID OFF : 关闭文件id选项 
\item TYPE DEFAULT : 设置文件id为默认类型 
\item TYPE NAME : 使用文件名作为文件id 
\item TYPE LIST hdrlist : 定义在文件id中显示的头段列表 
\item LOCATION UR|UL|LR|LL : 文件id放置位置,分别表示右上角、左上角、右下角、左下角
\item FORMAT EQUALS : 格式为头段名,等于号以及头段值 
\item FORMAT COLON : 格式为头段名,冒号,头段值 
\item FORMAT NONAMES : 格式只包含头段值 
\end{itemize}

\SACTitle{缺省值}
\begin{SACDFT}
fileid on type default location ur format nonames
\end{SACDFT}

\SACTitle{说明}
文件id标识了绘图的内容,默认的文件id包括事件名、台站名、分量、参考日期及时间。如果需要也可以使用文件名代替默认的文件id。还可以定义一个特殊的文件id,这个id最多由10个SAC头段变量构成。文件id的位置以及格式也可以修改。

\SACTitle{例子}
将文件名放在左上角:
\begin{SACCode}
SAC> fileid location ul type name
\end{SACCode}

定义一个特殊的文件id,包含台站分量、经纬度:
\begin{SACCode}
SAC> fileid type list kstcmp stla stlo
\end{SACCode}

文件id为头段名后加一个冒号:
\begin{SACCode}
SAC> fileid format colon
\end{SACCode}
