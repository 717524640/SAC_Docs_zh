\section{fileid}
\label{cmd:fileid}

\SACTitle{概要}
控制文件id在SAC绘图上的显示

\SACTitle{语法}
FILEID [ ON | OFF ] [ Type Default | Name | List hdrlist ]Location [ UR | UL | LR | LL ] [ Format Equals | Colons | Nonames]

\SACTitle{输入}
\begin{itemize}
\item FILEID ON : 打开文件id选项,不改变文件id类型或位置 
\item FILEID OFF : 关闭文件id选项 
\item TYPE DEFAULT : 设置文件id为默认类型 
\item TYPE NAME : 使用文件名作为文件id 
\item TYPE LIST hdrlist : 定义在文件id中显示的头段列表 
\item LOCATION UR|UL|LR|LL : 文件id放置位置,分别表示右上角、左上角、右下角、左下角
\item FORMAT EQUALS : 格式为头段名,等于号以及头段值 
\item FORMAT COLON : 格式为头段名,冒号,头段值 
\item FORMAT NONAMES : 格式只包含头段值 
\end{itemize}

\SACTitle{缺省值}
FILEID ON TYPE DEFAULT LOCATION UR FORMAT NONAMES

\SACTitle{说明}
文件id标识了绘图的内容,默认的文件id包括事件名、台站名、分量、参考日期及时间。如果需要也可以使用文件名代替默认的文件id。还可以定义一个特殊的文件id,这个id最多由10个SAC头段变量构成。文件id的位置以及格式也可以修改。

\SACTitle{例子}
将文件名放在左上角:
\begin{SACCode}
SAC> FILEID LOCATION UL TYPE NAME
\end{SACCode}

定义一个特殊的文件id,包含台站分量、经纬度:
\begin{SACCode}
SAC> FILEID TYPE LIST KSTCMP STLA STLO
\end{SACCode}

文件id为头段名后加一个冒号:
\begin{SACCode}
SAC> FILEID FORMAT COLON
\end{SACCode}

\SACTitle{最近修订}
October 11, 1984 (Version 9.1)
