\SACCMD{copyhdr}
\label{cmd:copyhdr}

\SACTitle{概要}
从内存中一个文件copy头段变量给其他所有文件

\SACTitle{语法}
COPYHDR [FROM name|n] hdrlist

\SACTitle{输入}
\begin{itemize}
\item FROM name: 从内存中文件名为name的文件中copy头段列表 
\item FROM n : 从内存中第n个文件中copy头段列表 
\item hdrlist: 用空格分割的要copy的头段变量 
\end{itemize}

\SACTitle{缺省值}
COPYHDR FROM 1

\SACTitle{说明}
这个命令允许你从内存中的一个文件copy任意头段变量值到内存中其他所有文件。你可以
择从哪个文件copy头段。其目的是为了使数据有相同的头段使得数据易于处理。

\SACTitle{例子}
假设你使用PPK在文件FILE1中获得了多个时间标记,并将其储存到头段变量T3和T4中,为了将这些标记copy到FILE2和FILE3中:
\begin{SACCode}
SAC> r FILE1
SAC> ppk
SAC> ... use cursor to mark times T3 and T4.
SAC> r more FILE2 FILE3
SAC> copyhdr from 1 T3 T4
\end{SACCode}
假设你读取了很多文件,你想要copy文件ABC中的头段变量evla和evlo到其他所有文件中去,这时使用文件名而非数字会更简单:
\begin{SACCode}
SAC> COPYHDR FROM ABC STLA STLO
\end{SACCode}

\SACTitle{头段变量改变}
几乎所有头段变量都可以改变

\SACTitle{最近修订}
May 15, 1987 (Version 10.2)
