\SACCMD{picks}
\label{cmd:picks}

\SACTitle{概要}
控制在多数绘图上时间标记的显示

\SACTitle{语法}
\begin{SACSTX}
PICKS [ON|OFF] [tmarker VERTICAL|HORIZONTAL|CROS] [WIDTH v] [HEIGHT v]
\end{SACSTX}

\SACTitle{输入}
\begin{itemize}
\item PICKS ON : 打开震相拾取显示 
\item PICKS OFF : 关闭震相拾取显示 
\item tmarker : SAC的时间标记头段变量名,可以取O | A | F | Tn(n=0-9)
\item VERTICAL : 在时间标记处绘制垂直线,震相拾取id位于线的右下方。
\item HORIZONTAL : 位于最接近时间标记的数据点处绘制水平线,震相拾取id位于线上或线下
\item CROSS : 在时间标记处绘制垂直线,在最近的数据点处绘制水平线 
\item WIDTH v : 拾取标记的显示宽度为v 
\item HEIGHT v : 改变拾取标记的显示高度为v。高度和宽度仅对HORIZONTAL和CROSS使用,其数值为占图形的比例 
\end{itemize}

\SACTitle{缺省值}
\begin{SACDFT}
pick on width 0.1 height 0.1
\end{SACDFT}
所有的拾取标记类型都是VERTICAL

\SACTitle{说明}
这个命令控制多数SAC绘图上时间标记的显示信息。这些时间标记标识了如震相到时、事件发生时刻等。当打开显示选项时,每一个定义了的时间标记都会在绘图上相应时刻处绘制一条线,并且在其旁有一个时间标记id。时间标记id是一个8字符长的头段变	量。头段变量中KA、KF、KO以及KTn分别是A、F、O 和Tn的时间标记id。如果时间标记id未定义,那么标记的名字就是就本身。每一个时间标记可以被显示为一条垂直线、一条水平线或一个交叉线。

\SACTitle{例子}
以交叉符号显示时间标记T4、T5和T6,并改变交叉符号的高度和宽度:
\begin{SACCode}
SAC> picks t4 c t5 c t6 c w 0.3 h 0.1
\end{SACCode}
