\section{plotsp}
\label{cmd:plotsp}

\SACTitle{概要}
用多种格式绘制谱数据

\SACTitle{语法}
PlotSP [ ASIS | RLIM | AMPH | RL | IM | AM | PH ] [ LINLIN | LINLOG | LOGLIN | LOGLOG ]

\SACTitle{输入}
\begin{itemize}
\item ASIS :  按照谱文件当前格式绘制分量
\item RLIM :  绘制实部和虚部分量 
\item AMPH :  绘制振幅和相位分量 
\item RL :  只绘制实部分量 
\item IM :  只绘制虚部分量 
\item AM :  只绘制振幅分量 
\item PH :  只绘制相位分量 
\item LINLIN|LINLOG|LOGLIN|LOGLOG : 设置x-y轴为线型还是对数型,与单独的LINLIN等命令区分开
\end{itemize}

\SACTitle{缺省值}
PLOTSP ASIS LOGLOG

\SACTitle{说明}
SAC数据文件可能包含时间序列文件或谱文件,IFTYPE决定文件所有哪种类型。多数绘图命令只能对时间序列文件起作用,这个命令则可以绘制谱文件。

你可以使用这个命令绘制一或两个分量。每一个分量绘制在一张图上。你也可以设置横纵坐标为线型或对数型,其仅对该命令有效.

\SACTitle{例子}
获得一个谱文件振幅的对数-线性的绘图:
\begin{SACCode}
SAC> READ FILE1
SAC> FFT
SAC> PLOTSP AM LOGLIN
\end{SACCode}

\SACTitle{错误消息}
\begin{itemize}
\item[-]1301: 未读入数据文件
\item[-]1305: 对时间序列的非法操作
\end{itemize}

\SACTitle{最近修订}
May 15, 1987 (Version 10.2)
