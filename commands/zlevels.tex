\SACCMD{zlevels}
\label{cmd:zlevels}

\SACTitle{概要}
控制后续等值线图上的等值线间隔

\SACTitle{语法}
\begin{SACSTX}
ZLEVELS [SCALE] [RANGE v1 v2] [INCREMENT v] [NUMBER n] [LIST v1 v2 ... vn]
\end{SACSTX}

\SACTitle{输入}
\begin{description}
\item [SCALE] 根据数据自动确定等值线的标尺范围 
\item [RANGE v1 v2] 用户设置等值线的范围(最小和最大)为v1和v2。可以使用SCALE选项,也可以使用RANGE选项,但不可同时使用二者 
\item [INCREMENT v] 设置等值线之间的增量为v 
\item [NUMBER n] 设置等值线的条数为n,你可以使用INCREMENT或NUMBER选项但不可二者同时使用 
\item [LIST v1 v2 .. vn] 设置一系列等值线上的值为v1、v2等等,如果使用这个选项,则其他选项均被忽略 
\end{description}

\SACTitle{缺省值}
\begin{SACDFT}
zlevels scale number 20
\end{SACDFT}

\SACTitle{示例}
参考contour中zlevels的使用

\SACTitle{限制}
等值线的最多数目为40

\SACTitle{相关命令}
\nameref{cmd:contour}
