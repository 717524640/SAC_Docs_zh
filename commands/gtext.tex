\SACCMD{gtext}
\label{cmd:gtext}

\SACTitle{概要}
控制绘图中文本质量以及字体

\SACTitle{语法}
\begin{SACSTX}
GT!EXT! [S!oftware!|H!ardware!] [F!ont! n] [S!ize! T!iny!|S!mall!|M!edium!|L!arge!]
\end{SACSTX}

\SACTitle{输入}
\begin{itemize}
\item SOFTWARE:  绘图中使用软件文本
\item HARDWARE:  绘图中使用硬件文本 
\item FONT n:  设置软件文本字体为n,n取值为1到8 
\item SIZE size:  改变缺省文本大小,具体请参考TSIZE 
\end{itemize}

\SACTitle{缺省值}
\begin{SACDFT}
gtext software font 1 size small
\end{SACDFT}

\SACTitle{说明}
软件文本使用了图形库的文本显示功能,字符作为小线段存储,因而具有任意的大小以及旋转至任意的角度,使用软件文本将在不同图形设备上产生相同的结果,但是其速度会慢于硬件文本,尤其对于终端来说。目前有8种可用的软件字体:simplex block (font 1), simplex italics (2), duplex block(3), duplex italics (4), complex block (5), complex italics(6), triplex block (7),以及triplex italics (8).  

硬件文本使用图形设备自身的文本显示功能,因而文本大小因不同的设备而不同,所以使用硬件文本会导致在不同的图形设备上看到不同的图。如果一个设备有超过一个硬件文本尺寸,那么最究竟预期值的那个将被使用。其最主要优点在于速度较快,因而当速度比质量重要时可以使用。

\SACTitle{例子}
选择triplex软件字体:
\begin{SACCode}
SCA> gtext software font 6
\end{SACCode}

\SACTitle{相关命令}
\nameref{cmd:tsize}
