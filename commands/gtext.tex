\SACCMD{gtext}
\label{cmd:gtext}

\SACTitle{概要}
控制绘图中文本质量以及字体

\SACTitle{语法}
\begin{SACSTX}
GT!EXT! [S!OFTWARE!|H!ARDWARE!] [F!ONT! n] [SIZE size] [SYS!TEM! system] [N!AME! name]
\end{SACSTX}

\SACTitle{输入}
\begin{description}
\item [SOFTWARE]  绘图中使用软件文本
\item [HARDWARE]  绘图中使用硬件文本 
\item [FONT n] 设置软件文本字体为n,n取值为1到8 
\item [SIZE]  改变缺省文本大小,可以取TINY、SMALL、MEDIUM、LARGE,这些缺省文本尺寸
    的具体大小可以参考TSIZE命令
\item [SYSTEM system] 修改字体子系统,可以取值为SOFTWARE、CORE、XFT
\item [NAME name] 修改CORE或XFT子系统的默认字体名,可以取Helvetica、Times-Roman、
    Courier、ZapfDingbats
\end{description}

\SACTitle{缺省值}
\begin{SACDFT}
gtext software font 1 size small
\end{SACDFT}

\SACTitle{说明}
软件文本使用了图形库的文本显示功能,将每个字符以线段的形式保存起来,因而可以任意
缩放或旋转至任意角度。使用软件文本在不同图形设备上可以产生相同的结果,但是其速度会
慢于硬件文本。目前有8种可用的软件字体:
\begin{enumerate}
\item simplex block
\item simplex italics
\item duplex block
\item duplex italics
\item complex block
\item complex italics
\item triplex block
\item riplex italics
\end{enumerate}

硬件文本使用图形设备自身的文本显示功能,因而文本在不同的设备上尺寸可能不同,
所以使用硬件文本会导致在不同的图形设备上看到不同的图。如果一个设备有超过一个
硬件文本尺寸,那么最接近预期值的那个尺寸将被使用。
其最主要优点在于速度较快,因而当速度比质量重要时可以使用。

\SACTitle{例子}
选择triplex软件字体:
\begin{SACCode}
SCA> gtext software font 6
\end{SACCode}

\SACTitle{相关命令}
\nameref{cmd:tsize}
