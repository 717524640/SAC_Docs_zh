\SACCMD{oapf}
\label{cmd:oapf}

\SACTitle{概要}
打开一个字母数字型震相拾取文件

\SACTitle{语法}
OAPF [ STANDARD | NAME ] [ file ]

\SACTitle{输入}
\begin{itemize}
\item STANDARD : 在写震相拾取文件时使用标准文件id。标准文件id包含了头段中事件名、台站名、分量方位角以及入射角。 
\item NAME : 使用SAC文件名代替标准文件id 
\item file : 要打开的字符数字型震相拾取文件,如果该文件已经存在,则其将会被打开,新的震相拾取会加到文件的底部 
\end{itemize}

\SACTitle{缺省值}
OAPF STANDARD APF

\SACTitle{说明}
震相拾取文件可以作为自动拾取(APK)以及人工拾取(PPK)命令产生的简单数据库。每个震相拾取被写在文件的一行上。这个文件的每个常规行包含一个文件id、一个震相拾取id、震相拾取的时间、拾取的幅度以及一些格式信息。这些行为80个字符长。文件id是一些标准的头段信息或者文件名。拾取到的时间是GMT时间或者时间偏移。这依赖于产生这些拾取的命令如APK或PPK中指定的选项。这将导致4个不同的格式,在第79列上以不同的字符标识这些行,如那些波形和峰峰值的读取数据,在80列以后不附加字段。一行的最大长度为200。下面会展示不同格式的行。

\SACTitle{错误消息}
\begin{itemize}
\item[-]1903: 不能关闭前面的震相文件
\item[-]1902: 不能打开震相文件
	\begin{itemize}
	\item[-]可能是文件名中的非法字符引起的
	\end{itemize}
\end{itemize}

\SACTitle{相关命令}
PLOTPK , APK , CAPF

\SACTitle{最近修订}
January 8, 1983 (Version 8.0)
