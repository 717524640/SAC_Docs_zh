\SACCMD{yvport}
\label{cmd:yvport}

\SACTitle{概要}
定义Y轴的视口

\SACTitle{语法}
YVPORT yvmin yvmax

\SACTitle{输入}
\begin{itemize}
\item Yvmin : Y轴视口的最小值,范围为0.0到yvmax 
\item Yvmax : Y轴视口的最大值,范围为ymin到1.0 
\end{itemize}

\SACTitle{缺省值}
YVPORT 0.1 0.9

\SACTitle{说明}
视口是实际绘出的图形窗口的一部分(参见VSPACE)。用于定义视口和视窗的坐标系称为虚拟坐标系。虚拟坐标系不依赖于特定物理设备显示表面的尺寸、形状或分辨率。AC的坐标系在x和y方向都是0到1的范围。视窗左下角的坐标为(0.0,0.0),右上角的坐标为(1.0, 1.0)。这个坐标系的使用便于你指定一个图形的位置,而不必考虑特定的输出设备。

XVPORT和YVPORT命令控制在视窗中哪个位置上绘制指定的图形。默认值使用了视窗的部分,在图形的每边留下一些空间绘制坐标轴、标签和标题。你可以使用这个命令把一个给定的图形安排在任何位置。当与BEGINFRAME和ENDFRAME命令一起使用,这些命令让你创建特殊的构图以在相同的框架中放置若干不同的图形。

\SACTitle{例子}
参见BEGINFRAME

\SACTitle{相关命令}
VSPACE, YVPORT, BEGINFRAME

\SACTitle{参考文献}
Principles of Interactive Computer Graphics,Second Edition;William M. Newman and Robert F. Sproull; 1979; McGraw-Hill.

\SACTitle{最近修订}
January 8, 1983 (Version 8.0)

