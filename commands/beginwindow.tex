\SACCMD{beginwindow}
\label{cmd:beginwindow}

\SACTitle{概要}
启动一个新的X绘图窗口

\SACTitle{语法}
\begin{SACSTX}
B!EGIN!W!INDOW! n
\end{SACSTX}

\SACTitle{输入}
\begin{description}
\item [n] 要启用的绘图窗口的编号,目前n的取值为1到10
\end{description}

\SACTitle{缺省值}
\begin{SACDFT}
beginwindow 1
\end{SACDFT}

\SACTitle{说明}
现在的图形终端或工作站大多支持多窗口,即启动多个窗口,并在每个窗口中显示相同或不同的图像。

windows命令可以控制每个X绘图窗口的位置和形状,而beginwindow则用于启用该绘图
窗口,接下来所有的绘图命令都将显示在该绘图窗口中。若你所选择的绘图窗口没有打开,
则beginwindow会首先创建这个窗口。

需要注意的是,window命令只在绘图窗口被创建之前起作用,即window命令是一个参数设定类
命令。在多数系统上,均允许通过鼠标拖曳的方式动态改变这些窗口的大小。一般情况下,
在动态改变窗口大小或比例之后,当前窗口的绘图会自动重画以适应新窗口。

需要注意的是,这个命令没有与之对应的endwindow。

\SACTitle{相关命令}
\nameref{cmd:window}
