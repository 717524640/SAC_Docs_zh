\section{beginwindow}
\label{cmd:beginwindow}

\SACTitle{概要}
开始在一个新的图形窗口中绘图

\SACTitle{语法}
BeginWindow n

\SACTitle{输入}
\begin{itemize}
\item n: 要绘图的图形窗口号,目前n的取值为1到10
\end{itemize}

\SACTitle{缺省值}
BEGINWINDOW 1

\SACTitle{说明}
许多新的图形终端和工作站支持多窗口的概念。不同的工作可以在不同的窗口中完成并且将结果输出到不同的窗口中,X-windows是目前常用的窗口系统之一。如果你使用的设备支持X系统,你就可以在SAC中使用多个图形窗口来显示你的结果。如果你没有使用这样一个设备,那么SAC会接收并忽略所有与多图形窗口有关的命令。

有两个命令控制多窗口选项所使用。WINDOW命令能够控制图形窗口的位置和形状。	BEGINWINDOW命令让你选择在哪个窗口显示接下来的绘图。如果你所选择的图形窗口没有打开,则BEGINWINDOW会首先创建这个窗口。WINDOW命令只在窗口n被创建之前起作用。在多数系统上,你可以通过鼠标移动或弹出式菜单动态改变这些窗口的大小。通常但不总是(你应该检查一下你自己的系统),移动一个窗口导致当前绘图自动重画,然而改变一个窗口的尺寸则导致当前图形重画但不改变比例。但在改变窗口尺寸后的下一个绘图仍将有正确的比例。

*这个命令可没有与之对应的endwindow命令,因为根本不需要~*

\SACTitle{相关命令}
WINDOW

\SACTitle{最近修订}
May 15, 1987 (Version 10.2)
