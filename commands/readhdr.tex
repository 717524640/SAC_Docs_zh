\SACCMD{readhdr}
\label{cmd:readhdr}

\SACTitle{概要}
从SAC数据文件中读取头段到内存

\SACTitle{语法}
\begin{SACSTX}
R!EAD!H!DR! [MORE] [DIR CURRENT|name] [filelist]
\end{SACSTX}

\SACTitle{输入}
\begin{description}
\item [MORE] 将新数据头段放在内存中老文件头段之后。若如果忽略,则新数据文件的头段将代替内存中原文件的头段
\item [DIR CURRENT] 从当前目录读取文件。这里的当前目录是指启动SAC的目录
\item [DIR name] 从目录name中读取文件,目录名可以是绝对路径或相对路径 
\item [filelist] 文件名列表。其可以是简单文件名也可以使用通配符,路径名可以是相对路径或绝对路径
\end{description}

\SACTitle{说明}
这个命令将一系列SAC文件的头段读入内存,你可以列出头段内容(\nameref{cmd:listhdr})、
改变头段值(\nameref{cmd:chnhdr})、将头段写回磁盘(\nameref{cmd:writehdr})。当你
只需要文件的头段的时候,只读取头段要比读取整个文件到内存快很多。

\SACTitle{错误消息}
\begin{itemize}
\item[-]1301: 未读入数据文件(未给出要读取的文件列表或列表中文件不可读)
\item[-]1314: 数据文件列表不得以数字开头
\item[-]1315: 文件列表中的最大数目为1000
\item[-]1335: 非法操作-只有头段在内存中
\end{itemize}

\SACTitle{警告消息}
\begin{itemize}
\item[-]0101::打开文件错误
\item[-]0108::文件不存在
\item[-]0114::读取文件错误
\end{itemize}

\SACTitle{相关命令}
\nameref{cmd:read}、\nameref{cmd:listhdr}、\nameref{cmd:chnhdr}、\nameref{cmd:writehdr}、
\nameref{cmd:readerr}
