\section{readhdr}
\label{cmd:readhdr}

\SACTitle{概要}
从SAC数据文件中读取头段到内存

\SACTitle{语法}
READHDR [ MORE ] [ TRUST ON | OFF ] [COMMIT|ROLLBACK|RECALLTRACE] [ DIR CURRENT | name ] [ filelist ]

\SACTitle{输入}
\begin{itemize}
\item MORE : 将新数据头段放在内存中老文件头段之后,如果忽略,则新数据文件头段将代替老的。注意:如果MORE选项没有指定则COMMIT, ROLLBACK和RECALLTRACE不起作用\\
\item TRUST ON|OFF : 这个选项用于解决将文件从SAC转换到CSS格式过程中出现的问题。当转换数据时,匹配的事件id意味着文件含有相同的事件信息,或者可能是两个不同格式合并之后出现的干扰。当TRUST打开时,SAC将更有可能接受匹配事件id 作为相同的事件信息,这依赖于READ命令与当前内存中数据的历史。
\item COMMIT|ROLLBACK|RECALLTRACE  : 参见具体章节 
\item DIR CURRENT : 从当前目录读取所有简单文件名,这个目录是你启动SAC的目录 
\item DIR name : 从目录name中读取全部简单文件,其可以为绝对/相对路径 
\item filelist :  file | wild 
\item file : 一个文件名。其可以是简单文件名或路径名。路径名可以是相对/绝对路径。 
\item wild : 通配符以扩展文件名列表 
\end{itemize}

\SACTitle{说明}
这个命令将一系列SAC文件的头段读入内存,你可以列出头段内容(LISTHDR),改变头段值(CHNHDR),将头段写回磁盘(WRITEHDR),这比读取整个文件到内存快的多,当然这在你只需要头段的时候才可行。

\SACTitle{错误消息}
\begin{itemize}
\item[-]1301: 未读入数据文件
	\begin{itemize}
	\item[-]未给出要读取的文件列表
	\item[-]列表中文件不可读
	\end{itemize}
	\item[-]1314: 数据文件列表不得以数字开头
	\item[-]1315: 文件列表中的最大数目为1000
	\item[-]1335: 非法操作-只有头段在内存中
\end{itemize}

\SACTitle{警告消息}
\begin{itemize}
\item[-]0101::打开文件错误
\item[-]0108::文件不存在
\item[-]0114::读取文件错误
	\begin{itemize}
	\item[-]一般SAC遇到这样的错误时会跳过该文件并继续读取余下文件。这些错误可以在READERR设置为fatal
	\end{itemize}
\end{itemize}

\SACTitle{相关命令}
READ,LISTHDR,CHNHDR,WRITEHDR,READERR,COMMIT,ROLLBACK,RECALLTRACE

\SACTitle{最近修订}
Oct. 27, 1998 (Version 0.58)
