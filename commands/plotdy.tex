\section{plotdy}
\label{cmd:plotdy}

\SACTitle{概要}
绘制一个带有误差棒的图

\SACTitle{语法}
PLOTDY [ ASPECT ON | OFF ] [ PRINT pname ] name|number [ name|number ]

\SACTitle{输入}
\begin{itemize}
\item ASPECT  : ON 保持3/4的纵横比。OFF允许纵横比随着窗口维度的变换而变换 
\item PRINT {pname} : 打印输出到打印机 
\item name : 数据文件列表中数据名 
\item number : 数据文件列表中的数据号 
\end{itemize}

\SACTitle{说明}
这个命令允许你绘制一个带有误差棒的数据集。你选择的第一个数据文件(通过名字或文件号指定)将沿着y轴绘制第二个数据文件是dy值,如果选择了第三个数据文件则其为正的dy值

\SACTitle{例子}
假定你有一个等间距的ASCII文件,其包含了两列数据。第一列是y值,第二列是dy值,你可以像下面那样读入SAC并用数据绘制误差棒:
\begin{SACCode}
SAC> READTABLE CONTENT YY MYFILE
SAC> PLOTDY 1 2
\end{SACCode}

\SACTitle{错误消息}
\begin{itemize}
\item[-]1301: 未读入数据文件
\end{itemize}

\SACTitle{最近修订}
July 22, 1992 (Version 10.6f)
