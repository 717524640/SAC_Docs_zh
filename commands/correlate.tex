\section{correlate}
\label{cmd:correlate}

\SACTitle{概要}
计算自相关和互相关函数

\SACTitle{语法}
CORrelate [Master name|n] [Number n] [Length ON|OFF|v] [Type type]

其中type可以取 Rectangle|HAMming|HANning|Cosine|Triangle

\SACTitle{输入}
\begin{itemize}
\item Master name|n : 通过文件名或文件号指定某文件为主文件,其他文件将对这个文件相关
\item Number n : 设置要使用的卷积窗的数目
\item Length ON : 打开固定窗长选项开关
\item Length OFF : 关闭固定窗长开关
\item Length v : 打开固定窗长选项开关,并将窗长度设置为v秒
\item Typr Rectangle : 对每个窗应用一个矩形函数,这等价于不对窗加上函数
\item Type HAMming|HANning|Cosine|Triangle : 对每个窗应用xx函数
\end{itemize}

\SACTitle{缺省值}
CONVOLVE MASTER 1 NUMBER 1 LENGTH OFF TYPE RECTANGLE

\SACTitle{说明}
对于你申明为主信号的信号将做自相关,主信号与内存中的其他信号将计算互相关函数。这个命令的窗特性允许你计算一系列数据窗的平均相关函数,窗的数目可以选择,而且你有5个标准的窗函数可以选择。当这个窗特性被打开,对每一个窗将计算一个互	相关函数,然后这些互相关函数将被平均,剪裁到与原始数据文件相同的长度,并且取代内存中的文件。你也可以选择窗的长度。当窗长度(LENGTH选项)以及窗数目(NUMBER选项)超过数据文件中的点数(NPTS)时,SAC可以自动计算出窗口的混叠。缺省状态下这个窗特性是关闭的。

\SACTitle{例子}
利用内存中第三个文件作为主文件计算互相关函数:
\begin{SACCode}
SAC> CORRELATE MASTER 3
\end{SACCode}
你也可以通过文件名指定主文件。假设你有数据文件,每个包含1000个噪声点。为了使用10个包含100个数据点的窗(即没有窗的混叠)计算平均相关函数,每个窗应用一个HANNING窗:
\begin{SACCode}
SAC> CORRELATE TYPE HANNING NUMBER 10
\end{SACCode}
为了使每个窗有20\%的混叠,可以设置窗长度为120个数据点,假设采样间隔为0.025(即每秒40个采样点),这就意味着窗长为3秒:
\begin{SACCode}
SAC> CORRELATE TYPE HANNING NUMBER 10 LENGTH 3.0
\end{SACCode}

\SACTitle{头段变量改变}
DEPMIN, DEPMAX, DEPMEN

\SACTitle{致谢}
这个命令基于Dave Harris(DBH)开发的算法。

\SACTitle{最近修订}
Dec. 4, 1996 (Version 52a)
