\section{调用libsac库}
\subsection{next2}
SAC在做FFT时要保证数据点数为$2^n$个,对于不足$2^n$个点的数据需要补零至$2^n$次方个点。

next2函数定义为:
\begin{lstlisting}[style=C]
int next2(int num)  // 输入为num,返回值为大于num的最小2次幂
\end{lstlisting}

\subsection{xapiir}
xapiir用于设计IIR滤波器,并对数据进行滤波。这个子函数底层调用了design和apply两个
子函数。
\begin{lstlisting}[style=C]
void 
xapiir(float    *data,      // 待滤波的数据,滤波后的数据保存在该数组中
       int       nsamps,    // 数据点数 
       char     *aproto,    // 滤波器类型
                            //  - 'BU'  :   butterworth
                            //  - 'BE'  :   bessel
                            //  - 'C1'  :   chebyshev type I
                            //  - 'C2'  :   chebyshev type II
       double    trbndw,    // chebyshev滤波器的过渡带宽度
       double    a,         // chebyshev滤波器的衰减因子
       int       iord,      // 滤波器阶数
       char     *type,      // 滤波类型,'LP','HP','BP','BR'
       double    flo,       // 低频截断频率
       double    fhi,       // 高频截断频率
       double    ts,        // 采样周期
       int       passes     // 通道数,1或2
)
\end{lstlisting}

相关示例代码为~\lstinline{filterc.c}~和~\lstinline{filterf.f}~。

\subsection{firtrn}
\begin{lstlisting}[style=C]
void                                                                                
firtrn( char    *ftype,     // 类型,取'HILBERT'或'DERIVATIVE'
        float   *x,         // 输入数据                      
        int      n,         // 数据点数                                             
        float   *buffer,    // 临时数组,长度至少4297
        float   *y          // 输出数组
)
\end{lstlisting}

\subsection{envelope}
该子函数用于计算数据的包络函数,其底层调用了firtrn函数。
\begin{lstlisting}[style=C]
void                                                                                
envelope( int    n,     // 数据点数
          float *in,    // 输入数据
          float *out    // 输出数据
)
\end{lstlisting}

相关示例代码为~\lstinline{envelopec.c}~和~\lstinline{envelopef.f}。

\subsection{crscor}
该子函数用于计算两个数据的互相关,此互相关在频率域中完成,相对时间域互相关而言
效率更高。

\begin{lstlisting}[style=C]
void                                                                                  
crscor( float   *data1,     // 数据1
        float   *data2,     // 数据2                                       
        int      nsamps,    // 数据点数                                               
        int      nwin,      // 相关窗数目                                            
        int      wlen,      // 窗内数据点数,最大值为2048                                          
        char    *type,      // 窗类型,可以取'HAM','HAN','C','R','T'
        float   *c,         // 输出数据,长度为2*wlen-1
        int     *nfft,      // 相关序列的数据点数                            
        char    *err,       // 错误消息                                  
        int      err_s      // 错误消息长度
)
\end{lstlisting}

相关示例代码为~\lstinline{convolvec.c}、~\lstinline{convolvef.f}、~\lstinline{correlatec.c}~
和~\lstinline{correlatef.f}。
