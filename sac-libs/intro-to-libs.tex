\section{SAC库简介}
SAC提供了两个函数库:libsacio.a和libsac.a,用户可以在自己的C或Fortran程序
中直接使用函数库中的子函数。这些库文件位于\lstinline{sac/lib}中。

\subsection{libsacio.a库}
这个库文件的子函数可用于读写SAC数据文件、头段变量、黑板变量。这些子函数可以在用户
的C或Fortran程序中直接使用。

libsacio.a中可用的子函数如下:
\begin{itemize}
\item rsac1 读取等间隔文件;
\item rsac2 读取不等间隔文件和谱文件;
\item wsac1 写入等间隔文件;
\item wsac2 写入不等间隔文件;
\item wsac0 可以写等间隔文件或不等间隔文件;
\item getfhv 获取浮点型头段变量值
\item setfhv 设置浮点型头段变量值
\item getihv 获取枚举型头段变量值
\item setihv 设置枚举型头段变量值
\item getkhv 获取字符串头段变量值
\item setkhv 设置字符串头段变量值
\item getlhv 获取逻辑型头段变量值
\item setlhv 设置逻辑型头段变量值
\item getnhv 获取整型头段变量值
\item setnhv 设置整型头段变量值
\end{itemize}

对于C源码,用如下命令编译
\begin{lstlisting}[style=Shell]
$ gcc -c source.c -I/usr/local/sac/include
$ gcc -o prog source.o -lm -L/usr/local/sac/lib -lsacio
\end{lstlisting}
也可以利用SAC提供的sac-config命令简化此编译命令:
\begin{lstlisting}[style=Shell]
$ gcc -c source.c `sac-config -c`
$ gcc -o prog source.o -lm `sac-config -l sacio`
\end{lstlisting}

对于Fortran77源码,用如下命令编译
\begin{lstlisting}[style=Shell]
$ gfortran -c source.f
$ gfortran -o prog source.o -L/opt/sac/lib/ -lsacio
\end{lstlisting}
也可以利用SAC提供的sac-config命令简化此编译命令:
\begin{lstlisting}[style=Shell]
$ gfortran -c souce.f
$ gfortran -o prog source.o `sac-config -l sacio`
\end{lstlisting}

\subsection{libsac.a库}
这个库是从101.2版本才引入的,其是libsacio.a的超集,包含了几个数据处理常用的子函数。

libsac.a包含如下子函数:
\begin{itemize}
\item xapiir  无限脉冲响应滤波器;
\item firtrn  有限脉冲滤波器,Hilbert变换;
\item crscor  互相关;
\item next2   返回比输入值大的最小的2的幂次;
\item envelope 计算包络函数; 
\end{itemize}

对于C源码,用如下命令编译
\begin{lstlisting}[style=Shell]
$ gcc -c source.c -I/usr/local/sac/include
$ gcc -o prog source.o -lm -L/usr/local/sac/lib -lsac
\end{lstlisting}
也可以利用SAC提供的sac-config命令简化此编译命令:
\begin{lstlisting}[style=Shell]
$ gcc -c source.c `sac-config -c`
$ gcc -o prog source.o -lm `sac-config -l sac`
\end{lstlisting}

对于Fortran77源码,用如下命令编译
\begin{lstlisting}[style=Shell]
$ gfortran -c source.f
$ gfortran -o prog source.o -L/opt/sac/lib/ -lsac
\end{lstlisting}
也可以利用SAC提供的sac-config命令简化此编译命令:
\begin{lstlisting}[style=Shell]
$ gfortran -c souce.f
$ gfortran -o prog source.o `sac-config -l sac`
\end{lstlisting}
