\section{调用libsacio库}
\subsection{rsac1}
子函数rsac1用于读取等采样间隔的SAC数据。

函数定义如下:
\begin{lstlisting}[style=C]
void                                                                                
rsac1(  char    *kname,     // 要读入的文件名
        float   *yarray,    // 数据被保存到yarray数组中                                               
        int     *nlen,      // 数据长度                             
        float   *beg,       // 数据开始时间,即头段变量b                                             
        float   *del,       // 数据采样周期,即头段变量delta
        int     *max_,      // 数组yarray的最大长度,若nlen>max_则截断                        
        int     *nerr,      // 错误标记,0代表成功,非零代表失败              
        int      kname_s    // 数组kname的长度
)
\end{lstlisting}

相关示例代码为~\lstinline{rsac1c.c}~\footnote{代码位于~\lstinline{sac/doc/examples},下同。}
和~\lstinline{rsac1f.f}。

\subsection{rsac2}
子函数rsac2用于读取非等间隔采样的。
\begin{lstlisting}[style=C]
void 
rsac2(  char    *kname,     // 要读入的文件名                                                         
        float   *yarray,    // 因变量数组                                       
        int     *nlen,      // 数据长度                                           
        float   *xarray,    // 自变量数组
        int     *max_,      // 数组最大长度                                             
        int     *nerr,      // 错误标记                                         
        int      kname_s    // 数组kname的长度
)
\end{lstlisting}
相关示例代码为~\lstinline{rsac2c.c}~和~\lstinline{rsac2f.f}。

\subsection{wsac1}
子函数wsac1用于写等间隔SAC文件。
\begin{lstlisting}[style=C]
void                                        
wsac1(  char  *kname,       // 要写入的文件名                                                       
        float *yarray,      // 要写入文件的数组                        
        int   *nlen,        // 数组长度                                     
        float *beg,         // 数据起始时刻                                           
        float *del,         // 数据采样周期                                        
        int   *nerr,        // 错误标记                                        
        int    kname_s      // 文件名长度
)
\end{lstlisting}

相关示例代码为~\lstinline{wsac1c.c}~和~\lstinline{wsac1f.f}。

\subsection{wsac2}
写非等间隔SAC文件。

\begin{lstlisting}[style=C]
void                    
wsac2(  char  *kname,       // 文件名
        float *yarray,      // 因变量数组                                                   
        int   *nlen,        // 数组长度                                             
        float *xarray,      // 自变量数组                                               
        int   *nerr,        // 错误码                                             
        int    kname_s      // 文件名长度
)
\end{lstlisting}

相关示例代码为~\lstinline{wsac2c.c}~和~\lstinline{wsac2f.f}。

\subsection{wsac0}
子函数wsac0相对来说更加通用也更复杂,利用该函数可以创建包含更多头段的SAC文件。

\begin{lstlisting}[style=C]
void                                                                                
wsac0(  char  *kname,       // 文件名
        float *xarray,      // 自变量数组                                                 
        float *yarray,      // 因变量数组                                             
        int   *nerr,        // 错误码                                             
        int    kname_s      // 文件名长度
)
\end{lstlisting}

要使用子函数wsac0,首先要调用子函数~\lstinline{newhdr()}~创建一个完全未定义的头段区,并利用其它
头段变量相关子函数设置头段变量的值,并由wsac0写入到文件中。必须要赋值的头段变量
为delta、b、e、npts、iftype。

相关示例代码为~\lstinline{wsacnc.c}~、~\lstinline{wsacnf.f}。n=3,4,5。

\subsection{getfhv}
获取浮点型头段变量的值。
\begin{lstlisting}[style=C]
void                                                                                
getfhv( char  *kname,       // 头段变量名
        float *fvalue,      // 浮点型头段变量的值                                              
        int   *nerr,        // 错误码                                    
        int    kname_s      // 变量名长度
)
\end{lstlisting}

相关示例代码为~\lstinline{gethvc.c}~和~\lstinline{gethvf.f}。

至于如何获取和设置其它类型的头段变量,方法类似,不再多说。

\subsection{readbbf}
读取一个黑板变量文件。
\begin{lstlisting}[style=C]
void
readbbf( char   *kname,     // 要读取的文件
         int    *nerr,      // 错误码
         int    kname_s     // 文件名长度
)
\end{lstlisting} 

writebbf与之类似,不再列出。

\subsection{getbbv}
获取一个黑板变量的值。
\begin{lstlisting}[style=C]
void
getbbv( char    *kname,     // 黑板变量名
        char    *kvalue,    // 黑板变量的值
        int     *nerr,      // 错误码
        int     kname_s,    // 变量名长度      
        int     kvalue_s    // 变量值的长度
)
\end{lstlisting}
setbbv的函数定义与其类似,不再列出。
